\documentclass[article,msc=informatik,type=msc,colorback,accentcolor=tud9c]{tudthesis}




\usepackage{amsmath}
\usepackage[english]{babel}
\usepackage{float}
\usepackage{multirow}
\usepackage{lipsum,tabularx}
\usepackage{listings}
\usepackage{color}
\usepackage{glossaries}
\usepackage{subcaption}
\usepackage[printonlyused,withpage]{acronym}

   
\definecolor{dkgreen}{rgb}{0,0.6,0}
\definecolor{gray}{rgb}{0.5,0.5,0.5}
\definecolor{mauve}{rgb}{0.58,0,0.82}

\lstset{frame=tb,
  language=Java,
  aboveskip=3mm,
  belowskip=3mm,
  showstringspaces=false,
  columns=flexible,
  basicstyle={\small\ttfamily},
  numbers=none,
  numberstyle=\tiny\color{gray},
  keywordstyle=\color{blue},
  commentstyle=\color{dkgreen},
  stringstyle=\color{mauve},
  breaklines=true,
  breakatwhitespace=true,
  tabsize=3
}

\newcommand{\getmydate}{%
  \ifcase\month%
    \or Januar\or Februar\or M\"arz%
    \or April\or Mai\or Juni\or Juli%
    \or August\or September\or Oktober%
    \or November\or Dezember%
  \fi\ \number\year%
}

\begin{document}
  \thesistitle{A Mobile Honeypot for Industrial Control Systems }
    {}
  \author{Shreyas Srinivasa}
  \birthplace{Bangalore, India}
  \referee{Emmanouil Vasilomanolakis} {Prof. Dr. Max M{\"u}hlh{\"a}user}[]
  \department{Fachbereich Informatik}
  \group{Telekooperation \\ Prof. Dr. Max M{\"u}hlh{\"a}user}
  \dateofexam
  \tuprints{12345}
  \makethesistitle
  
  
  
  \newpage
  \begin{abstract}
   
  The number of attacks towards the cyberspace is constantly increasing with their complexity. Over the last years Enterprise networks are targeted with hazardous malware to exploit their data and communication. \ac{ICS} are responsible for distribution of basic necessities like electricity, water and gas. Such systems are highly complex, critical and huge.  \ac{ICS} systems comprise of sensory devices, controllers, real time units, command servers connected with an internal network. For better control and management, these devices are connected to the Internet.Failure of such critical systems lead to devastated environments and hence there is a need to secure them. 
  
  \vspace{3mm}
  \ac{ICS} systems are further classified to \ac{SCADA} systems. \ac{SCADA} forms a standard for majority of \ac{ICS} systems deployed worldwide. These systems have been targeted previously with malware such as Stuxnet. Stuxnet was capable of controlling the \ac{PLC} and infecting them with malicious code to change their behavior.  There is a need to detect the propagation of such malware. \ac{ICS} systems are hard to protect using passive systems like \ac{IDS} as they rely on a signature database. Also, strategized off-beat attacks cannot be detected using IDS. \textit{Zhung et al.} showed that although the detection accuracy of the \ac{IDS} is high the number of false positives are also high as these systems are sensitive to noise\cite{ye2004robustness}. An active mechanism that provides alert mechanisms on malicious packets and features an analysis framework with lesser false positives is necessary.
  
  
  \vspace{3mm}
  The advent of better smartphones has changed the way people interact and also leverage its resources to constantly provide better productivity to users.
  We propose HosTaGe \ac{ICS}, a mobile \ac{ICS} honeypot which actively detects attacks with respect to \ac{ICS} \ac{SCADA} systems.  HosTaGe, an Android based mobile honeypot is extended to HosTaGe \ac{ICS}, to accommodate \ac{ICS} profiles like a nuclear power plant. 
  
  
  \vspace{3mm}
  The protocols and the target systems that are specific to  \ac{ICS} \ac{SCADA} systems like Modbus and S7 are implemented to simulate the behavior of end systems. The behavior and response mechanisms are also implemented to facilitate the attacker to pursue his attacks. Malware detection is a crucial challenge that involves studying the behavior and its propagation techniques. We create the environment for the propagation of Stuxnet, a popular malware that can bring down \ac{ICS} systems and detect its propagation using HosTaGe \ac{ICS}. The results are further verified by the Virustotal~\cite{total2012virustotal} database. 
  
  
  \vspace{3mm}
  The attack strategies used by attackers nowadays involve distributed, single level protocol, multi level protocol/multistage  and payload level attacks. Multistage attacks refer to attacks that originate from the same source and attempt to exploit different types of protocols within a small window of time. It is one of the prevailing and effective strategies used by attackers these days to perform social engineering and compromise the end systems. During the course of the thesis, multistage attack patterns were observed. The detection of such attacks was incorporated into the design of our honeypot. HosTaGe \ac{ICS} features the detection mechanism for multistage attacks. Interesting results obtained during the evaluation period are briefly summarized in the Evaluation section.
  
  
  \vspace{3mm}
  HosTaGe \ac{ICS} incorporates a signature generator module where signatures are generated based on the attacks recorded with Modbus, \ac{S7} and HTTP protocols. The signatures that are generated by Hostage ICS can be incorporated into the Bro \ac{NIDS} to check incoming packets for malicious payloads. This feature enables HosTaGe ICS to be deployed inline with Bro \ac{NIDS} to provide better attack detection capabilities.
  
  
   \end{abstract}
  
  \newpage
  
  \tableofcontents
  \newpage
  
  
 
         
  \newpage
  \section{Introduction} \label{Introduction} 
   
   Power, water and gas form basic necessities for human life today. \ac{ICS} are used to automate the generation and distribution of these vital necessities. These are control systems used in industrial production that include SCADA\cite{boyer2009scada} and DCS\cite{butkovskiy1969distributed} systems. These systems manage critical infrastructure and are deployed in industrial sectors. \ac{SCADA} systems are used to control distributed assets using centralized data acquisition and supervisory control. DCS manage active production systems within a local area such as a factory using supervisory and regulatory control. PLCs are devices used for discrete control of specific applications and generally provide regulatory control for the critical machinery\cite{Webb:1998:PLC:551899}. \ac{PLC}s automate the control and management of critical systems in an \ac{ICS} environment.
   
   
   \vspace{3mm}
   \ac{SCADA} systems are distributed systems  used to manage and control geographically dispersed assets, often scattered over large distances. The data acquisition and control are centralized in these environments and are critically managed. \ac{PLC}s are major components in such operations.
   They are deployed to regulate the sensor information and provide an interface to convert signals from human-to-machine and vice versa. \ac{PLC}s are continuously being evolved in order to have better performance and simplified communication. The protocol responsible for communication between these devices is Modbus\cite{modbus2004modbus}. \ac{PLC}s today are able to communicate to the Internet through the advancements in the field of \ac{PLC} and Modbus-TCP\cite{swales1999open}. This makes the devices being controlled through the Internet. \ac{SCADA} systems were mainly designed for interaction between devices in a distributed network without considering security. This made the systems and the communication vulnerable. 
   
   
   \vspace{3mm}
   \ac{PLC}s handle critical systems, but have lesser computational resources as compared to standard desktop hosts. As modern day \ac{PLC}s have Internet communication capabilities, they are open for attacks from adversaries. This is due to the lack of security measures employed in securing both the device and the communication. As these devices have limited resources, ensuring security mechanisms may prove expensive. The manufacturers focused on providing better compatibility for communication rather than securing it. This vulnerability was exploited early by the attackers. In 2009, Stuxnet\cite{Langner:2011:SDC:1990763.1990881} a malware infected a nuclear enrichment facility in Iran. Stuxnet is still considered to be the most complex engineered malware until today for its stealthy and propagation techniques\cite{zetter2011digital}. It was considered to be powerful enough to cause a devastative impact. There are also recently discovered worms like Flame \cite{virvilis2013big}, that are observed to have similar impact. This opens up new challenges in securing \ac{SCADA} networks. 
   
   \vspace{3mm}
   Many enterprises and infrastructure use \ac{PLC}s today to automate processes. It is extensively used in airports, packaging industries, fuel stations, frozen food storage units and even in super markets. Securing the networks that use such devices is necessary. \ac{IDS} form front line defense systems against external attacks. They are widely deployed in enterprise networks to safeguard their internal networks. \ac{NIDS}\cite{1377213} are passive devices that  are used to monitor networks or systems in enterprise infrastructure. The detection mechanism is either anomaly or signature based. Signature\cite{wu2008signature} based IDS monitors the packets and compare them with a database of signatures. Anomaly\cite{garcia2009anomaly} based IDS detect intrusions by monitoring the system activity and classifying it as normal or anomalous. 
   
   
   \vspace{3mm}
   The signature based approach can be easily tricked by introducing varied payloads to defy the signatures  checked by the \ac{IDS}. A well strategized and tailored attack can easily pass through the system to gain access into the network \cite{kendall1999database}. A more active approach towards detecting strategized attacks with lesser false positives is the use of honeypots\cite{Provos:2004:VHF:1251375.1251376}. Honeypots are decoys that pose as target systems for the external viewer. It simulates specific services and reply mechanisms to emulate the complete system behavior. Any traffic directed towards honeypots can be considered as an attack. The number of false positives are comparatively lower in honeypots than in \ac{NIDS}. They can also work as an additional line of defense along with \ac{IDS} in traditional as well as in \ac{ICS} networks~\cite{winn2015constructing}.
   
   
   \vspace{3mm} 
   A study made by DELL\cite{DELLSecurityPoster2015} showed that the attacks on Industrial components like \ac{PLC}s doubled over the years, and even more dangerous is the fact that such incidents are unreported. The research found a hundred percent increase in attacks against \ac{ICS} environments like \ac{SCADA}.
   
        
    \begin{figure}[ht]
            \centering
            \includegraphics[scale=0.5]{chart1}
            \caption[SCADA Attack Types]{\label{f:SCADA Attack}SCADA attack methods\cite{DELLSecurityPoster2015}}
            \end{figure}  
            
            
   \vspace{3mm}         
   Figure~\ref{f:SCADA Attack} gives an understanding of the key \ac{SCADA} attack methods. It shows that about half of the total attacks were based on improper assignment on bounds of a memory buffer, improper input validation and vulnerabilities in credentials management. These vulnerabilities pose as a huge threat to \ac{ICS}. Figure~\ref{f:SCADA Hits} represents the number of attacks performed over the months. There is a steep increase in the number of attacks performed over the months, expressing the need to safeguard \ac{ICS} systems and also detect these attacks.
     
        \begin{figure}[ht]
           \centering
           \includegraphics[scale=0.6]{graph}
           \caption[SCADA Hits]{\label{f:SCADA Hits}SCADA hits on a monthly basis.\cite{DELLSecurityPoster2015}}
           \end{figure} 
     
      
   \vspace{3mm}   
   The majority of industrial systems today use \ac{SCADA} for controlling and automating their processes.
   Securing these devices are as much important like any other hosts in the network because these devices are programmable and could affect the normal automatized working. An overview of the above survey indicates various vulnerabilities and the need to secure the \ac{ICS} environment. As \ac{PLC}s are not only used in industrial infrastructure, but also in smaller environments where automation is required. It is very important and crucial to have less complex, usable and robust monitoring system. Such systems can be used to identify malicious packets and provide a flexible and analytical framework. 
   
   \vspace{3mm}
   Smart phones provide enormous computing capability with high flexibility.  They form a platform for developing light weight and reliable applications. These devices commute with the user and are highly available. They are able to connect to wide area networks and offer various modes of operational connectivity. We propose a lightweight, low interaction mobile based honeypot HosTaGe ICS which leverages the computing power and the above features of modern day smart phones to detect \ac{ICS} specific attacks. 
   
    
 
  
   \subsection{Contribution}\label{Contribution}
      
   This thesis aims at identifying and detecting the \ac{ICS} \ac{SCADA} specific attacks using HosTaGe \ac{ICS}, a low interaction mobile honeypot platform. The master and slave profiles, which are a part of \ac{ICS} infrastructure are simulated by emulating the constituent protocols. An analysis of the communication paradigm and the security loopholes of an  \ac{ICS} \ac{SCADA} system is made, to simulate the services offered by the system.   
  
  
  \vspace{3mm} 
   The thesis also concentrates on contributing to many security related research questions of  \ac{ICS} \ac{SCADA} systems like identifying \ac{ICS} specific malware , its propagation and devising a method to detect malware through our honeypot. A detailed study about the malware and its propagation is made to model the detection capability into HosTaGe ICS. 
   
   
   \vspace{3mm}
   During the course of the thesis, a pattern of attacks was observed and inferred to be Multistage attacks. We formally model the multistage attack strategy used by attackers to compromise the targets. Through an effective model, we further implement the detection of such multistage attacks. 
   
   
   \vspace{3mm}
   Honeypots today are used inline with \ac{IDS}s for increased detection accuracy. We also contribute towards this paradigm by a signature generation module that generates signatures for attacks received on specific protocols. The attacks identified through our multistage attack detection module are also leveraged to create policies for the Bro Intrusion Detection Monitor. The signatures and policies generated from HosTaGe ICS can be deployed on the Bro \ac{NIDS} for effective monitoring of malicious packets.   
   
   
      \vspace{3mm}
   Search engines are an effective tool for looking up resources in the Internet. Shodan a search engine provides its users with information regarding vulnerable devices in the Internet. Shodan probes the Internet to find vulnerable devices. The effective probing strategy of Shodan fetches as much information required to determine the vulnerabilities of a target system. Our honeypot setup received these probes during the evaluation period. We further investigate the impact of these probes on our honeypot and make interesting observations.
   
    
    \vspace{5mm} 
   
   \subsection{Outline}\label{Outline}
      
   The thesis mainly focuses on detecting attacks with respect to \ac{ICS} environment. The outline of the thesis is as follows. Section~\ref{Background and Related Work} provides a pre-study of \ac{ICS} \ac{SCADA} systems, security concerns with respect to \ac{ICS} \ac{SCADA} systems, the Modbus protocol  and honeypots. 
   
   \vspace{3mm}
    A study about the Siemens Simatic \ac{PLC} family is presented in  section~\ref{Siemens Simatic Series}.
    Section~\ref{Related Work} provides an overview of the related work in the field of honeypots and also \ac{SCADA} specific honeypots. Multistage attacks are briefly discussed in section~\ref{Multistage Attacks} and section~\ref{Signature Generation} giving an introduction to Signature Generation module of HosTaGe ICS.
   
   
   \vspace{3mm}
   Section~\ref{Proposed System and System Design} proposes a solution with a system design involving discussion of HosTaGe as an \ac{ICS} perspective in section~\ref{HosTaGe ICS Perspective}.  The protocols, formal models  and detection mechanisms are briefed in their corresponding sections.
   
   
   \vspace{3mm}
   We present the implementation ideas of HosTaGe ICS in section~\ref{Implementation} as a honeypot with discussion related to detecting internal attacks, malware detection, multistage attacks in the forthcoming sections. The implementation of the signature generation module is elaborated in section~\ref{Signature Generation Module}. 
   
   
   \vspace{3mm}
   The evaluation of the thesis is discussed in section \ref{Evaluation}. It focuses on evaluation of a single protocol attack and detection of Multistage signatures. The impact of Shodan probes and the observation is open in section \ref{Shodan Evasion}.   
   Section~\ref{:Performance Evaluation} provides an overview of the performance of HosTaGe ICS as an Android application. The limitations of our work  are discussed in section~\ref{Limitations}.
   
   
   \vspace{3mm}
    Section~\ref{Conclusion} aggregates the work done on HosTaGe and the future enhancements are specified in section~\ref{Future Work}.
     
     
     
   
   \newpage  
   \section{Background and Related Work}\label{Background and Related Work}
   
   The following sections provide a background on \ac{ICS} \ac{SCADA}, its security perspectives, the Modbus and S7  protocols, Siemens Simatic S7 \ac{PLC}  and honeypots. Section~\ref{Related Work} discusses the related work in the area of honeypots, further classified to mobile honeypots and SCADA honeypots. 
   
   \subsection{Background } \label{Background}
    
   \ac{ICS} form a dominant portion in present day industries.  \ac{ICS} components include actuators, sensors, networking devices, controlling systems and \ac{PLC}s . The sensors form a major portion of \ac{ICS} as they provide continuous feed of critical information which is used to automate and control other systems. The \ac{PLC} is another such important component for the \ac{ICS} . This interface allows a programmer to implement a logic to automate the systems based on the data received from sensors. There are mainly two different types of \ac{ICS}. The major type is \ac{SCADA}, which are deployed on geographically widespread and are controlled using a central location. Examples include nuclear power plants, water and power distribution where there is a need for constant monitoring and critical automation.  The second type of \ac{ICS} is the \ac{DCS}. On the contrary, \ac{DCS} are not centralized, but distributed across a network. We focus more on \ac{ICS} \ac{SCADA} systems as they are being deployed in major infrastructures as a standard today.
    
   The \ac{SCADA} infrastructure have a lot of components and devices which need constant communication between them. The \ac{ICS} \ac{SCADA} systems involve the interaction of many sub-modules and devices that will be discussed below in the following sections.

	\vspace{5mm} 
	\subsubsection{ICS SCADA} \label{ICS SCADA}

	\ac{SCADA} is an industrial automation control system at the core of several industries today including energy, oil and gas, power, water and recycling, manufacturing and many more. \ac{SCADA} provides the benefit of, simple configuration and usability. 
	
	
	\vspace{5mm} 
	The basic architecture of \ac{SCADA} involves communication of information from sensors or manual inputs to \ac{PLC}s or \ac{RTU}s. \ac{PLC}s process the information as per the logic deployed in them and then forward this information to workstations/servers running \ac{SCADA} applications. Figure~\ref{f:SCADA Architecture} describes the basic architecture of a \ac{SCADA} system which consists of sensors, \ac{PLC}s, master hosts and the controller hosts. Data is fetched from the sensors and is processed on \ac{PLC}s. The \ac{HMI} displays this processed data to the user. 

	 \begin{figure}[ht]
	 	
        \centering
        \includegraphics[scale=0.85]{SCADA1}
        \caption[SCADA Architecture]{\label{f:SCADA Architecture}SCADA Architecture }
        \end{figure}


	\vspace{3mm}
	\ac{SCADA} systems involve control and network components. The following are the list of control components in \ac{SCADA} systems:

	\begin{itemize}


	\item\textbf{\ac{RTU}:}These devices are time constraint hardware  that fetch data from sensors and convert it to digital data. The converted data is further sent to the master system or the supervisory system. \ac{RTU}s receive commands from the supervisory systems. The devices possess arithmetic processors to perform logical operations on the data received from sensors.
	


	\item\textbf{\ac{PLC}:}
	These devices fetch data from sensors that are embed to heavy machinery such as spinning centrifuges. Further, they convert the analog data received to digital data. \ac{PLC}s possess higher processing power than \ac{RTU}s to perform complicated operations on the data received. The controllers can embed logic into \ac{PLC}s for a specific operation.
	\ac{PLC}s are preferred over \ac{RTU}s in certain critical environment due to their flexibility and customization features. \ac{PLC}s can be made compatible for interaction with devices from different manufacturers. These devices also posses expansion slots for scalability. 
	
	
	

	\item\textbf{Telemetry systems:} They are used to connect \ac{PLC}s and \ac{RTU}s with supervisory system centers, and controllers in enterprise networks. In a typical scenario, they include leased lines, LAN cables with switches and  \ac{WAN} circuits lines. Some networks also have wireless transmission capabilities. Such networks have wireless telemetry via satellite uplink and broadband radio of microwaves. 
	
	

	\item\textbf{Data and Control Server:} Data acquisition systems are software that are used to achieve connectivity to the device interfaces through telemetry lines with the deployed \ac{PLC}s and \ac{RTU}s. The software enable users to access remote device data from field devices through \ac{SCADA} protocols.
	
	
	
	\item\textbf{\ac{HMI}:} The \ac{HMI} is responsible for presenting the processed data in human readable form. Users interact with the field devices and controllers through \ac{HMI} to issue commands and control them. The \ac{HMI} acts a client interface to receive data from remote data acquisition servers. 
	
	
	
	\item\textbf{Historian software:} This software stores and represents the data graphically for statistical evaluation. It contains a database that stores data which is time stamped. The data can also be stored as a part of event handling. This software is also capable of retrieving data from the data acquisition servers.
	



	\end{itemize}
	Different network characteristics exist for every layer within the control systems. The network topologies vary with vendors or manufacturers and also on different implementations. Modern day \ac{SCADA} systems are open to the Internet and as a result, enterprise integration can also be achieved. The control networks work in hand with the corporate enterprise networks to manage more efficiently and control the systems  from outside networks. The following are the major network components of an \ac{ICS} network:

	\begin{itemize}


	\item\textbf{Fieldbus Network:} This is used to achieve connectivity between \ac{PLC}s and other controllers in the network. The need for point to point cabling is avoided between the controller and every slave using the fieldbus network. Fieldbus supports various communication protocols. Every device has a unique identifier for the messages sent. This way there is no collision of messages during processing.
	
	
	\item\textbf{Control Network:} The controller networks connect the supervisory systems and master systems to the slave devices. 
		

	\item\textbf{Communications Routers:} The basic functionality of a router is to transmit messages between networks. Routers in \ac{SCADA} systems are used to connect \ac{LAN} to a \ac{WAN} and also connecting huge enterprise networks to bridge long-distance networks. They also connect field devices to outer networks for command and control. 
	
	 
	\end{itemize}

	\ac{SCADA} applications help in monitoring and analyzing the data. The application help the device controllers and operators to work efficiently. Modern \ac{SCADA} systems allow real time data from the plants to be accessed from anywhere in the world. This also means that it provides attackers an opportunity to exploit the data and its availability. Exploiting \ac{SCADA} systems can cause catastrophe as it may result in huge damage to the environment and people in the plant. We try to identify the attacks and exploits  and detect them using our mobile honeypot.
	
	\vspace{5mm} 

	\subsubsection{Security Overview of SCADA ICS}\label{Security Perspective of SCADA ICS}

	\ac{ICS} \ac{SCADA} systems are highly distributed. They are used to control and manage geographically dispersed plants, often scattered over thousands of kilometers. In these areas, centralized data acquisition and control are critical to system operation. They are applicable in circulation systems such as water distribution and wastewater collection systems, oil and natural gas pipelines and electrical power grids on systems. A \ac{SCADA} control center provides centralized monitoring and control for long distance separated sites with monitoring systems. The systems also process the data acquired from these sites. The data received from remote sites that are driven by automation or by an operator,  supervisory commands can be pushed to remote station control devices, which are often referred to as field devices. These devices manage local operations such as switching control units and pumps, retrieving data from sensors and local environment monitoring.
	
	\vspace{3mm}
	
	The devices at the field sites are controlled and managed through the control center. This requires a secure and robust communication line between the control center and the field site. This network is critical. This is usually established through the Modbus TCP/IP over the Ethernet. It is usually advised \cite{stouffer2011guide} to place the \ac{SCADA} devices on a network that is not physically connected to any other networks. 
		
		\vspace{5mm}
	 \textit{Bailey et al.} \cite{conf/bwcca/DissoJB13} proposed honeypots as a solution to \ac{SCADA} security monitoring. \ac{SCADA} systems today have evolved from complex infrastructure to simple and well packaged network architecture. This makes \ac{SCADA} systems usable for critical systems as it is simple to manage. Further, the threat landscape of \ac{SCADA} is huge due to the various critical environments under which it is deployed.  Below are some of the underlying threat areas that \ac{SCADA} systems are subjected to today.
	 \vspace{5mm}
	 \begin{itemize}
	 
	 
	 \item\textbf{Supply Chain:} A product is engineered and manufactured by a group of vendors in different manufacturing units. The final product is aggregated from all the units later for assembly before it is provided for the requester. This makes some room for unintentional error and also induced error like, the insertion of  malicious functionality with respect to a specific customer, usage of low quality, less secure electronics and finally, poor engineering practices. The errors may also occur at the final assembly parts where these parts are not assembled or tampered for specific induced intentions. The requesters have no idea of these errors until the casualties occur as they do not have quality check units to investigate the products once they are received or deployed. 
	 \vspace{5mm}
	 \item\textbf{Vendor Backdoors:} These are the most serious threats that are hard to detect. Vendors install backdoors\cite{SANSVendor} on products to gain access to control end systems. Backdoors sometimes also have covert data leaking channels to remote command and control servers. These backdoors are not documented in any catalogs provided to the customers. Securing from such attacks is very difficult as they are not accounted for. In such cases honeypot acts as a good solution as it is capable of storing all the connection activities which make later investigations possible.
	 \vspace{5mm}
	 \item\textbf{Protocol Vulnerabilities:} The protocols that are associated with SCADA like Modbus and S7 communication were not designed keeping security as a requirement. They do not support authentication, encryption and non-repudiation as they were proved to be expensive operations on the data communicated. In such environments, unauthenticated hosts can form legitimate packets and establish connection with any devices. Many of the protocol interaction takes place in plain text thereby making the communication completely visible. Unauthenticated systems running tools like sniffers and monitors can misuse this for knowing and probing the environment.
	 \vspace{5mm}
	 \item\textbf{Malware Propagation Techniques:} SCADA systems form the backbone of \ac{ICS}. The initial infrastructure of the intranet involves different components and devices as discussed in section~\ref{ICS SCADA}. Through continuous probing and sniffing techniques, adversaries learn more about the infrastructure components and also the interaction between systems. This is a useful technique to design strategies for malware propagation in SCADA networks. It could also be called Social Engineering \cite{Thornburgh:2004:SED:1059524.1059554}.
	 
	  The other form of designing strategies would be the use of attack trees to assess and determine the vulnerabilities and flaws of a target system. This was initially proposed by \textit{Bruce Schneier} \cite{schneier1999attack}. Attack trees provide a structured and flexible means of pursuing analysis on protocols, applications and networks. They form a strong representation of the attack strategies designed for compromising systems in a \ac{SCADA} network communication using the Modbus protocol. The research by \textit{ Eric J. Byres et al.} \cite{byres2004use}provides a deeper insight of attack trees for Modbus protocol. We discuss the vulnerabilities of Modbus protocol in further sections. 
	 \vspace{5mm}
	 \item\textbf{Botnets and Search Engines:} The Modbus-TCP protocol enabled ethernet communication capability to \ac{SCADA} systems. As a result \ac{SCADA} systems are able to connect to the Internet. Google is the most widely used and popular search engine. Through crafted Google queries, critical information can be retrieved .This information might be some accidental information revealed by employees about the company network infrastructure or critical information. This technique is referred to as Google Dorks. \textit{Trend Micro}, a security software company provides an overview of how Google Dorks revealed critical information of \ac{ICS} systems\cite{wilhoit2013s}.  
	 
	 Search engines like Shodan which is acclaimed to be the scariest search engine \cite{goldman2013shodan}on the Internet expose vulnerable devices that are connected to the Internet.  Shodan is specialised in indexing the headers in contrary to the other search engines that match the content of the page. Shodan database is like a goldmine for attackers which provides enough information like the services, protocol information, \ac{IP} subnets and also location of the systems. \textit{Roland et al.} \cite{bodenheim2014evaluation} provide an evaluation of Shodan's  ability to identify Internet-facing \ac{ICS} devices.
	 
	
	 Bots constantly probe the Internet for vulnerable devices. The information retrieved by bots are sold in the black market. These prove to be very effective bots with fewer false positives. Botnets \cite{ianelli2005botnets}, a networked collection of bots can be rented specifically to perform such activities. Botnets also act as strong probing tools to identify vulnerable \ac{SCADA} devices connected to the Internet.
	 
	 
	 \vspace{5mm}
	 \item\textbf{Espionage and Cyber Exploitation:} Amidst adversaries, there are also espionage planned by governments to keep track of developments on developed countries. This motivation vary from simple curiosity to national level attacks. An example would be of the Stuxnet worm which was discovered to be developed by America and Israel together and was deployed in a nuclear enrichment facility in Iran. The actual motive is unclear but by the aspects it may look a cyber espionage designed by one of the nations. It is very important to note that malware like Stuxnet remained undetected over a long duration and was capable enough to create a environment hazard. Hence, these kind of threats cannot be ignored and rather must be focused in order to ensure better security. 
	 
	 \end{itemize}
	 
	 In addition \textit{Igure et al.}\cite{igure2006security}  specify three challenges that must be focused to secure \ac{SCADA} networks. The first challenge is to improve the access controls
	 to the \ac{SCADA} networks. An access control solution will ensure restricted access for an attacker to enter into the \ac{SCADA} network. The second challenge is to improvise the security in the \ac{SCADA} intranet and to develop efficient security-monitoring tools.
	 The third challenge involves improvising the overall management of the devices in the \ac{SCADA} network by identifying loopholes and vulnerabilities.  These security mechanisms ensure better safety in the \ac{SCADA} network. The monitoring tools like \ac{IDS} identify malicious packets and also monitor the network for suspicious activity. 
	 
	\vspace{3mm}
    The above specified threat areas provide an insight into the security perspective of \ac{SCADA} systems.
    Further as specified, we discuss the Modbus protocol which forms the backbone of \ac{SCADA} systems and its security aspects.
    
    \vspace{5mm} 
    	\subsection{Modbus Protocol}\label{Modbus Protocol}
    
    	Modbus denoted by \ac{IETF} \ac{RFC} 2026 is a serial communication protocol. It was first published by Modicon for communication in its \ac{PLC}s. Modbus is now a standard that connects industrial devices together. The basic configuration involves connecting a \ac{SCADA} supervisory control system to a \ac{PLC} or \ac{RTU}. The data types are derived from its use in production devices where a single-bit physical output is called a coil, and a single-bit input is discrete input or a contact. The device requesting the information is called the Modbus Master and the devices rendering the data are Modbus Slaves. In a standard Modbus network, there is one Master and up to 247 Slaves, where each Slave device has a unique address from 1 to 247. The Master can also write data to the Slaves.
    	Modbus TCP/IP specification was introduced to Modbus to integrate corporate intranet with \ac{PLC} systems. This made the network better manageable, scalable and also cost-effective. Modbus TCP/IP offers many advantages:
    
    	\begin{itemize}
    
    	\item\textbf{Simplicity:} The \ac{TCP} is wrapped with the  Modbus instruction set. The setup involves simple driver initialization at end devices for communication. Low development cost, hardware and compatibility with different \ac{OS}s makes it simple.
    
    	\item\textbf{Standard Ethernet:}  Ethernet ingrates easily into simple chipsets and boards. The cost of implementing Ethernet to Modbus is low and also provides ample resources as there are many developers working on optimizing the technology. Ethernet port 502 is used by the Modbus TCP/IP protocol.
    
    	\item\textbf{Open:} The Modbus protocol has been open source since 2004 and has a dedicated organization working towards development,optimization and maintenance.
    
    	\item\textbf{Compatibility:} Modbus provides interoperability among various vendors and also compatibility with devices of other manufacturers. 
    
    	\end{itemize}
    
    	Modbus TCP/IP is an Internet protocol. This makes the devices open to the Internet. This was a particular feature that was incorporated to facilitate better control and making device maintenance through remote systems over the Internet. Modbus is an industrial networks protocol. Modbus TCP/IP helps in better management of distributed \ac{ICS}. It is also necessary to understand the security aspects of Modbus as \ac{SCADA} systems rely on it for communication between field devices. The underlying security issues of Modbus are discussed below:
    	
    	\vspace{5mm}
    	
    	\begin{itemize}
    	
    	\item\textbf{Lack of Confidentiality:} The Modbus traffic is transmitted in plain text and does not involve any encryption. This makes man-in-the-middle attacks easily possible. The traffic can be viewed using simple sniffing and traffic monitoring tools. 
    	
    	\item\textbf{Lack of Integrity:} There are no integrity checks build into the Modbus application and as a result it relies on lower level layers to preserve integrity. 
    	
    	
    	\item\textbf{Lack of Authentication:} There is no authentication at any level of the Modbus protocol, with the exception of some commands that explicitly introduce an authentication mechanism. It is observed from vulnerability analysis that this authentication mechanism is hard-coded and can be easily retrieved by simple packet forging techniques. 
    	
    	
    	\item\textbf{Simplistic Framing:} The Modbus-TCP frames are carried over established TCP connections. Although TCP is a reliable protocol for data  transmission, the drawback of not preserving record boundaries for the Modbus protocol, makes it easy to inject arbitrary frames. This provides an opportunity for man in the middle attacks. 
    	
    	
    	\item\textbf{Lack of Session Structure:} Modbus-TCP has short span sessions like SNMP and HTTP protocols. The master host initiates a request to the slaves which results in a single action. The lack of authentication and poor TCP \ac{ISN} generation in embedded system frameworks, enables attackers to inject packets without caring for existing sessions.  
    	 	
    	\end{itemize}
    
    \vspace{5mm} 
    
    
    \subsection{S7 Protocol}
    
    
    The SCADA systems were developed basically for simplified serial communication considering that the devices operating on it are legitimate and are programmed to perform the logic specified. Thus, there is no actual security measure to enforce secure communication. There is no authentication check to verify if the data received is from a valid source. Moreover, all the data is in plaintext making it possible for any machine in the network to read all the intercepted data.
    
    \vspace{3mm}
    The Siemens S7 protocol is widely deployed in \ac{SCADA} networks in the Siemens S7 \ac{PLC}s, which are the leading controllers used in \ac{ICS} environments. The S7 protocol structure is complex when compared to the Modbus protocol as it allows parallel access to multiple variables of different data datatypes. 
    Related information regarding the protocol structure of Siemens S7 protocol is hard to find as it is a proprietary protocol designed by Siemens for communication between its \ac{PLC}s. However, we try to fetch as much information regarding this protocol by the traffic traces found on the Internet \footnote{https://wiki.wireshark.org/S7comm} and some related information.\cite{kleinmann2014accurate} 
     
     
     \vspace{3mm}  
    The Siemens Simatic S7 product line was introduced in 1995 for models S7-200, s7-300 and S7-400 and later for models S7-1200, S7-1500. The Step 7 software is responsible for the \ac{HMI} of these \ac{PLC}s. TCP/IP based connectivity for the protocol is provided by the Siemens Ethernet Driver. However, connectivity can also be achieved through third party drivers. The protocol communicates over \ac{TCP} port 102 and interacts over the \ac{COTP} and TPKT\cite{pouffary1997iso}. These protocols add their own headers in the TCP segment and thereby encapsulate the S7 packet within the \ac{COTP} packet. 
     
     
     \vspace{3mm}
    S7 provides a common communication interface where the device can choose to be clients, master or peers. There are two implemented versions of the S7 protocol. Simatic S7 \ac{PLC}s implement the S7 implmentation and the later devices implement the \textit{0x32} variant. A newer variant \textit{0x72} introduces some security features. 
    
    
     \vspace{3mm}
    The maximum length of a S7 \ac{PDU} lies between 112 to 960 bytes. The packet is divided into three parts. The first part is a fixed header which includes the \ac{COTP}, TPKT and \ac{TCP} header. The second part is the parameter that indicates the \ac{PLC}s variables that are accessed and lastly the data part which includes the data to be written. 
    
    
    \vspace{3mm}
   The S7 protocol does not provide security features similar to Modbus. The protocol was designed to establish communication between Siemens Simatic \ac{PLC}s. The protocol is also not available as open source for a detailed study. The packets sent through ISO TSAP \cite{rose1987iso} are in plaintext. These packets can be easily forged by capturing them and modifying it to exploit the \ac{PLC}s. Thus the attacker can intercept any packet and modify it to issue commands like powering off the CPU and changing the ladder logic of the \ac{PLC}. This also enables an attacker to secretly place a code which acts as a backdoor that helps is gaining remote access to the \ac{PLC} directly. A man-in-the-middle attack is also possible by intercepting the packets that flow from the software to the \ac{PLC}.
   \textit{Dillon Beresford } in his paper \cite{beresford2011exploiting} shows various ways a Siemens Simatic S7 \ac{PLC} communicating through a S7 protocol can be compromised. In his report, he elaborates the vulnerabilities of Siemens S7 \ac{PLC}s and simplistic tools to exploit them. The five attacks portrayed are:
   
   \begin{enumerate}
   
   
   \item The TCP Replay attack over the ISO TSAP
   \item Bypassing the basic authentication of S7 protocol
   \item Powering the CPU on and off of the \ac{PLC}s.
   \item Read and write of the memory registers.
   \item Gaining shell access on the \ac{PLC}
   
   
   \end{enumerate} 
   
   
   The exploit is performed using penetration testing tools like Metasploit \cite{maynor2011metasploit} and Nmap \cite{lyon2009nmap}  that help in providing pre-defined modules for the attack. Listing~\ref{lst:Nmap Modbus Discovery Script} shows the Modbus discovery script for the detection of the Modbus service status on the target system. HosTaGe ICS received many attacks which were based on this script. Listing~\ref{lst:Metasploit Modbus Detect Script} shows the Metasploit script for detecting the Modbus service on the target system. Listing~\ref{lst:Modbusclinet} shows the Metasploit script that can be used to fetch the values stored in Modbus registers. The script can also be used to push values into the Modbus registers and coils.
   We also leverage the capability of these tools to study the attack and probing strategies to model out attack detection vectors. 
   
   
   
   \subsection{Siemens Simatic Series}\label{Siemens Simatic Series}
   
   	The Siemens S7 200 is a micro-programmable logic controller which can control a wide variety of devices to support various automation needs. The S7-200 features monitoring, inputting values and outputs as validated by the logical program, which can include Boolean logic, counting, process timing, complex mathematical operations, and communications with other intelligent devices. It can control and communicate with devices like automatic pressure controllers, centrifuge pumps and water cooling systems. The STEP 7 programming package provides a user-friendly application programming interface to develop, edit, and monitor the logic needed to control the application that monitor devices. The Siemens Simatic S7 \ac{PLC}'s use PROFINET which is based on Ethernet for communication. There are over 3 million PROFINET devices deployed worldwide. 
   	
   	
   	\vspace{3mm}
   	Siemens S7 200 \ac{PLC}s boasts of a compact design, powerful performance, optimum modularity and open communications. This Micro \ac{PLC} has been in successful use in millions of applications around the world,in both standalone and networked solutions. 
   	
   	
   	\vspace{3mm}
   	This \ac{PLC} uses communication protocols such as PROFINET, an advanced version of Modbus communication protocol. This protocol is also based on Ethernet. It also supports TELNET, \ac{HTTP}, \ac{FTP}, \ac{SNMP}, \ac{SMTP}, Modbus and \ac{S7} protocols. Though this \ac{PLC} is designed to be used to control critical systems, security was not a part of its design. The above mentioned protocols were not customized to facilitate secure communication. The standards were defined to create an interconnected environment between industrial automation devices and common networking protocols.Security was either ignored or rather was thought to be expensive on these devices. This makes it an easier target for attackers. 
   
   
   	\vspace{3mm}
   	The Simatic S7 \ac{PLC} is also subjected to various vulnerabilities and attacks including the Stuxnet as discussed earlier. We simulate the Siemens Simatic S7 200 \ac{PLC} as a target system in our honeypot.  
   
     
	\subsection{Honeypots} \label{Honeypots}


	A honeypot  is a decoy server or a system in a network which is closely monitored for adversaries. It is also defined as:
	\textit{A honeypot \cite{spitzner2003honeypots}  is an information system resource whose value lies in unauthorized or illicit use of that resource}. They are mostly deployed inside firewalls, but they could be deployed in any part of the network. It is designed to be a system with vulnerabilities and services that are offered by a real target system. Any attempt to connect to these systems could be considered as an attack. All the activities are logged and further traced. The general idea is that once an adversary detects a vulnerable system and tries to attack it, he would come back with more sophisticated attacks. The initial part of discovery and knowing the general services and loopholes is called system social engineering.
	Honeypots provide active monitoring components that wait for attacks and respond to the attacks by luring the attacker to pursue more.
	\vspace{5mm} 
	There are certain main functionalities that the honeypots must possess in order to perform their main functionality. 

	\begin{enumerate}
	\item Honeypots must simulate the system that they are intend to focus on. This gives the attacker a feeling of approaching a real system. The honeypot may simulate the complete functionality of the system or just the services offered by the system. 

	\item A proper response mechanism which keeps the attacker engaged to the honeypot. This facilitates better logging of the attack and also provides more data to analyze the attacks. 

	\item It mainly has three perspectives. First, an attacker perspective, by posing as a vulnerable system; second, an administrator who can identify and log the attacks made by the attacker and third, being able to present and analyze the attacks logged by the administrator. 

	\item Honeypots must not induce additional load on the infrastructure. Honeypots must be designed to be lightweight and having no influence on the production systems. 

	\item Honeypots are basically exposed to exploits, threats and malware. It is very important to see that this data is not leaked through the network which can later infect other systems in the network.

	\item Based on the previous condition, honeypots must be robust to withstand the attacks and exploits and not fail.

	\end{enumerate}

	\vspace{3mm} 
	It is clear that honeypots are valued because of the interaction mechanism that they provide for any communication request. They can be used to study and gather exploits, malware and threats in an early attack phase. There are many advantages that one could consider for using honeypots as an additional security monitor. There are various advantages of honeypots:

	\begin{itemize}

	\item\textbf {Effective Data Sets:} Honeypots collect data only when there is a communication requested with it. The data collected at honeypots may not be immense but is good enough to analyze and detect attacks. The logs provide information about the attacker \ac{IP}, time of attack and protocol used to carry out the attack. This ensures lesser false positives.

	\item\textbf {Reduced False Positives:} Among other security approaches like IDS and firewalls, false positives are quite common. The biggest challenge is to reduce false positives. Honeypots  could be designed to reduce false positives. Any communication with the honeypot is unauthorized. This makes honeypots efficient in detecting attacks.

	\item\textbf {Catching False Negatives:} Honeypots have advantages over signature based detection systems. Signature based systems do not categorize unknown attacks. They rely on a signature system to be updated on their local database to identify and detect unknown attacks. The probability of a detecting a new exploit is low. Honeypots detect all attacks irrespective of their signatures, thereby increasing the possibility of detecting new attacks. 

	\item\textbf {Encrypted Communication:} The current standards in Transport layer includes using encrypted \ac{TLS} communication between nodes. Some attacks fail to detect because of the encrypted data and communication. All enterprise networks employ secure protocols like \ac{SSH},\ac{IPSec}, \ac{HTTPS}, \ac{TLS} in their infrastructure. This may cause problems in detecting exploits and analyzing the attacks later. Honeypots solve this issue as they are end points in the communication. The hosts directly interact with the node and hence all the traffic and data can be decrypted and analyzed later.

	\item\textbf{Compatibility to new architecture:} Technology evolves every moment. It is very essential to consider future compatibility with newer standards and technology. Most of modern day IDS or firewalls are not compatible with \ac{IPv6} which promises to be the next standard on Internet addressing. Honeypots can be made compatible to newer standards and technology as they are not mediators or devices but act as end points. However, devices could be simulated by honeypots. 

	\item\textbf{Flexibility:} Honeypots can be deployed locally or open to the external network. Honeypots could be deployed on any environments based on the requirements such as specific servers, host systems or protocol level emulation. It could be used to simulate any software, hardware, servers, workstations and devices. 

	\item\textbf{Minimal Resource Consumption:} Honeypots can run on low resource machines as they are just simulations and are may not depict full functionality of the system simulated. Honeypots today can run on smartphones as they possess the required resources which are good enough to run a honeypot.

	\end{itemize}
	
	
	\subsection{Related Work} \label{Related Work}
	There has been extensive research going on in the field of honeypots and honeypot-based Signature Generation. 
	We also discuss multistage attacks, an attack strategy employed by attackers towards attacking multiple protocols. Multistage attacks are effective in obtaining more vulnerabilities of the system at the protocol level. Detecting these attacks is crucial as it forms a basis for analyzing strategic attacks.  In the following sections we discuss related work in these areas.
	
	\vspace{3mm} 
	 \subsubsection {Types of honeypots }\label{Types of honeypots}

	Honeypots can be classified into two types based on the attacker ability to interact with the application or services. They can be categorized to \textit{high-interaction honeypots} and \textit{low-interaction honeypots}. This classification is mainly based on the honeypot's interaction with the attackers. High interaction honeypots typically composed of the actual device, its operating system and all the applications that run on that device. In short, the exact machine is used as a honeypot with all its services. This provides better interaction as we are using the device itself as a honeypot. There are also better chances that based on the vulnerability known, all the exploits work on the device. The main advantage of such honeypots is that it is the machine itself being exposed and has greater chances of attracting attackers. The disadvantage would be that if the honeypot is completely compromised, then it has to be rebuilt in order to log other attacks. The validity of such honeypots is not guaranteed. 
	
	\vspace{3mm}
	A low interaction honeypot on the other hand is a software based or simulation based honeypot approach. The system to be subjected to an attack is simulated by the honeypot along with its main services. The honeypot can run on any system, for example it can run on a Linux machine and simulate a honeypot for a Windows \ac{IIS} server. It can simulate or mimic the network stack and the operating system of the target host. All connections and communication with this device is logged. The advantage of low interaction honeypots is that they are completely flexible and easy to maintain. Low interaction honeypots are also likely not to get compromised as they just mimic the services or in short the basic communication mechanism. It is on the researcher to design these honeypots accurately to get productive results.

	\vspace{5mm} 
	\subsubsection  {Honeynets} \label{Honeynets}

	Honeynets \cite{holz2008virtual} are a networked collection of honeypots that look like common network services and servers. They provide an overview of the simulation of server and network components. This overview provides better modeling of the honeypot and the services to be simulated. The simulation could be a collection of honeypots depicting as a Domain Controller, web server, application server, file server and so on which provide a facade of a enterprise network. Honeynets 	usually consist  of high -interaction honeypots, low - interaction honeypots, or a combination of both. Using high interaction honeypots only for this approach would be more expensive.
	Honeynets are placed behind a Honeywall , which acts as a bridge to the honeynet. It includes network monitoring, packet capture, and IDS capabilities. Honeynets provide an insight into design of low interaction honeypots along with other devices in the network like \ac{NIDS}.
	
	
	\vspace{3mm}
	Honeynets require complicated setup and resources to deploy them to the network. It also involves complex configurations as it pertains to simulation of many target hosts through honeypots and a separate logging mechanism. However, Honeynets provide an idea of simulating the infrastructure and also providing a framework for further analysis. We look forward to implement a similar framework  for the analysis of malicious packets in our honeypot. 
	
	
	\vspace{5mm} 
	\subsubsection  {Mobile honeypots}\label{Mobile honeypots}

	Smart phones have huge computing capability  to host resource intense applications.  The phones also have  efficiently built software kernels that are capable of processing huge data. We are also able to stay online every moment and to connect to various network infrastructures. We plan to leverage the capabilities of a smartphone to deploy our honeypot.In this section we discuss mobile honeypots that relates to our idea of designing a mobile honeypot for detecting \ac{ICS} specific attacks. 
	\vspace{5mm} 
	
	
	The power of mobility, computing resources, usability and flexibility make Mobile devices a good platform to host low interaction honeypots. Some researchers believe that mobile honeypots are still not well defined and could be used to define either a probe deployed on a mobile device or on a mobile operating system. It can also be defined for a system that is controlled in the network of mobile devices \cite{wahlisch2012first}.
	
	\vspace{5mm} 
	
	Early research on mobile honeypots focused only on  Bluetooth communications[5,17]. The continuous advances in the field of smartphone technology has enabled better opportunities towards honeypot research on smart phones. 

	
	There has been existing work that focused on detection of mobile specific malware. The first to discuss the idea of a honeypot for smartphones were \textit{Mulliner et al.}, by providing the initial ideas, challenges and an architecture for their proposed system\cite{mulliner2011poster}. Nomadic Honeypots\cite{Liebergeld_nomadichoneypots:} concentrates on mobile specific malware and also trades off with a lot of personal information.

	\begin{itemize}

	\item\textbf{HoneyDroid} HoneyDroid \cite{mulliner2011poster} is a smartphone honeypot for Android operating system which claims to be the first ever honeypot in the mobile honeypots category which makes use of smart phone hardware to host the honeypot.It is built on a Linux micro-kernel and is customized to impose restrictions on the Android operating system for monitoring its activities. The architecture is comprised of an event monitor, to monitor active connection requests and also system calls in the kernel level; filters to mitigate any attempts of malware trying to affect the system and a log software to log all the activities. 
	
	\vspace{3mm}
	This honeypot is also focused on detecting attacks from apps installed in the device which try to infiltrate the kernel for gaining unauthorized access. The system  involves virtualization which enables simulation of various services. 
	This could also result in an overhead, hereby causing a signature which can be detected by attackers and malware. However, the direction of HoneyDroid was to introduce the concept of mobile honeypots. 


	\item\textbf{Cellpot:}  Cellpot \cite{liebergeld2014cellpot} concentrates on detection and defense of attacks in the cellular network. It comprises of a collection of honeypots, or honeynets that are deployed on mobile phones. Cellpot consists of applications like SMS spam prevention, mobile phone theft and malware protection. The honeypot mainly is concentrated towards Small Cells\cite{liebergeld2014cellpot}, wireless infrastructure deployed in customers site and operated in licensed bands. The main use of Small cells is to support the need of coverage and capacity. These points are a good place to deploy the honeypots to detect malware and other intrusion attacks.
	
		\vspace{3mm}
	 Denial Of Service is the most common category of attack in the area of cellular networks, and with the help of few devices,this attack can be executed successfully. Introducing a honeypot approach for detecting such attacks at small cells is a feasible solution.The concept of Cellpot is to detect, collect intelligence and mitigate threats based on the cellular network that are operated on the base stations. Further, it has the ability to 	deploy countermeasures against detected threats, and enables a wide area of applications. It provides a good platform for mobile network operators to easily deploy the application and run additional applications to reduce signaling overhead.

	\item\textbf{Nomadic Honeypots:}Nomadic Honeypots \cite{Liebergeld_nomadichoneypots:} propose a concept that provides a framework to enable mobile network providers to collect threat intelligence data on smartphones. It was the first proposed honeypot on a smartphone platform. It places the smart phone user as a centric role and requires the device to be used in a normal way. The honeypot is deployed on the smartphone in a separate partition which serves the purposes of providing all communication of the mobile OS, hosting infrastructure for data collection, facilities for snapshots and logging and lastly provides a secure backchannel for the operator. The main partition hosts the mobile OS and a strict line of isolation is followed between the partitions to ensure the robustness of the honeypot infrastructure. 
	\vspace{3mm}
	
	The operator can further use the collected data to gain intelligence on mobile threats. Snapshots of the mobile OS file system to do an offline forensic analysis of attacks could be easily provided which can gain an thorough insight on the nature of the threats and use the findings to protect his customers.

	\item\textbf{HosTaGe:}\cite{Vasilomanolakis:2013:TNI:2516760.2516763},\cite{Vasilomanolakis:2014:HMH:2659651.2659663} is an Android application which acts as a Mobile honeypot, determined to detect malicious networks and probe for attacks. It is user centric and aims at creating security awareness to its users. The results obtained in this process are synchronized with a global repository and also can be shared locally through bluetooth. The current version has capabilities of emulating as Windows, Unix, Apache Server, \ac{SQL} and Paranoid host. Attacks through HTTP, \ac{SMB}, \ac{SSH}, \ac{HTTPS}, Telnet and \ac{FTP} can be identified.  HosTaGe is one of the mobile honeypot capable of simulating different targets on a mobile platform. 
	
	\end{itemize}   
	
	
	
      
   \vspace{5mm}          
   \subsubsection{SCADA honeypots} \label{SCADA honeypots}

	Analysing the security concerns of \ac{ICS} \ac{SCADA} systems and the advantages of honeypots, a solution could be implemented to combine the needs and features. \ac{SCADA} honeypots could be deployed in \ac{ICS}  Networks for monitoring and analysis. They act as an additional line of defense providing warnings and notifications for attacks. Designing a \ac{SCADA} honeypot involves studying the architecture of the \ac{SCADA} systems and the components, protocols involved in communication and processing of data. Further, as discussed before, \ac{SCADA} networks comprise of hardware devices like \ac{PLC}s and \ac{RTU}s which play a very critical role in processing and communication of data. \ac{SCADA} systems rely on \ac{PLC}s for data processing. If \ac{PLC}s are targeted by attackers to compromise their working, it could bring down the entire plant, hereby resulting in a huge catastrophe.
	
	\vspace{3mm}
	 Modern day \ac{PLC}s offer TCP/IP communication which can be used to control and manage the data flow between other \ac{PLC}s and control servers. On investigating attacks that have occurred in the past, Stuxnet a malware, was found to be injected in a Nuclear Enrichment Facility in Iran in the year 2009. Stuxnet was found to be injected into the internal network using a USB drive to one of the host control systems. The malware spread from that system to other systems through intranet and remained hidden from operators. Stuxnet was able to interfere with the working of a \ac{PLC} that controlled centrifuges and managed to compromise the conditions on which the \ac{PLC} depends. It was only by the observation of an operator that the \ac{PLC} was causing the centrifuges to run more fast than usual was detected. But nobody could determine what caused the centrifuges to run abnormally.  
	 
	 
	 
	\vspace{5mm} 
	Detecting such kinds of attacks is not only complex but also very necessary. Such kind of attacks cannot be detected neither by signature based systems, nor by firewalls. Some organizations took initiative to design honeypots for \ac{SCADA} systems. They are elaborated as follows:
	
	
	\vspace{5mm} 
	\begin{itemize}
	\item\textbf{SCADA Honeynet}
	SCADA Honeynet Project\cite{5198796} is a project aimed at building honeypots for industrial networks. It was the first of its type. \ac{SCADA} Honeynet was designed to simulate the \ac{PLC}s and detect attacks performed on them.The short-term goal of the project was to determine the feasibility of building a software-based framework to simulate a variety of industrial networks such as \ac{SCADA}, \ac{DCS}, and \ac{PLC} architectures. It provided scriptable industrial protocol simulators to test actual protocol implementation. The design was an integration of stack level, protocol level, application level and hardware level. The honeypot was carefully designed to cover all the services offered by the \ac{SCADA} systems, including the networking devices like routers and a direct serial device. The setup of the honeypot is complex as it involves the configuration of two virtual machines as per the network infrastructure.
	
	
	
	
	\item\textbf{Trend Micro SCADA honeypot}   Trend Micro a global security software company conducted an experiment\footnote{http://www.trendmicro.com/cloud-content/us/pdfs/security-intelligence/white-papers/wp-whos-really-attacking-your-ics-equipment.pdf} to detect attacks on \ac{SCADA} by setting up 12 honeypots in 8 countries. The honeypots obfuscated a public municipal water control system based on \ac{SCADA} that was connected to the Internet. Attacks were basically focused on fondling with the pump system.  The objective of this experiment was to analyze the attacks of the Internet-facing \ac{ICS} \ac{SCADA} devices and the reason behind them. Further, the research aimed at identifying if the attacks performed on these systems were intentional by specific interests.
         
  	The honeypot architecture design used, is a combination of high-interaction and device based production honeypots. A total of three honeypots were deployed to ensure as much of the target surface as possible. All three honeypots were made public, facing the Internet. The honeypots were assigned different public \ac{IP} addresses of different subnets distributed in the United States. The honeypot provided a full simulation of the pumps depicting a real scenario. The design used high interaction honeypots that were deployed on server instances and the real hardware was used to also simulate the environment. 
  	 
	
	\item\textbf{Digital Bond}	A security research and consulting firm created a honeypot system that comprised of two virtual machines. The honeypots are open source. One of the virtual machine acts as a \ac{PLC} honeypot and the other is a monitoring engine that logs all the traffic information. This system is also called a Honeywall. Honeywalls can also be used to monitor high interaction \ac{PLC} honeypots. The Honeywall comprises of Snort IDS and signatures with respect to \ac{PLC}. The services that are simulated are FTP, TELNET, HTTP, SNMP and Modbus TCP. This honeypot focused on utilizing the signature set of \ac{IDS} to monitor the network.\ac{IDS}. 


	
	\item\textbf{Conpot}
	Conpot\footnote{http://conpot.org/} is a low interactive server side \ac{ICS} honeypot designed to be easy for deployment, modification and extension. It provides a range of common industrial control protocols capable of emulating complex infrastructures to convince an adversary .To improve the deceptive capabilities it also provides the possibility to host a customized \ac{HMI} to increase the honeypots attack surface. The default configuration of Conpot simulates the Siemens Simatic S7-200 \ac{PLC} with the Modbus, S7 and HTTP protocol simulation. Conpot provides ideal SCADA honeypot capabilities. The logging mechanism observed does not provide enough information for the attack analysis. 
	
	\end{itemize}
	
	The research works discussed involve complex setup and  simulation of a specific target system. They also need more resources and computing power to host the honeypots. The existing research work on mobile honeypots concentrate on designing a honeypot to detect mobile phone specific attacks.  We aim to create a honeypot that is less complex, provides greater analytical capabilities, high detection ability and flexibility.
	
	
	\subsubsection{Multistage Attacks}\label{Multistage Attacks}
	
	
	In multiple networks there occurs a large number of stealthy scans, worms outbreaks and distributed denial-of-service attacks simultaneously. These attacks are very difficult to identify using \ac{IDS} which monitors only some part of the Internet. \textit{Zhou et al.}  present survey\cite{zhou2010survey} of co ordinated attacks and collaborative intrusion detection. Such attacks can be detected using \ac{CIDS}. This part of the report summarizes the detection of attacks using \ac{CIDS}s. However there are two main challenges to be faced in \ac{CIDS} research, which includes its architecture and alert correlation algorithms. 
	
	
	\vspace{3mm}
	Attackers have recently been able to perform complicated attacks by targeting or leveraging large number of hosts that are distributed over large geographic area. For instance attackers can scan large number of hosts simultaneously to look into vulnerable systems or softwares, which are stealthy scans. Also they can use worms, which are self replicating computer programs to spread and multiply malicious code on many vulnerable systems quickly. Distributed denial of service(\ac{DDOS}) can also be performed  by overloading the target link by thousands of compromised hosts to disrupt and deny its service. These different ways of attacks could be integrated by an attacker which can result in a significant threat to the Internet security.
	
	
	\vspace{3mm}
	There exists four groups of alert correlation techniques used by \ac{CIDS}s \cite{zhou2010survey}. They are similarity based, attack scenario based, multi-stage and filter based. The similarity based technique correlates alerts based on the similarity between the alert attributes. The similarity between the alerts is calculated using a function, the resulting score tells if the alerts will be correlated or not. The attack scenario based technique correlates the approach based on predefined attack scenarios. The attack scenarios for this technique can be specified by the user or it can be trained by the datasets. The multistage technique is based on correlating the attacks based on the causality of earlier or later alerts. In this approach , there could be reconstruction of some of the complex attack scenarios by linking individual steps which are a part of same attack. In filter based technique, the alerts are prioritized based on the criticality of the protected system.
	
	
	\vspace{3mm}
	The main focus is made on multi-stage approaches in this thesis. We focus on the approach where there are possible logical links between the post-condition of an attack A and pre-condition of attack B. Therefore, executing a given attack can contribute to executing another attack. Whenever an alert is raised, it will be compared with the earlier alerts to check if correlation conditions are satisfied or not, which results in set of correlated pairs of alerts. These pairs will be verified if they belong to existing attack scenarios. If the scenario is verified then it is joined with the earlier existing scenario, or else a new scenario will be started. We  model this correlation approach formally through \ac{EFSM}s and discuss it further in section~\ref{Formal Model}.	
	
	\vspace{3mm}
	In this approach of multistage , the attacks are correlated based on the causal relationship between alerts, and most of them can detect unknown attack scenarios. This approach mainly focus on correlated alerts and discards others that cannot be correlated. However, the reason for this discarding of the alerts that are not correlated is not analyzed rigorously and the accuracy of the correlation is affected by the false alarms generated by individual \ac{IDS}s. Also, the complete library of attack steps is expensive as there are large number of attack types.	A language called LAMBDA is used to correlate the alert from different \ac{IDS}s and \ac{CIDS}. The attack is described by four main components:
	
	\begin{enumerate}
	\item Pre-condition and post-condition: The condition of the target to be satisfied for performing an attack, and the impact on the target system after the attack is successful.
	\item Scenario: combination of attack events or steps for completing an attack.
	\item Detection: steps for attack detection. This events sets may be different from scenario as some attack steps cannot be observed in \ac{IDS}s.
	\item Verification: some conditions on target system in order to check an attack is succeeded, like vulnerabilities existing in the system.
	\end{enumerate}
	
	
		
	
	\subsubsection{Signature Generation}\label{Signature Generation}
	
	Signature-based Intrusion Detection Systems rely on a signature database for the detection of malicious payloads. This signature database has to be updated with signatures for newer malware. Some attacks, as discussed in the previous section are target specific, for example specific to an organization. Detection of such attacks is very complex. Major IDS fail to detect such attacks due to the absence of signatures to detect such kind of attacks. Honeypots act as active entities that capture such attacks with no impact to production systems. The attacks can be analyzed and studied offline to determine the its complexity and impact towards internal systems. The attack information captured by honeypots could be used to generate signatures that IDS depend on to detect attacks. This is possible by either deploying a honeypot is the same network as in the IDS, or could be a collaborated honeypot that is capable of generating signatures that popular \ac{IDS} can integrate. 
	
	
	\vspace{3mm}
	A similar approach was proposed with Honeycomb\cite{kreibich2004honeycomb} . It focuses on a honeypot system that automates generation of attack signatures for network intrusion detection system. The system uses this pattern matching technique and checks protocol conformance on various levels in protocol hierarchy to the captured network traffic in a honeypot system.The attack signatures are required to describe the characteristic elements of attacks. The system explained in this part of the report supports signatures for Bro and Snort \ac{NIDS}s.
	
	\vspace{3mm}
	The system makes use of Honeycomb, a system that generates signature for malicious attacks on network traffic automatically. The honeypot honeyd\cite{Provos:2004:VHF:1251375.1251376} is extended by a subsystem that checks the traffic inside the honeypot at different levels in protocol hierarachy. Honeypots mainly monitor and log the activities of entities that attack or probe the system. The paper describes the extension of \textit{honeyd}, a low level interaction open-source honeypot. honeyd simulates the hosts with individual networking personalities. It intercepts on the nonexistant hosts and use the simulated systems to respond to this traffic.

	\vspace{3mm}
	The idea is to keep the system free of any particular knowledge related to certain application layer protocols. An overview of the Signature Generation Algorithm is shown in Figure~\ref{f:Honeycomb Signature Generation Overview}. Every received packets results in initiating the same sequence of activities in Honeycomb. These activities involves:


	\begin{enumerate}
	\item If any connection state is existing for new packet, that state is updated, or else a new state is created.
	\item If the packet is outbound, processing is stopped in this state.
	\item The protocol analysis is performed by Honeycomb at network and transport layer.
	\item For every stored connection, header is compared to detect the matching \ac{IP} networks, initial \ac{TCP} sequence numbers, etc. If the connections have the same destination port, Honeycomb does a pattern detection on the exchanged messages.
	\item If there are no useful signatures created, processing stops. Or else the signature is used to augment the signature pool.
	\end{enumerate}
	

	
	\vspace{3mm}
	\textit{HoneyAnalyzer}~\cite{thakar2005honeyanalyzer} is an analysis tool that extracts signatures of intrusion detection patterns using honeypots. The basic idea is to analyze \textit{honeyd} c.f~\cite{provos2003honeyd} logs stored in an \ac{RDBMS} using a web interface. The signature extraction consists of three parts.
		
		\begin{enumerate}
		
		\item\textbf{Data Capture:} A logging component which consists of \textit{honeyd} and \textit{tcpdump}~\cite{jacobsen2005tcpdump} for collecting data.
		
		\item\textbf{Data Analysis:} Tool for analysis and extraction which involves analyzing data of signature extraction mechanism to identify specific attack signatures.
		
		\item\textbf{Signature Extraction:} Extract refined attack signatures.
		
		\end{enumerate}
		
		The data capture component logs all the communication of an attacker through a honeypot. The \textit{honeyd} honeypot has a log mechanism for reporting the connections that are attempted and completed. To analyze the attacks completely we need the payload information of the connections established or attempted. \textit{tcpdump} is responsible for this activity. It captures the packets with their payload. The data analysis component is used to extract specific attack signatures. This component has a web interface that provides a graphical output which provides the administrator details about the most attacked port and the respective IP address. The strategy for extracting specific attack signature in honey-analyzer is as explained below:
		
		\begin{enumerate}
		
		\item Simulating the network using \textit{honeyd}.
		
		\item Start traffic analysis through \textit{tcpdump}
		
		\item The attacks received on the \textit{honeyd} log file is parsed and pushed into the database using a script.
		
		\item The web interface provides an overview of the attack patterns and the data for analysis. The web interface provides packet information, realtime network traffic, attack ports and connections per second. 
			
		\end{enumerate}
		
		This approach relies on the experience of a user to determine the malicious traffic and does not generate signatures. However, it identifies and extracts attack specific signature data. This data has to be composed into meaningful signatures that can be deployed on \ac{IDS}. Honey-analyzer proposes the idea of using the attack related data that is captured on honeypots to create signatures.
		
		\vspace{3mm}
		In this thesis, we aim to create \ac{ICS} attack specific signatures for \ac{IDS}. The related work forms an important basis for our design. We also argue to be the first honeypot to generate signatures for \ac{ICS} environment.  
		
		\begin{figure}[H]
			     
			           \centering
			           \includegraphics[scale=0.5]{hsigen}
			           \caption[SCADA Hits]{\label{f:Honeycomb Signature Generation Overview}Honeycomb Signature Generation Overview.\cite{kreibich2004honeycomb}}
			           
			\end{figure}
		
	\subsection{Summary}
	This section discusses about ICS SCADA. The section also consists of a brief explanation about \ac{ICS} \ac{SCADA} components which include network and control components. It explains the architecture of \ac{SCADA} and shows the data communication and its processing on \ac{PLC}s. We also see security aspect with respect to \ac{ICS} \ac{SCADA} and various threats that \ac{SCADA} is subjected to. The section also explains about Modbus protocol, its advantages and security issues. We also saw about honeypots, its main functionalities and its advantages. We provide an overview of the Siemens Simatic series which controls and provides automation needs in \ac{ICS} infrastructure today. Furthermore, the related work section explains about different types of honeypots such as Honeynets, mobile honeypots and \ac{SCADA} honeypots. The chapter also briefs us on related work on multistage attack and signature generation through honeypots. 
    


  \newpage    
  
  \newpage       
  \section{Proposed System and System Design} \label{Proposed System and System Design} 
  
  	We consider the all of the requirements to design and  develop a mobile honeypot system for the detection of \ac{ICS} specific attacks. We also propose a honeypot capable of detecting malware and generating signatures for increasing the detection capability of \ac{IDS}. In what follows we discuss the proposed system and propose a design for our honeypot.
  
  \subsection{Proposed System}\label{Proposed System}
    
   
      In this thesis, a low interaction mobile honeypot mechanism to simulate an industrial \ac{PLC} will be designed and implemented. The design also aims at detecting attacks and making inferences about the adversaries and attacks. The final implemented version will be integrated to the HosTaGe app along with advanced mechanisms that HosTaGe already provides to its users.  
      As the proposed system deals with implementing a low interaction honeypot, the challenge involves implementing only the essential components or services, that satisfy the discovery and vulnerability to attack them, for example, the network stack. Along with basic attack detection, the system must also have a short response time, robust design to withstand the attacks and also maintain a log of the exploit for further analysis and backtracking. An attempt will be made to detect attacks forged with popular identified worms like Stuxnet. The conclusions on the attacks made will be pushed on to a central repository where the details of the attack are made public for users worldwide.
    	
    	
    	\vspace{3mm}   	
    	The proposed system also aims to identify Multistage attacks and generate signatures for IDS. The attacks containing malware are formally modeled to analyze their propagation through the network and their dropping mechanism. This model helps in identification and detection of such malware and also generating respective signatures.There are several systems that propose the idea of Multistage attacks and their detection. 
    	
    	
    	\vspace{3mm}   	
   		In the following sections we discuss the design of the proposed HosTaGe ICS mobile honeypot. The discussion involves the \ac{ICS} perspective for HosTaGe, the simulated Siemens Simatic S7-200 \ac{PLC} system , protocols supported by the \ac{PLC}, the formal model, detection mechanisms and signature generation mechanisms.
   		
   		
  \vspace{3mm} 
  \subsection{HosTaGe ICS Perspective}\label{HosTaGe ICS Perspective}
  
  HosTaGe has implemented mechanisms to emulate different kind of hosts like a windows host, linux host, webserver, \ac{FTP} server, \ac{SSH} server and more. The simulation of industrial level \ac{SCADA} based \ac{PLC} will be added to the the existing list of simulated hosts and services. To simulate \ac{PLC}s it is important to understand their communication and control infrastructure. \ac{PLC}s have network interfaces that support Ethernet, TCP/IP, Modbus\cite{4627171}, DeviceNet\cite{898793}, ControlNet\cite{898793}, Foundation Fieldbus\cite{1435740}. The manufacturers have their own in built shells to support \ac{FTP} commands.  The communication capability is realized through the Ethernet module which is deployed on an embedded \ac{OS}. This \ac{OS} includes the network protocol implementation for protocols such as Modbus/TCP. The Telnet and FTP server have the required information to identify the vendor and the firmware version deployed on the device. The components in the network stack  of the \ac{PLC}that have to be simulated are the basic TCP/IP mechanism, Modbus TCP, \ac{FTP} server, HTTP server and a Telnetd server. The HTTP server simulates the controlling portal to  manage the \ac{PLC}.
 
   \vspace{3mm}    
  The discovery and identification of the \ac{PLC} in the network can be through a network nmap scan that reveals information about the host name, ports 21, 80 and 502(Modbus) open.      
  The main objective is to detect attacks made using the protocols offered by the Siemens Simatic S7 200 \ac{PLC} . A logging mechanism logs the information about the attacker in pursuit.  
  The architecture of HosTaGe ICS is as shown in Figure~\ref{f:Hostage ICS Architetcure}. A discussion of the components is followed in the below sections ~\ref{HosTaGe Core},~\ref{Logger},~\ref{Graphical User Interface},~\ref{Port Handling} and ~\ref{HosTaGe Services}.
  
  
  \begin{figure}[h]
  \centering
             \includegraphics[scale=0.75]{Hostage-ics-arch}
             \caption[HosTaGe ICS Architecture]{\label{f:Hostage ICS Architetcure}HosTaGe ICS Architecture}
  \end{figure}
       
       
       
       \vspace{5mm} 
      \subsubsection{HosTaGe Core} \label{HosTaGe Core}
      The HosTaGe Core forms the basic core mechanism of HosTaGe ICS. It is responsible for running the core mechanism and functionality of HosTaGe. It provides an interface for the activation and deactivation of implemented emulated protocols. The core interacts directly with the \ac{GUI} to provide notifications to the user of the connections made. The main components or sub modules of HosTaGe Core are Emulator and Connection Guard.
       
       \begin{itemize}
       
       \item\textbf{Emulator}
       The Emulator is responsible for the emulation of protocols in the Protocol Emulation . It is a multi threaded module which dedicates a thread for every protocol to be emulated and also actively listens to the incoming malicious traffic for the respective service ports. It calls the Logger module to log all the activities occurring at that port. The emulated protocol can accept multiple simultaneous connections at the same instance. A Connection Handler is started for every incoming connection request. Every Connection Handler communicates with the initiating client, providing a basic protocol interface based on the selected protocol for emulation. The Port Binder module is responsible for binding the ports . 
       
       The emulator provides a selection interface for the user to choose the protocol to be emulated. For the \ac{ICS} perspective we provide additional protocols like Modbus, S7, \ac{SNMP}, \ac{SMTP} and also a dynamically adapting version of \ac{SMB} and \ac{HTTP} which form a mainstream for attacking in \ac{ICS} systems.Adding new protocols to the Emulator module is hassle free. 
       
       
       \item\textbf{Connection Guard}
       A honeypot has to be robust itself to detect malicious attacks and survive them. However, it is always possible to compromise a system through tailor made strategies. The connection Guard mechanism limits the number of connections that can be received by HosTaGe ICS. This measure makes sure that our system is safe under the Denial Of Service attacks. In such cases, the number of connections are limited by the source \ac{IP} or for the destination port. The connections are also terminated over a period of time. 
       
       \end {itemize}
       \vspace{5mm} 
       \subsubsection{Logger}\label{Logger}
       The Logger module is responsible for logging the attack and connection data into the HosTaGe ICS SQLite database. It also supports export of the logs generated in different formats into the phone memory. The formats include plain text and in \ac{JSON} for data processing by third party applications. The logs are also synchronized with the remote repository for global synchronization. 
       
       \vspace{5mm} 
       \subsubsection{Graphical User Interface}\label{Graphical User Interface}
       HosTaGe ICS provides a user friendly and an interactive GUI for its users. The Overview shows the current state of the HosTaGe service with respect to the secure state of the network in which the device is placed. The Overview could be sleep, scanning- without any attack previously detected on the network,  scanning- with an attack previously on the network and on attack detection. This provides an overview of the network health condition for the users. Figure~\ref{f:HosTaGe ICS Overview} shows the GUI of the Overview screen of  HosTaGe.
       
        \begin{figure}[h]
         \centering
                    \includegraphics[scale=0.10]{overview}
                    \caption[HosTaGe ICS Overview]{\label{f:HosTaGe ICS Overview}HosTaGe ICS Overview}
         \end{figure}
       
       HosTaGe offers various functional modes through the Profiles module. This module enables the user to select the Target system type that has to be emulated. Every profile has predefined protocols that have to be emulated which combine together to form the emulation of a target system. However, users can also select from a list of services that they want to include in the profile. HosTaGe also provides protocols to be emulated as per users choice. Users can decide upon the protocols to be emulated. HosTaGe ICS adds Modbus, \ac{S7} and \ac{SMTP} to the list of protocols supported by HosTaGe. Figure~\ref{fig:HosTaGe ICS Profile View} shows the Profile view and Figure~\ref{fig:HosTaGe ICS Protocol View} shows the Services view of HosTaGe ICS. 
       
       
       \begin{figure}[H]
       \centering
       \begin{minipage}{.5\textwidth}
         \centering
         \includegraphics[width=.4\linewidth]{profile}
         \caption[HosTaGe ICS Profile View]{HosTaGe ICS Profile View}
         \label{fig:HosTaGe ICS Profile View}
       \end{minipage}%
       \begin{minipage}{.5\textwidth}
         \centering
         \includegraphics[width=.4\linewidth]{protocol}
         \caption[HosTaGe ICS Protocol View]{HosTaGe ICS Protocol View}
         \label{fig:HosTaGe ICS Protocol View}
       \end{minipage}
       \end{figure}
       
     
       
       \vspace{5mm} 
       \subsubsection{Port Handling}\label{Port Handling}
       The Android \ac{OS} has a security policy of allowing only signed applications access to privileged network ports(below 1024). As HosTaGe is a third party application, by default it does not have access to these ports. The policy sounds fair with respect to the security enforcements but adds additional challenges in the implementation of the protocols for HosTaGe. \ac{ICS} specific profiles include emulation of protocols like Modbus, \ac{S7}, HTTP which fall in the network privilege ports category. A script was implemented to achieve this challenge in native C and cross compiled for Android OS. This program binds a port passed to it as a parameter and sends the file descriptor back to the caller through a UNIX domain socket\cite{Vasilomanolakis:2013:TNI:2516760.2516763} . The program uses a Java server sockets to create a socket from a file descriptor which ensures connection to the privilege ports. Some applications already bind to some previlege ports. In this case, HosTaGe notifies user that the port is bound to some other service. 
       
       
       However, a root privilege or a rooted Android device is required for the above mentioned program to work. We also look forward to overcome this situation for the app to work in the future. 
        \vspace{5mm} 
       \subsubsection{HosTaGe Services}\label{HosTaGe Services}
       HosTaGe relies on services that run in background to achieve its constant attack detection activities. These services ensure monitoring the network, managing the connections, handling of attacks and notifications to the user through the app \ac{GUI}. The HosTaGe app currently has 3 services running and are described as follows:
       
       \begin{itemize}
       
       
       \item\textbf{HosTage service:}The main service that HosTaGe relies on is the HosTaGe Service that controls and co ordinates the mainstream activities of HosTaGe. It is responsible for managing connections, listening on ports, enabling protocol services, attack notifications and logging of data. The HosTaGe service is responsible for the basic functionality of HosTaGe. 
       
       
       
       \item\textbf{Multistage Attack detection Service} The Multistage Attack detection service constantly checks the records database for any Multistage attacks. The attack records for a pre-defined time period are retrieved and scanned for Multistage attacks.
       
       
       \item\textbf{Synchronization Service} HosTaGe ICS synchronizes the attack records received with a remote repository periodically. This synchronisation is aimed at creation of an updated central repository with blacklisted \ac{IP}s and payload information. This information is used by a Collaborative Intrusion detection  system for monitoring traffic based on the blacklisted \ac{IP}s.  
       
       
       \end{itemize}
       
       
    \vspace{5mm}     
  
 	\subsection{Protocols}\label{Protocols}
	
	
	\vspace{3mm}
	The \ac{ICS} \ac{SCADA} systems include the master and slave devices.Our design must be capable of simulating the services of both the master and slave devices. The Siemens Simatic S7 supports a wide range of protocols which include Modbus/PROFIBUS \ac{TCP}, \ac{HTTP}, TELNET, \ac{FTP}, \ac{SNMP}, \ac{SMTP} and \ac{S7}. Modbus \ac{TCP} and \ac{S7} are the communication protocols and the rest of the protocols are enabled as added features.  Considering the other protocols form an important part of the security analysis as malware are usually designed keeping the flaws of the network and the software bugs in mind.

	\begin{itemize}

	\item\textbf{HTTP:} HTTP is supported by the majority of \ac{PLC}s for remote configuration purposes. The HTTP web server in the \ac{PLC} enables GET/POST messages for information exchange. This \ac{HTTP} server is simulated by HosTaGe through a dynamic \ac{HTTP} protocol implementation. A default welcome page is displayed when the adversary tries to navigate to the device's webpage.

	\item\textbf{Telnet:} The Telnet protocol allows accessing a basic shell on the devices in which users are able to dump memory, delete files and execute commands. It provides command and control to the target remote devices. It enables file system based commands and directory listing. Users or applications can communicate with the \ac{PLC} for file and backup operations.

	\item\textbf{FTP:} \ac{FTP} provides file transfer and communication between end devices. These are usually files containing sensor readings and logs.

	\item\textbf{SNMP:} The Siemens S7 family of \ac{PLC}s the configuration of client devices through \ac{SNMP}. This allows to remotely manage devices on the network.

	\item\textbf{SMTP:} SMTP is mainly enabled for notification servivce in case of device failure or data inconsistency.

	\item\textbf{SMB:} Although SMB is not a part of Siemens S7,  it is a main component of the Master devices that control the slave \ac{PLC}s. The \ac{SMB} protocol is used to share files in a network of hosts. This protocol is typical on enterprise intra-networks due to file sharing requirements.

	\item\textbf{Modbus/PROFIBUS TCP:} Modbus TCP acts as a strong communication mechanism between the slaves and the master devices. It forms a backbone for industrial systems automation. Modbus has instruction sets for the interaction of devices. \ac{PLC}s have registers 1 as memory units. The instruction sets are specified as functions which denote Read/Write (R/W) operations on the registers of the \ac{PLC}s 
	The protocol is used for communication exchange between \ac{PLC}s and control systems.  
 
	\item\textbf{S7:} The \ac{S7} protocol is a Siemens proprietary protocol utilized in \ac{PLC}s of the Siemens S7 family. It is used for programming the \ac{PLC}s, communication of data between \ac{PLC}s, acquisition \ac{PLC} data from \ac{SCADA} systems and for diagnostic purposes. The protocol forms as a base for accessing the registers for R/W operations and also programming the \ac{PLC} for user defined tasks.

	\end{itemize}
	\vspace{5mm} 
	\subsection{Formal Model}\label{Formal Model}
	
	The attack strategies employed by attackers can be interpreted as a sequential approach. This interpretation could be formally represented as a state model. The formal model could be used as a design perspective for the detection of attacks.\textit{ Sengar et al.} \cite{Sengar06voipintrusion} propose the idea of \ac{VoIP} intrusion detection system which is based on an approach of using state machines for the modelling of network protocols and the interaction between them. Using this idea they propose a \ac{VoIP} IDS based on the protocol state machines. The idea is that to utilize the state transitions made in protocol state machines for intrusion detection. A protocol state machine based IDS can be viewed as a variant of anomaly detection mechanism. Once the protocol state machine is constructed and the relevant attribute features are identified, the approach not only reduces the number of false alarms but also found to be able to detect previously occurred and unseen attacks.
	
	\vspace{3mm}
	A state machine deals with low-level abstraction of protocol. It can express protocol design in terms of desirable or undesirable protocol states and state transitions. The approach constructs a communicating finite state machine where the output of one machine is connected to the input of other machine. An extended finite state machine called "Mealy" finite state machine is used for the approach. The Mealy machine extends with input and output parameters, context variables, operations and predicates.


	\vspace{3mm}
	We further extend the idea proposed to HosTaGe ICS to detect Multistage and File Injection attacks. We employ the Mealy state machine model \cite{wagner2005moore}  to define the stages of attacks and construct formal state machine diagrams to represent attack process.


	\vspace{3mm}
	The detection mechanism of HosTaGe ICS is formalized with an \ac{EFSM}. An \ac{EFSM} has all the properties of a normal \ac{FSM} with an extended design of utilizing \textit{if} -conditions, instead of only boolean conditions, to specify how a state transitions to a new state \cite{InteractiveSystems}. The formal model of our proposed detection mechanism is given by Attack Detection \ac{EFSM} $M = (S, s_0,I,O,V,P,\delta,\lambda)$~\cite{1234567}

 
 
 	\begin{figure}[ht]
        \centering
        \includegraphics[scale=0.25]{EFSM_attack_detection}
        \caption[EFSM of the attack detection and signature generation mechanism.]{\label{f:EFSM}EFSM of the attack detection and signature generation mechanism}
        \cite{1234567}
        \end{figure}
        
        
	and is illustrated in Figure~\ref{f:EFSM}. The set of all states are represented with S. The \ac{EFSM} starts in the \textit{Normal Behavior} state, represented by $s_0 $. If any protocol communication is detected by the honeypot, the \ac{EFSM} transitions to the \textit{Attack} state. For as long as the same protocol attack is observed, the state remains the same. If a timeout occurs the \ac{EFSM} transitions to the \textit{Generate Signature} state followed by the \textit{Issue Alert} state. The signature generation is optional and will capture either single attack or multistage attack types. After an initial attack, observing attacks originating from other protocols (but the same host) that have not yet been observed moves the state to the next \textit{Multistage Attack Level x}, where \textit{x} corresponds to the number of different protocols observed after the first one.

	\vspace{3mm}
	The inputs \textit{I}, outputs \textit{O}, variables \textit{V} and predicates \textit{P} are tightly linked together. State transitions are carried out wherever specific inputs $i	\in I$ are received.This transitions may also generate an output $o \in O$. In the \textit{Normal Behavior, Attack} and \textit{Multistage Attack Level x} states, the supported protocols are used as inputs and outputs. As such, $\{Modbus,S7, SNMP, HTTP, Telnet, SMB, SMTP, HTTPS, SSH, FTP\}$ $\in I \in O$ for these states. The inputs of \textit{I} are not limited, however, to only ports. Special activities of interest on a protocol are also considered inputs.  For instance, the act of requesting a file through the \textit{SMB} protocol is an input on itself. \textit{V} is a finite set of variables. These variables are used to construct a set of predicates \textit{P} used for determining if a state transitions to another one. Each attack state hold a boolean variable $v \in V$ for each emulated port. If a particular port has been observed in the entire life of the \ac{EFSM}, the corresponding variable for that port will be set to true. Besides variables, predicates \textit{P} consist of the logic operator \textit{AND} and the arithmetical operator =.We define the \textit{Protocol Connection} predicate as the condition where a new protocol is observed without having observed other protocols yet. The \textit{Different Protocol} predicate indicates, as the name suggests, that a new protocol has been observed after having seen at least one other. If any of these predicates is true a state transition takes place.

	\vspace{3mm}
	The final element of our model is the set of transitions $\delta(s_i ,i, p) = s_j$ and the outputs $\lambda(s_i , i, p) = o$ generated by the transition itself. The set of transitions specify that whenever state $s_i \in S$ receives the input $i$ and the predicate $p \in P$ is satisfied, the
	\ac{EFSM} transitions to state $s_j$ and outputs $o \in O$. The outputs are used by the \textit{Generate Signature} state to create signatures for misuse analysis.



	\vspace{5mm} 
	\subsection{Detection Mechanisms}\label{Detection Mechanisms}
	HosTaGe ICS is designed to identify three different class of attacks: Single-Protocol Level Detection (SPLD), Multi-Stage Level Detection (MSLD) and Payload Level Detection (PLD). 
  
  \begin{itemize}
   
    
 \item\textbf{SPLD:} SPLD attacks refer to those that occur on a single-protocol, eg:HTTP connection attempts without observing other protocols or any extraordinary payload-level information. This is the simplest type of detection which still contains interesting analysis potential. 
  
  \item\textbf{MSLD:} MSLD refers to attacks that originate from the same source and attempt to exploit different types of protocols within a small window of time. These type of attacks are identified by the honeypot with the \ac{EFSM} shown in Figure~\ref{f:EFSM}. An important factor in MSLD is the time-window (tw) that determines whether an attack should be mapped as the \ac{SPLD} or the \ac{MSLD} class. This means that when the
  \ac{EFSM} is on the Attack state and no further activity is detected (for a maximum of tw)   a timeout will occur and the attack will be identified as \ac{SPLD}. The tw can be adjusted with respect to the monitored network and its requirements.
  
  \item\textbf{PLD:} Payload Level Detection is enabled for the \ac{HTTP}, \ac{SMB} and Modbus protocols. These protocols carry critical payload in the case of \ac{ICS} environment. There could be malware in the form of executables injected in the payload which can trick the end systems to execute or process them. Studies show that majority of the malware existing today spread through HTTP payloads. \ac{PLD} extends the applicability of the \ac{EFSM} with respect to the inputs I. Referring back to our formal model, the outputs $o \in O$ from the Attack and Multistage Attack Level x states are used in the Generate Signature state to create signatures. Signatures are also \ac{EFSM}s that comply with the presented model. As we already mentioned, the input is not limited only to a port or protocol but also to potentially interesting payload- level information. Figure~\ref{f:EFSM-stuxy} can be considered as an example of an \ac{EFSM} that represents a signatures generated by \ac{PLD}. This signature identifies Stuxnet attacks from the set of outputs O obtained from the Attack Detection \ac{EFSM} shown in Figure~\ref{f:EFSM-stuxy}. The detection of Stuxnet
  \ac{EFSM} assumes an initial Normal Behavior state and transitions to \ac{SMB} Attack if an \ac{SMB} protocol is observed. Stuxnet tries to inject an infected file through \ac{SMB}. After a file is received, it (or its hash value) is sent to VirusTotal and, if the file is indeed malicious, the \ac{EFSM} transitions to the Stuxnet Attack state where its presence can be reported.
  
  \begin{figure}[h]
          \centering
          \includegraphics[scale=0.28]{efsm-stuxy}
          \caption[EFSM for PLD in the case of Stuxnet propagation.]{\label{f:EFSM-stuxy}EFSM for PLD in the case of Stuxnet propagation}
          \cite{1234567}
          \end{figure}
  
  
  \end{itemize}  
  
  
  \vspace{5mm} 
  \subsection{Signature Generation}\label{Signature Generations}
  An Intrusion Detection System relies on its signature set and static response analysis in order to determine malicious packets and traffic. IDS depending on the signature set are called signature based IDs and the one's relying on heuristics are called anomaly based IDS.
  Signature based \ac{IDS} monitors the packets and compare them with a database of signatures. An example for signature based systems could be antivirus systems that detect malware by comparing the files with a known set of malware signatures. Anomaly based \ac{IDS} detect intrusions by monitoring the system activity and classifying it as normal or anomalous. This classification is based on rules or heuristics to determine anything which is different from normal system behavior.This type of IDS also relies on neural networks and artificial intelligence algorithms to classify normal traffic. 
  
  
  \vspace{3mm}
  Majority of Enterprise Networks include \ac{IDS} on their network and are dependent on them for identifying and detecting malicious attacks and traffic. It is also true that due lack of expertise or internal constraints, many Enterprise Networks do not prefer having honeypots as decoys to determine malicious attackers and traffic. One such feature why \ac{IDS} are preferred is because of commercial support and maintenance. \ac{IDS} developers constantly monitor for newer malware and develop signatures to keep the \ac{IDS} signature database updated. The administrators have to update the \ac{IDS} signature database to defend against newer malware. This update system would fail under some circumstances where the developers fail to recognize the malware or if its a tailored attack strategy against an organization.Thus we propose a model of HosTaGe ICS where our honeypot has the feature of generating signatures for an IDS which could be deployed on a active IDS that monitors the network.  
  
 \vspace{3mm}
  The Signature Generation model of HosTaGe ICS concentrates on utilizing the attack results logged by HosTaGe. The active attack correlation techniques determine newer attack strategies followed by attacker, help in understanding the payload of malicious traffic and also instantly generate signatures that can be deployed to an IDS. This model also binds the usage of both an \ac{IDS} and a honeypot in a network for gaining stronger security aspects by utilizing the advantages and features of both an IDS and honeypot.
  
  
   \begin{figure}[h]
            \centering
            \includegraphics[scale=0.50]{idshoneypot}
            \caption[Network Architecture of IDS and HosTaGe]{\label{f:Network Architecture of IDS and HosTaGe}Network Architecture of IDS and HosTaGe}
            \end{figure} 
  
  
  Figure~\ref{f:Network Architecture of IDS and HosTaGe} depicts the suitable network architecture for having both an IDS and HosTaGe. The firewall acts as a first line of defense allowing only required traffic. The \ac{NIDS} comes second monitoring every packet for any malicious entries. HosTaGe can be placed either at the internal network or at the DMZ depending on the required area of interest for monitoring. It functions well at both areas. Any attacks towards HosTaGe is logged. This logged data can be analyzed further and if found malicious, a signature file for matching the content of the payload can be generated. This signature can be mounted on the Intrusion Detection System for monitoring the packets along with other attack signatures.  
  
  
	
	
	\vspace{5mm} 
	\subsection{SCADA PLC Profiles}\label{SCADA PLC Profiles}

	\ac{ICS} \ac{SCADA} devices can be classified into master and slave device types based on the interaction and functionality. The master system is responsible for controlling the slaves and send them appropriate commands for a task. These systems are usually control servers or host systems connected to \ac{PLC}s or slaves, that receive critical information and updates from the sensors placed on devices and \ac{PLC}s. The other most important systems are the automation \ac{PLC}s. Slave devices interact with many other devices and collectively process information to perform a task assigned by the master. When a Modbus master wants information from a device, it sends a message that contains the device address, the data it needs and the checksum for integrity. The network is typically like a hub structure. The data is broadcast in the network and the device from which the information was requested only responds. The slave devices cannot initiate communication and only can respond to a request made from the master. 
	
 	\begin{figure}[h]
        \centering
        \includegraphics[scale=0.6]{Master-Slave}
        \caption[SCADA Master and Slave profile]{\label{f:SCADA Master and Slave}SCADA Master and Slave profile }
        \end{figure}
        
        
        Modbus/TCP allows multiple masters to poll the same device in parallel. A unit can be either a master or a slave but not both. 
        
 	The Figure~\ref{f:SCADA Master and Slave} represents devices connected on the industrial LAN and the Modbus master-slave communication. The master devices poll the slave devices and request information. The information is processed and sent back to the master. There is also possibility that a \ac{PLC} acting as a master polls its data to the other devices like \ac{HMI} and other \ac{PLC}s in the network. 


	\vspace{3mm}
	In the past there have been attacks both internal and external on \ac{SCADA} systems. Popular attacks using Stuxnet, were carried out internally by deploying the malware on a host computer with the help of a USB drive. However, the malware made use of the vulnerabilities of the host system to replicate and spread through the network. Detecting such kind of attacks are very important and cannot be ignored. These attacks are more dangerous than the external attacks as there are various mechanisms to detect attacks from external sites. Internal attacks have proved to be more catastrophic. We also concentrate on the slave profile. This is required as the slave devices today have Ethernet communication and can communicate with the Internet. Due to some network configuration loop holes, the device may be accessed due to the Internet or the device itself may be configured to be accessed through the Internet by the administrator. For example, the slave devices also run HTTP servers which can display the sensor information in the form of a webpage. This device may be configured to be accessed through the Internet to check and monitor the sensor readings from a external system. There is no doubt about the possibilities of attack of such systems from the Internet. Thus we concentrate on simulating both the master profile, to check internal attacks and also slave profile to check external attacks. 
	
	\subsection{Malware Analysis}\label{Malware Analysis}
	Malware have been a huge threat to \ac{ICS} systems. The anomaly is that in \ac{ICS} environments malware is most likely to originate from internal hosts rather than from external networks. 	Considering popular malware for \ac{ICS}like Stuxnet, which originates from a host in the internal network and then propagates to a control host, it is very much necessary to consider attacks from internal networks. Stuxnet has still been a threat to \ac{ICS} systems and its detection is imperative as it is known to cause huge hazardous impact. Figure~\ref{f:Stuxnet Attacks Worldwide} shows the overview number of Stuxnet attacks detected in various countries, just when Stuxnet malware was discovered.
	
	\begin{figure}[h]
	\centering
	        \includegraphics[scale=1]{stuxworld}
	        \caption[Stuxnet Attacks Worldwide]{\label{f:Stuxnet Attacks Worldwide}Stuxnet Attacks Worldwide }
	\end{figure}
	
	
	 We first look at Stuxnet malware, its  propagation techniques and a proposed design to detect the malware.

	\vspace{5mm}

	\subsubsection{Stuxnet malware- A study}\label{STUXNET Malware- A study}

	Stuxnet was detected in the year 2010 in a Nuclear Enrichment  Facility, at Natanz, Iran. It was observed to be the most sophisticated and well engineered malware ever seen.Unlike other malware Stuxnet did not steal, manipulate or wipe information of all hosts in a network. It was a target oriented malware designed to compromise an assigned target. Stuxnet is designed to attack controller systems specifically by using \ac{SCADA} applications as a means of distribution. The malware was not remotely controlled rather it was stand alone and did not need any Internet access. As a part of validation, the malware contacted some command and control servers and to prove the compromise of the target system.
	
	
	\vspace{3mm}
	 The controllers are the Programmable Logic Controllers that control the data flow by using the application logic specified by the user. It is real time and is connected to controlling hosts through a fieldbus network. The controllers work on Ladder logic which is a small program that is void of any security metrics like integrity, confidentiality and authentication. The logic on controllers are in turn use to control physical conditions of heavy machinery. Any undesired changes occurring on the ladder logic results in improper functioning of heavy machinery. Heavy machinery refer to complex machinery like the centrifuges, pumps, temperature regulators, pressure systems and so on. The working of these machinery is very critical and any deviations can cause catastrophic disasters.

	\vspace{3mm}
	Stuxnet was designed to manipulate the ladder logic on which the heavy machinery rely. The idea was to inject or manipulate the controller ladder logic of the controllers. This operation was very slow but the main advantage was the stealthy behaviour of Stuxnet. Stuxnet could go unnoticed even bypassing the anti-virus programs. This ability was provided to the malware through some agencies that specialize in this sector. The malware also passed the Microsoft Software Driver Signing authentication system as it had digital signed drivers that could verify the authenticity. The digital drivers were reportedly stolen from popular driver providers. These special capabilities made Stuxnet the most stealthy malware. Stuxnet makes use of client-side applications as attack vectors on contrary to other malware that use browsers, plugins or web applications. The use of client side applications to as an attack vector is expected by the administrators, however Stuxnet manages to remain stealthy and undetected. Below are the features of Stuxnet malaware.

	\begin{itemize}

	\item\textbf{Multiple Distribution Vectors:} An attack vector is a technique through which an attacker can gain access to a host or a network to deliver a malicious payload. Stuxnet has multiple attack vectors to facilitate its distribution. It makes use of 5 Zero Days from Microsoft Windows to support its propagation. The malware is first injected to the target environment using a USB device containing the malware itself. This was achieved by social engineering methods, for example, giving away free USB drives at a Nuclear Science Conference which had the employees of such plants. Stuxnet spread itself through USB media, hereby exploiting the LNK Zero Day vulnerability.
	
	\vspace{3mm}
	 A Zero-Day vulnerability is an undisclosed computer system vulnerabililty which could be exploited to adversely affect the host programs, data and also other hosts in the network . The other attack vectors that Stuxnet leverages~\cite{murchu2010stuxnet} are Network shared drives, Windows Printer spooler vulnerabilites,  Windows Server RPC vulnerability, \ac{WinCC} Database servers, Step 7 Project files and P2P mechanism. Additionally, Stuxnet also utilizes the vulnerbilities of the Siemens Step 7 software that is used manage the \ac{PLC}s. These are basically rootkit techniques aimed at compromising and pulling down the the target system completely in a stealthy manner.  Stuxnet infects the libraries that Step 7 uses to manage the \ac{PLC}s and replaces it with its own version. 

	\vspace{3mm}

	\item\textbf{Malicious Payload} The payload carried by Stuxnet is not huge but highly effective. It carries subtle payload files, that exploit the vulnerabililties of target system. As discussed previously, Stuxnet employs different attack vectors for its propagation. The payload is initially transferred to the transmission media and then, the malware flags to check if the current host could be the target host. If true, Stuxnet actively begins to compromise the host else, it waits for further propagation. Stuxnet payload consists of three parts. The first is a worm that executes all the methods with reference to the base payload of the attack. The second is a \textit{lnk} which executes the propagated worm files. The third component is the \textit{rootkit} which is responsible for stealthy behavior of the malware.   
	
	
	\vspace{3mm}
	\item\textbf{Code Obfuscation} Code packing is another term for code obfuscation. It is a technique followed that makes binary and textual data unreadable and very complex to understand. This protects the code to be detected as a malware. Obfuscation also makes it difficult for cyber monitors, anti virus engines and even humans to reverse engineer it, to check if the code is a potential malware. Code packing is also done to hide important declarations that reveal insights about malware behavior.There are several algorithms through which obfuscation can be achieved. Stuxnet follows code packing for its entire payload thereby making it difficult to reverse engineer and also detect it.


	\vspace{3mm}
	\item\textbf{Anti-AV Functionality} It is the feature where the malware safely bypasses the anti virus scan without any alarm. Stuxnet gains this feature by its signed software signatures and also because of code obfuscation. These features help Stuxnet to look as normal files and system level code.


	\vspace{3mm}
	\item\textbf{Data Masking} It is described as unauthentic version of an organization's data that is structurally similar. It is used for purposes such as software testing and user training. This results in manipulating original data that may be used by critical systems for controlling and managing heavy machinery. Stuxnet targets data and ladder logic implemented in \ac{PLC}s inorder to compromise the working of critical systems. The failure of critical systems lead to serious consequences that may put human lives under risk.


	\vspace{3mm}

	\item\textbf{Robust Malware Architecture} 
	
	\begin{figure}[h]
	\centering
	        \includegraphics[scale=0.8]{stux-arch}
	        \caption[Stuxnet Architecture]{\label{f:Stuxnet Architecture}Stuxnet Architecture}
	\end{figure}

	Stuxnet has an architecrure as shown in the Figure~\ref{f:Stuxnet Architecture}. An \ac{ICS} network as described in the figure has many hosts which are bound by an internal network. The internal network also connects the hosts to the controllers (\ac{PLC}s) that manage the industrial machinery like motors and pumps. The propagation steps is as below:
	\vspace{3mm}

	
	\begin{enumerate}

	

	\item A USB drive which is infected with Stuxnet is provided to employees of the plant through some public events as goodies. Usually, social engineering techniques are employed to perform this step.

	
	\item The infected USB drive is plugged into a host which is a part of the internal network. Stuxnet propagates itself to the host system by exploiting the zero day vulnerabilities.  

	
	\item Stuxnet checks if the host system is connected to the Internet. If connected, it updates itself from a Command \& Control server.

	
	\item The malware then checks for network shared drives that is mapped into the host machine. If mapped, Stuxnet slowly propagates itself to the shared network drive. Stuxnet also checks if the host has the controlling application installed for programming the controllers. 

	

	\item As discussed earlier, Stuxnet looks for various possible attack vectors like Print spooler service and basically other services that are accessed by many hosts. It then targets such services for propagation to other hosts.

	

	\item Once Stuxnet reaches a controller PC, where there is limited or no  connectivity observed, it validates the system by checking if there an application like Step 7, installed to program the controllers. Stuxnet now tried to establish itself into the application logic to modify the ladder logic pushed into the controllers. This scenario is similar to that of a Man-in-the-Middle attack where Stuxnet's malicious files act as adversaries.

	

	\item The modified ladder logic now controls the heavy machinery. These machines work based on the values input to them by the controllers. If invalid inputs are passed, these critical systems fail to work normally and lead to a disaster.

	

	\end{enumerate}

	The above architecture also depicts the design aspects of Stuxnet. This malware was carefully engineered considering a typical \ac{ICS} environment to facilitate its propagation. It is also accurately implemented to suit multiple attack vectors. 


	\item\textbf{Digital Signature} The applications installed for the Windows operating systems have to be signed with a trusted certificate. The certificates have to be obtained by the vendors to sign their applications. As Stuxnet was to injected to Windows hosts, it had to bypass the signature check mechanism without getting detected. This is possible only by obtaining a legitimate vendor certificate. The developers of Stuxnet stole the certificates of trusted vendors like Realtek, JMicron and imbibed them on the malware files. This way the probability of malware detection is greatly reduced.

	

	\end{itemize} 

	

	

	\subsubsection{Stuxnet-Propagation Techniques}\label{Stuxnet-Propagation Techniques}
	After Stuxnet was detected in June 2010, there were many enthusiastic researchers who studied Stuxnet by reverse engineering, cryptoanalysis, and forensical methods. Security enthusiasts who were curious also conducted research in their private labs to investigate more about Stuxnet as it was declared to be the most sophisticated malware ever. Dissecting Stuxnet was a challenge that many Organizations also looked forward for to optimize their products. Vendors like Siemens and Microsoft were on their peril waiting for researchers to come up with their theories.  Nicolas et al \cite{falliere2011w32} from Symantec Labs presented a dossier about Stuxnet and its propagation techniques. The W32.Stuxnet Dossier provides the attack scenario, timeline, a survey of infected hosts and organizations throughout the world and Stuxnet architecture. Stuxnet relies on vulnerabilities to exploit the plant. Stuxnet is dormant without these vulnerabilities. It utilizes the Windows lnk exploit, 4 Zero days of Windows and the Siemens Step 7 software vulnerabililty. In what follows, we discuss various propagation methods that Stuxnet employs to reach its target system.


	\vspace{3mm}
	Stuxnet does not have a dedicated process. It hides behind trusted processes. Table ~\ref{Trusted Processes} shows a list of trusted processes under which Stuxnet hides. In the course, it injects into these trusted processes. Once established Stuxnet implements a Microsoft \ac{RPC} server and client by exploit the Zero days and performs automatic updates in LAN. The \ac{WinCC} is used for supervision and control of Siemens industrial systems. Microsoft \ac{SQL} is used for logging. The password is hardcoded in the \ac{SQL} server and is publicly available. This vulnerability is leveraged by Stuxnet to gain access as an administrator and modifying the tables with malicious \ac{DLL} representation.
	
	\begin{table}[H]
	\centering
	\caption{Trusted Processes}
	\label{Trusted Processes}
	\begin{tabular}{|l|l|}
	\hline
	\textbf{Product}       & \textbf{Target Process} \\ \hline
	Kaspersky KAV          & avp.exe                 \\ \hline
	Mcaffee                & Mcshield.exe            \\ \hline
	AntiVir                & avguard.exe             \\ \hline
	BitDefender            & bdagent.exe             \\ \hline
	Etrust                 & UmxCfg.exe              \\ \hline
	F-Secure               & fsdfwd.exe              \\ \hline
	Symantec               & rtvscan.exe             \\ \hline
	Symantec Common Client & ccSvcHst.exe            \\ \hline
	Eset NOD32             & ekrn.exe                \\ \hline
	Trend Pc-Cillin        & tmpproxy.exe            \\ \hline
	Windows                & Lsass.exe               \\ \hline
	Windows                & Winlogon.exe            \\ \hline
	Windows                & Svchost.exe             \\ \hline
	\end{tabular}
	\end{table}
	
	
	Stuxnet propagates using three mechanisms:
	
	\begin{enumerate}
	\item It infects the host using infected removable media (such as USB flash drives and external portable hard disks);
	\item It propagates through the \ac{LAN} to distribute itself to all hosts having access to the network drive and print spooler. 
	\item Injects into the Siemens Step 7 software controller logic files and propagates to the \ac{PLC}s. It injects into the ladder logic of the \ac{PLC}s.
	
	\end{enumerate}
	
	Leveraging any of the above three vectors, it uses seven vulnerability exploitation techniques for distributing to new hosts in a network.
	
	
	\begin{enumerate}
	
	\item Exploits a zero-day vulnerability in Windows Shell handling of LNK files.The vulnerability was present in all versions of Windows since Windows NT 4.0. 
	\item Stuxnet uses several techniques to try to replicate itself to accessible network shares and distribute itself through the network.
	\item Injects itself to printer servers using a zero-day vulnerability.
	\item It uses an older Conficker \ac{RPC} vulnerability to propagate through unpatched Windows computers as shown in Figure~\ref{f:Stuxnet RPC Propagation}. The \ac{RPC} infected client requests newer Stuxnet version from an infected Stuxnet RPC server. 
	\item It contacts Siemens \ac{WinCC} SQLServer database servers and installs itself on those servers through database calls. 
	\item Injects copies of itself into Siemens STEP 7 project files to auto-execute whenever the files are loaded.
	\item An earlier version of Stuxnet used an old variant of \textit{autorun.inf} exploit~\cite{larimer2011beyond} to propagate through USB drives. 
	
	\end{enumerate}
	
	
	\begin{figure}[H]
		\centering
		        \includegraphics[scale=0.6]{Stux-rpc}
		        \caption[Stuxnet RPC Propagation]{\label{f:Stuxnet RPC Propagation}Stuxnet RPC Propagation}
	\end{figure}
	
	\subsubsection{Malware Detection Design}\label{Malware Detection Design}
	After a thorough analysis of the propagation techniques of Stuxnet, identifying a environment to facilitate its propagation considering HosTaGe ICS is challenging. The most feasible technique that could be simulated by HosTaGe is the network share environment. We realise this environment through protocol simulation feature of HosTaGe ICS. HosTaGe ICS will contain the Master profile that emulates a network drive under the same /24 network. Further, the files propagated will be verified for malware. We discuss the actual simuation mechanism in section ~\ref{Implementation} 
	
	
	\subsection{Summary}
	This section describes the proposed system for HosTaGe ICS which aims to detect \ac{ICS} specific attacks, multistage attacks and generation of signatures for Bro \ac{IDS}. The section provides architecture for HosTaGe ICS with explanation for components HosTaGe Core, logger, Graphical User Interface and services. The HosTaGe core forms the core mechanism of HosTaGe ICS which consists of Emulator and Connection Guard. The logger is responsible for logging the attacks and connection data in SQLite database. We explain how port binding is handled in HostaGe and also mention the services that HosTaGe offers. The section also explains about protocols HTTP, Telnet, FTP, SNMP, SMTP, SMB, Modbus and S7. The section provides a description on design of Formal model of attacks, that represents attack stages. 
	
	
	\vspace{3mm}
	Further, the detection mechanisms is classified into three types of attack classes which includes \ac{SPLD}, \ac{MSLD}, and \ac{PLD}. A brief description on the design of signature generation , where HostaGe ICS makes use of attack results logged by HosTaGe is discussed. An overview of the master and slave profiles of \ac{SCADA} with the necessity to check for internal and external attacks. Further,  a brief explanation and study on Stuxnet malware is provided with its features and propagation mechanism. We also discuss the various attack vectors that Stuxnet employs to reach the target system.
	

	
		\newpage
	   \section{Implementation} \label{Implementation}

		We discuss the architecture, features and protocols offered by the Siemens Simatic S7 200 \ac{PLC} and also security concerns of \ac{ICS} \ac{SCADA} systems in section~\ref{Proposed System and System Design} . There were many exploit areas that were discovered. The \ac{PLC} was subjected to various exploits and attacks. However, large scale attacks like Stuxnet were successful because of vulnerabilities that existed on the Host controllers as well, that is, Windows OS hosts.  It made use of zero day exploits from both Windows OS and the Siemens \ac{PLC}s. The attack was well designed and strategized considering vulnerabilities on both systems. There are also small attacks like information leakage from an Internet facing \ac{PLC}, hosting a webserver. Over the years many vulnerabilities have been identified on the Siemens \ac{PLC}s. It becomes a great challenge to make these systems secure. The \ac{PLC}s have limited resources and thereby security measures like data encryption may prove expensive. Hence, data encryption was avoided. This decision of ignoring secure features induced several exploits for the device. 
		
		\vspace{3mm}
		The honeypot must be designed keeping all the discovered exploits inorder to be more effective in attracting the attackers. We consider both the external and internal attack approaches, hereby devising strategies to capture both kind of attacks. Before we design our honeypot, it is very important to understand the previous known attacks on \ac{PLC}s, their impact and the vulnerabilities that caused those attacks.
	 
	\vspace{5mm} 
 	\subsection{HosTaGe ICS honeypot} \label{HosTaGe ICS honeypot}
 	
 	Based on the design decisions made in the section~\ref{Proposed System and System Design}, HosTaGe is  implemented to simulate the services of the Siemens Simatic S7 200 \ac{PLC}. The services include the simulation of HTTP, Modbus, \ac{S7}, \ac{SMTP}, Telnet, \ac{FTP}, \ac{SNMP} protocols with respect to the \ac{PLC}. HosTaGe also supports creation of honeypot environment profiles. We leverage this feature to create profiles of nuclear power plant that includes a Modbus slave open to the Internet with the other protocols. The services offered by the other protocols are also customized based on the profile chosen by the  user. The services implementation involve handling incoming connection requests on respective ports of the protocols. The main objective of a honeypot is to keep the attacker engaged and allow him to pursue his attack strategy. A proper response mechanism is required which provides satisfying responses to the attacker commands and packets. For example, if the attacker tries to access the webpage by sending a GET request to port 80 of the honeypot, a webpage must be displayed to the attacker. Further, the webpage must also be dynamic depending on the honeypot profile chosen on HosTaGe. This feature has been implemented to keep the attacker engaged to HosTaGe while all the packets are captured and stored as records in HosTaGe attack database.
 	
 
 	We also implement the detection of malware propagation through the \ac{SMB} protocol. This detection forms an important aspect of detecting popular malware like Stuxnet through the network.  This forms as an extension to the implementation of the \ac{SMB} protocol for the Modbus Master profile. As discussed in the exploit areas, the most dangerous incident reported with respect to \ac{ICS} \ac{SCADA} systems is Stuxnet. We try to replicate the scenario to facilitate Stuxnet propagation by simulating HosTaGe as a shared network drive. We discuss this module further in the section~\ref{Stuxnet Propagation}.
 	
 	
 	During the course of thesis, an attack pattern was recognized and observed amongst the attacks recorded by HosTaGe. The attacks were later analysed to be Multistage attacks. Multistage attacks are attacks that originate from the same source \ac{IP} and attempt to attack multiple active protocols on the target system within a time window. This helps in reducing false positives and acts as a means to detect genuine attacks. The detection of Multistage attacks labelled as \ac{MSLD} in section~\ref{Detection Mechanisms} has been implemented based on the formal model shown on Figure~\ref{f:EFSM}. The approach to detect \ac{MSLD} is discussed further in section~\ref{Multistage Attack Detection} 
	
	
	The signature generation module is one of the important component. It focuses on creation of signatures for an IDS based on the attack data captured and stored on HosTaGe. Signatures are generated and exported as files which can be mounted on an IDS for signature based traffic analysis. The creation of policies or rules is also implemented for the detection of \ac{MSLD}. The implementation is further discusses in section~\ref{Signature Generation Module}.
	
 
	\subsection {Detecting Internal Attacks}\label{Detecting Internal Attacks}

	As discussed previously, \ac{ICS} \ac{SCADA} systems have master and slave profiles. Though the devices are subjected to attacks from external attacks, when made open to the Internet, it is proved that major attacks in the past were triggered by systems in the internal network. Attacks from malware such as Stuxnet spread from host systems in the same network. Attacks from internal systems have proved to be more effective and dangerous as they do not leave any fingerprints, also their signature cannot be identified by the anti-virus softwares and other protection tools. The Stuxnet worm was reported to be injected through a USB flash drive. It made use of zero day vulnerabilities of the Windows operating system, the most popular one being how the Windows operating system handles the LNK \cite{matrosov2010stuxnet} files, which are used by the operating system to interpret devices capable of AUTORUN functionality, and to detect the software to run the file based on its format. 

	An anatomy of similar kind of viruses and malware revealed that they made use of as many zero day vulnerabilities as possible to make the malware attack more effective and stealthy. Identifying such malware attacks through our honeypot mechanism is a challenge, as it involves careful design and simulation of services involved in such attacks. To achieve this, the conditions under which such worms propagate and try to sneak into the network is studied. Analysis of the studies made by researchers \cite{Langner:2011:SDC:1990763.1990881} shows that the worm looks for different attack vectors, exploits the zero day vulnerabilities of the Windows \ac{OS} and also the \ac{DLL} of the \ac{PLC} vendors.  


	As discussed above Stuxnet exploits the zero day vulnerabilities on a Windows host and is dormant without it. Hence it is required to simulate atleast one of the zero day vulnerability. \textit{Kolesnichenko et al.} \cite{kolesnichenko2011quantitative} in their research try to perform a Quantitative Analysis of Stuxnet to find out its feasibility and the required conditions to increase its precision. The best suited amongst the five was the propagation through the network shared drive. This service could be simulated like on a WebDav server. We could then wait for the virus to propagate itself into this simulated location. 
 
	\subsection{Malware Detection}\label{Malware Detection}
	Malware detection is a crucial component of HosTaGe \ac{ICS}.  As discussed in section~\ref{STUXNET Malware- A study} our main goal is to detect Stuxnet in an \ac{ICS} environment. To incorporate this, we simulate the required environment for its propagation. We leverage the network share, one of the attack vector used by Stuxnet for its propagation\cite{chen2011lessons}.The \ac{SMB} protocol also known as \ac{CIFS} is an application-layer network protocol used for file sharing, printer sharing and also certainly in Active Directory services. The protocol was initially based on Microsoft Windows operating system and was called the Microsoft Windows Network as it was a domain binded protocol. \ac{SMB} provided communication between hosts in an internal network for information exchange.  Stuxnet also uses printer spooler service as one of its attack vectors.
	
	
	We thus infer to implement \ac{SMB} protocol to simulate the network share drive or the print spooler attack vectors. The protocol emulation is a part of the simulation of Modbus Master profile. This profile emulates a set of default protocols that are open on a Windows host. Along with Modbus protocol and \ac{SMB}, the target system simulated resembles to a Master host in an \ac{ICS} \ac{SCADA} network. Stuxnet propagates from a host in an internal network to all hosts until it finds a host with direct access to a controller. The simulated Modbus Master provides the behavior required by Stuxnet for its infection. An environment is created to evaluate the detection of Stuxnet which is further discussed in the evaluation section~\ref{Stuxnet Propagation}.  Stuxnet propagates to the network share using the \ac{SMB} protocol. 
	
	
	Once the malware has propagated into the network share, it is detected by the file injection detection module. Furthermore, a hash of every file injected is calculated. This hash is verified by sending it to the VirusTotal database which returns if the file is a malware. The VirusTotal repository provides an \ac{API} to perform this validation. If the injected file is found to be a malware, it also returns the name of the malware, along with the inferences made with popular AntiVirus. 
	
	\begin{figure}[H]
		\centering
		\includegraphics[scale=0.3]{HosTaGeFile}
		\caption[HosTaGe ICS File Injection]{\label{f:HosTaGe ICS File Injection}HosTaGe ICS File Injection}
	\end{figure}
	
	
	\vspace{3mm}
	Figure~\ref{f:HosTaGe ICS File Injection} shows the File Injection detected by HosTaGe ICS with the attack results. The details of the file injected is shown in the CONVERSATION. It contains the the name of the file injected, checksum and details of the file as identified by Virustotal. If the file is found to be a malware, like in the figure~\ref{f:HosTaGe ICS File Injection},  Virustotal  responds with the details of the malware based on its malware database with respect to the anti virus vendors. We take into account the major players in the anti virus providers such as Kaspersky, Microsoft, McAffee, Symantec, AVG, TrendMicro and QuickHeal. The hash of the file is calculated to verify the actual contents of the file and not just the name of the file. This also reduces false positives of the attackers who just change the file names to threaten systems with a file injection.
	
	\vspace{3mm} 
	 Validation by the various Anti Virus providers ensures better analysis and inference. The Figure~\ref{f:HosTaGe ICS File Injection} shows the identification of a file as a part of the Stuxnet malware. The validation from various Anti Virus also reduces false positives and also provides conformation.  The Signature generation module is enabled for file injection. The signature file generated validates incoming packets for the signature, hereby identifying the malware injection and propagation. Identification of malware propagated through internal network forms a good defence mechanism in \ac{ICS} systems. Popular malware designed for \ac{ICS}, propagate to the target system by distributing itself to the internal network. The attack record also shows that the attack was from a Internal Network. 

	\subsection{Multistage Attack Detection}\label{Multistage Attack Detection}
	
	Multistage attack detection involves identifying multiple protocol attacks from an attacker in a specific time window. We use the attacker \ac{IP} address, the time of attack and the protocol targeted to identify a multistage attack. These attacks filter false positives amongst many attacks detected by HosTaGe ICS. As mulistage attacks involve attacks in a sequence, it can be formally modeled. This representation of both the attacking and detection strategy provides an overview to implement the multistage attack detection service.
	We adapt the formal state machine model discussed in section~\ref{Formal Model} to implement multistage attack detection. We consider time to be an important part in our detection mechanism as it eliminates false positives. Multistage attacks are identified by the attack correlation mechanism.


	\vspace{3mm}
	 The steps involved in the detection of multistage attacks is as follows:
	\begin{enumerate}
	
	\item All the attacks for a specific time period are fetched from the attack database. 
	\item The attacks are sorted based on the source \ac{IP} address. (attacker \ac{IP} address)
	\item Once sorted, the attacks are checked for the targeted protocols. If the target protocols are the different, a multistage attack is logged.
	\item The attacks are checked on sliding window basis to get lesser false positives.
	
	\end{enumerate}
	
	\begin{figure}[H]
		\centering 
		\includegraphics[scale=0.3]{HosTaGeMul}
				\caption[HosTaGe ICS Multistage Attack Detection]{\label{f:HosTaGe ICS Multistage Attack Detection}HosTaGe ICS Multistage Attack Detection}
		\end{figure}
	
	The logged multistage attacks contain the attacker \ac{IP} address and the targeted protocols with the timestamp. An analysis of the individual protocols can be performed with the attack logs of the protocols saved.  Significant observations can be made by every multistage attack as they provide a basis of target specific attacks towards an organization. These attacks remain undetected by IDS. Multistage detection service is implemented as a background service in HosTaGe ICS. When it is enabled, the attack logs are constantly scanned for multistage attacks. This service is invoked every fifteen minutes with the attacks fetched for the last twenty minutes. This service can also be disabled through the settings option of HosTaGe ICS. 


	\vspace{3mm}
	Figure~\ref{f:HosTaGe ICS Multistage Attack Detection} shows the multistage attack detection by HosTaGe ICS. The record shows an overview of the attacker \ac{IP} address and the protocols targeted. The attacks can be individually analyzed by selecting the respective attack record from the \textit{Attack Records} fragment. The above record can also be used to generate policies for the Bro \ac{IDS}. The policy generated checks incoming packets for attacker \ac{IP} address and the protocols targeted. The time frame is also included to avoid false positives. 
	
	
	\vspace{3mm}
	Listing~\ref{lst:Bro Policy for Multistage Attack detected by HosTaGE ICS} shows a policy generated by HosTaGe for a multistage attack detected.  Further, Bro provides datatypes such as subnet and vectors which help in implementation of network specific data. Bro has pre-defined modules such as \textit{connection\_established} where the task to be performed on a connection establishment could be specified. The logic derives the attacker \ac{IP}, subnet and the protocols attacked from HosTaGe ICS multistage attack record. HosTaGe ICS provides a method for mapping the protocols attacked to their respective ports. This information is passed to the policy generation module. 


	\vspace{3mm}
	Furthermore, we check if the source \ac{IP} of the connection request is the same as the attacker \ac{IP}, its subnet and protocol port as identified by HosTaGe ICS. We also specify the time interval to avoid false positives. On receiving the next connection, the policy checks for the next protocol specified in the protocol vector. If the source \ac{IP} matches the attacker \ac{IP} and the attacked protocols are as specified in the protocol vector, a multistage attack is detected by the policy.  We represent the following as an algorithm below in Listing~\ref{lst:Algorithm for Multisage attack detection policy}
	
	
	\begin{lstlisting}[caption=Algorithm for Multisage attack detection policy,label=lst:Algorithm for Multisage attack detection policy]
	
	Step 1:  Start
	
	Step 2:  Fetch attacker IP, attacked protocols array from HosTaGe ICS attack record.
		
	Step 3:  Set count=0; Wait for new connection.
		
	Step 4:  On new connection; for i in attacked_protocols_array
		If (source_ip == attacker_ip) go to step 5, 
		 else goto step 3
	
	Step 5:  If count = 0, go to Step 6. Else, goto Step 7.
		
	Step 6:  if {(target_protocol == attack_protocol_array[count])
		 count ++;
		 goto Step 4	
		}
	
	Step 7:  if {(target_protocol == attack_protocol_array[count])
		 print (Multistage attack detected)	
		}
		end for
	
	Step 8: End
	
	
	\end{lstlisting}
	
	\vspace{3mm}
	
	There were certain significant attacks observed during the evaluation period. A brief description about these results are discussed in the section~\ref{Evaluation}. 
	
	
	\vspace{3mm} 
	\subsection{Signature Generation}\label{Signature Generation Module}
		As discussed previously in section~\ref{Signature Generation Module}  Signature Generation module generates signatures for the attack traffic that is logged. We consider the Bro Network Security Monitor for generating signatures. The signature generation module is a combined model involving modules of HosTaGe. Additional modules have been implemented to support signature generation for the Bro \ac{IDS}. We highly make use of Bro's features to incorporate the signatures generated through HosTaGe. A brief discussion about the Bro \ac{NIDS} and the signature generation is provided in the below sections~\ref{Bro Network Security Monitor},~\ref{Signature Generator}.
		
		
		\vspace{3mm}
		\subsubsection{Bro Network Security Monitor}\label{Bro Network Security Monitor}
		Bro is an opensource, Unix based powerful network monitor that provides a strong analysis framework.  It provides a good platform for traffic analysis, signature matching and network forensics. It is originally written by \textit{Vern Paxson}, and now maintained by researchers at International Computer Science Institute, Berkeley, CA and worldwide.  
		Bro is implemented using the Bro scripting language. it is often associated to being a framework and could be used to build effective \ac{IDS}. Bro is open-source and provides lot of documentation with researchers involved from all over the world. 
		
		
		Bro can analyze traffic both live and offline to perform analysis, take network measurement, forensic investigation and traffic baselining.		
		Bro has many features that aim at building better IDS. The features are discussed below:
		
		\begin{itemize}
		
		\item\textbf{High-Speed and large volume monitoring:} Bro is capable of handling huge amounts of traffic. It is also quick in analyzing the traffic for the policies and signatures. Bro can also be deployed to work as a distributed
		security monitor, where it can share states and also propagate event data to notify other distributed peers.
	
		\item\textbf{No Packet Filter Drops:} As discussed in the previous point, Bro is capable of handling huge traffic flows. There is very minimal packet drops observed.
		If an application using a packet filter cannot consume packets as quickly as they arrive on the observed link, the filter buffers the packets for later consumption.
		However, the filter may run out of the buffer and at such point it drops further packets that arrive. From a security monitoring perspective, drops can cause inconsistent monitoring, 
		because the dropped packets might contain interesting traffic that identifies a adversary. Thus, Bro makes sure that the packet drops are minimized, thus making it possible to capture traffic ensuring continuity.
		
		\item\textbf{Real Time notification:}
		Detecting attacks online as they happen is very much useful in taking relevant steps for mitigation and notification in real time. Offline detection leads to delay in analyzing the attack and being proactive.
		Bro supports providing real time notifications of attacks. An attack detected live provides better trace back capabilities and to minimize damage.   
		
		\item\textbf{Independent Policies:}
		Bro separates the working mechanism from its policies. This offers more flexibilities in customizing the policies and not the mechanism. This also leads to simplicity in analyzing the policies and rewriting them. Policies could also be referred to as rules as in a IDS representation.
		
			
		\item\textbf{Language Support:}
		The Bro language is designed keeping network communications as a base. It offers datatypes for network basics like \ac{IP} address, subnet, port, source \ac{IP}, destination \ac{IP}, source port and destination port. It also supports conditional and non-conditional programming keywords like in , when and match which are very beneficial.
	
		\item\textbf{Scalable and Extensible:} As Bro can be deployed as a distributed environment, it can easily scale to monitor huge networks and share state information. Bro has coordinators, peers and master systems to manage the Bro deployments. 
		Bro is opensource and this enables to extend the platform for independent researchers. This is also provides the freedom to develop and test new policies and modules. 
			
		
		\item\textbf{Protocol Support:} 
		Bro processes both TCP and UDP packets. For each TCP packet, the connection handler checks the entire TCP header and validates the TCP checksum for the packet header and payload. 
		
		
		
		\end{itemize}
	
		         
		        \begin{figure}[H]																					
		           \centering
		           \includegraphics[scale=0.90]{bro-arch}
		           \caption[Bro's Architecture]{\label{f:bro-arch}Bro's Architecture~\cite{Bro}}
		           \end{figure} 
		
		Bro's architecture involves Bro Event Engine, Policy Script Interpreter and libcap. Figure~\ref{f:bro-arch} depicts the architecture of Bro and the analysis framework. The role of each component is described below:
		
		\begin{itemize}
		
		
		\item\textbf{libcap:} libcap is the packet capture library used by Bro. Libcap isolates Bro from details of network link technology. It also enables to Bro to work offline by accepting packets to analyse threats and malicious content.
		
		\item\textbf{Event Engine:} This layer performs integrity checks to assure that packet headers are well-formed and also verifies the \ac{IP} header checksum. Failure of checks generate events indicating the issue resulting in dropping of packets. If the checks succeed, then the event engine fetches the connection state with reference to the tuple of the two \ac{IP} addresses and the two TCP or UDP ports, creating new state if no states exist. It then dispatches the packet to a handler for the current connection. Bro maintains a tcpdump trace file associated with the traffic it sees. The connection handler decides, if the engine should record the entire packet or just its header to the trace file. Upon return whether the engine should record the entire packet to the trace file or just its header, or nothing at all.
		
		\item\textbf{Policy Script Interpreter:}
		The event engine checks if any events have been raised. If any events are generated it processes the events as per the event handler specified until last event.  Bro's emphasis on asynchronous events as the connection between the event engine and the policy script interpreter, provides lot of extensibility. Adding new functionality to Bro generally consists of adding a new protocol analyzer to the event engine and then writing new event handlers for the events generated by the analyzer. This holds true also for the policies and the signature. We  leverage this feature in HosTaGe to generate policies and signatures for Bro.
		
		
		\begin{figure}[H]
		\centering
		\includegraphics[scale=0.3]{hostagesig}
		\caption[HosTaGe ICS Signature Generation]{\label{f:HosTaGe ICS Signature Generation}HosTaGe ICS Signature Generation}	\end{figure}
		
		Figure~\ref{f:HosTaGe ICS Signature Generation} shows the signature generation Module of HosTaGe ICS. The module has been enabled for Modbus, S7, File Injection and multistage attacks. The screen HosTaGe Records shows an icon towards the top right corner. On press of this icon, a dialog box confirming the signature generation appears. The signature is generated and saved in the phone external memory after the user confirmation. This signature can be mounted into Bro IDS.
		
		
		
		\end{itemize} 
		
		\vspace{5mm} 
		\subsubsection{HosTaGe Records}\label{HosTaGe Records}
		HosTaGe stores all the packet data it receives in the form of records. The records include source \ac{IP}, source port, attack type, protocol and the packet conversation. The data stored as conversation is the payload of the connection packets. This packet data can be considered for pattern matching and signature generation. This data is passed as a parameter for signature generation module. 
		
		
		 
		\vspace{5mm} 
		\subsubsection{Signature Generator}\label{Signature Generator}
		The Signature generator contains pre-defined templates which are implemented for specific protocols like Modbus and \ac{SMB} for File Injection. These templates accept parameters from the conversation information stored. It also accepts the source \ac{IP} address as a parameter. The source \ac{IP} address is included to avoid false positives. As discussed previously in Bro's architecture, there are also policies that Bro uses to check the incoming traffic. The signature generator module also generates policies for the multistage attacks detected by HosTaGe. In the case of multistage attacks, the conversation information is not considered for the signature generation. The source \ac{IP} address and the protocols attacked are considered for the generation of policies for multistage attacks.
				
		\vspace{3mm}
		Figure~\ref{Signature Generator Class Diagram} depicts the modules involved in the generation of signature. Each module consisting of attributes and methods that are explained in this part of the report.
		The module Hostage is connected to module Listener. Hostage creates listeners for the protocol enabled by the user. The module Hostage consists of attributes which includes listener, context and connection information. The Hostage defines methods to start and stop listener, to notify UI and to get connection method. The attributes declared in the Listener module includes protocol, port, service, server and connection register. The module defines methods to start and stop service, and to log attacks. Each listener module can have one or more type of records. 
		
		
		\vspace{3mm}
		There are three types of records which include Attack record, Message record, and Network record. All three types of record merge to form a single module called Record module. An attack record consists of attackId, bssid, device, protocol, local\ac{IP}, remote\ac{IP}, localPort and remotePort. It defines methods to record the attack and write to Parcel. The module Message record consists of attackId, timestamp, id, type and packet with methods defining the messageRecord and writeToParcel. The Network record has bssid, ssid, timestamp , longitude, latitude and accuracy. It also consists of method that defines writeToParcel, network record and a method to set accuracy.
	
	
		\vspace{3mm}
		The three types of records are integrated in the Record module based on their respective attackId. The integrated information is further utilized in generation of signature. The signature generated module consists of methods that defines signature generation, and its file, methods for conversion and to get integrated record.  
	
	
		\vspace{3mm}
		Listing~\ref{lst:Bro Signature for Modbus service through Metasploit script} shows the generated signature for a Metasploit script that  checks for Modbus service on the target system. The pattern specified is checked for all incoming packets for specific contents. If the incoming packet matches the signature, an event is raised. The administrator can specify how to handle event. 
	
	
		\vspace{3mm}
		\begin{lstlisting}[caption=Bro Signature for Modbus service through Metasploit script,label=lst:Bro Signature for Modbus service through Metasploit script ]
		signature modbus-signature1 {
		    ip-proto == tcp
		    dst-port == 502
		    payload /\x21\x00\x00\x00\x00\x06\x01\x04\x00\x01\x00\x00/
		    event "Modbus attack detected!"
		}
		\end{lstlisting}
		
				

	\vspace{5mm} 
	\subsection{Logging Mechanism}\label{Logging Mechanism}
	
		
		Data analysis of the attacks received also form an important part of a honeypot. This feature enables administrators to carry out forensics on the attack data and check for new malware types. Through attack analysis the attack vectors can be mapped which help us understand the strategies followed by attackers to compromise systems. 
		The Attacks Log has an entry of all the incoming and outgoing connection information. Each connection received to HosTaGe is further divided as attack record, network record and the message record. 
	
	\newpage
	
		\begin{figure}[H]
					\centering
					\includegraphics[scale=0.8]{SigGen}
					\caption[Signature Generator Class Diagram]{\label{Signature Generator Class Diagram}Signature Generator Class Diagram}
					\end{figure}
			

	\subsection{Summary}
	In this section we focused on implementation of the HostaGe ICS honeypot. HosTaGe ICS is implemented to simulate the services of Siemens Simatic  S7 200 PLC. The detection of malware through \ac{SMB} protocol is also implemented which explained about detection malware like Stuxnet in the network.  Furthermore, we looked into detection of malware and its validation which was one of the crucial component in HosTaGe ICS. In this part of the section we explained the detection of malware once it is propagated into network share. 
	
	We also overview how multistage attack is detected by the HostaGe ICS which involved detection of multiple protocol attacks from an attacker in a particular time window. We briefly explain the algorithm of multistage attack detection which involved four main steps. We explain the implementation of signature generation which generated the signatures for the attack traffic that was logged. A brief discussion on Bro IDS is made, which discusses some features of Bro . 
	
	
	\newpage
 	\section{Evaluation} \label{Evaluation}
 		
 	In this section, we evaluate the detection capability and performance of HosTaGe ICS. Evaluation of the detection capability involves the analysis of the attacks received on individual protocols emulated by HosTaGe ICS. Further, the results obtained from the comparison of HosTaGe ICS with Conpot is evaluated. Detection of multistage attacks and signature generation were also the focus of the thesis. We evaluate the interesting results obtained through the multistage attack detection service and the signatures generated for the Bro \ac{NIDS} by HosTaGe ICS. We also evaluate the environment setup for the propagation of Stuxnet and detecting it through HosTaGe ICS. HosTaGe ICS was deployed on a mobile device with resource constraints. We discuss the evaluation and performance of HosTaGe ICS in the performance evaluation. We also evaluate the impact of Shodan services and probes received and thereby determine the detectability of HosTaGe ICS.
  
 	
 	\subsection{Experimental Setup}\label{Experimental Setup}
 	There were two experiments carried out that involved setting up of required environments for the evaluation.
 	The two environments are discussed below. 
 	
 	\vspace{3mm}
 	The first experiment was to setup an environment for the evaluation and comparison of Conpot with respect to HosTaGe ICS. Figure~\ref{Conpot and HosTaGe environment setup} shows the environment setup for the evaluation of HosTaGe ICS with Conpot. HosTaGe ICS was deployed on a rooted Samsung Galaxy S4 mobile phone. Conpot an \ac{ICS} specific honeypot was deployed on a Raspberry Pi using an image file, Honeeepi~\footnote{https://redmine.honeynet.org/projects/honeeepi/wiki}. Honeeepi has a collection of honeypots and profiles which can be customized based on the network. Both honeypots were setup on a /24 network and exposing the devices to the Internet by lifting rules on a firewall. The services emulated on these devices were publicly open to the Internet. The devices were behind NAT and all the packets were routed to the honeypots from the router using port forwarding. This helped us to have a static \ac{IP} address on both devices that could be accessed publicly. The \ac{IP} addresses of the devices were on the subnet. This gave a feeling to the attackers of a serial line connected to two devices that are open to the Internet. 
 	
 	\begin{figure}[H]
 						\centering
 						\includegraphics[scale=0.5]{conhost}
 						\caption[Conpot and HosTaGe environment setup]{\label{Conpot and HosTaGe environment setup}Conpot and HosTaGe environment setup}
 						\end{figure}
 	
 	\vspace{3mm}
 	Conpot logs all the connection data onto a log file. The attack based on HTTP, Modbus, S7 and SNMP protocols is logged and all other connection requests that is not relevant were discarded. The logs contain the source \ac{IP} address, the timestamp and the protocol attacked. The logs of protocols like Modbus and S7 contain header information of the attacks. 
 	
 	\vspace{3mm}
 	HosTaGe ICS logs all the attacks in its database. HosTaGe ICS has two \ac{ICS} specific profiles. These profiles emulate the master and slave devices with the respective protocols. The master profile extends simulation of a Windows XP host with Modbus, S7 and SMB protocols, while the slave simulates a \ac{PLC} with Modbus, S7, SNMP, SMTP, FTP, HTTP and Telnet protocols. All the connection attempts to these protocols are logged as attack records in HostaGe ICS database. Furthermore, HosTaGe ICS is also able to detect file injection, portscan and multistage attacks. The attack information logged by HosTaGe provides lot more information when compared to Conpot. The attack records contain the conversation or the message exchange taking place between the attacker and the honeypot hereby providing better analysis opportunities. The detailed summary of the attacks logged by Conpot and HosTaGe ICS are described in the forthcoming sections. 
 	
 	\vspace{3mm}
 	The second environment was setup for the evaluation of the signature generation of HosTaGe ICS. Bro, a powerful network analysis framework that could be customized to be an IDS. It offers powerful customization and logged mechanisms for administrators. The signatures generated by HosTaGe is deployed on the Bro network security monitor. 
 	
 	\vspace{3mm}
 	Figure~\ref{Bro and HosTaGe ICS environment setup} shows the environment setup of Bro with HosTaGe ICS. Bro was setup on a Debian virtual machine. Conpot also was installed in the virtual machine to attract more attacker traffic. Bro logs all the connection attempts to the system. Logs are further classified based on TCP, UDP and also on various protocols. Logging can be customized by the administrator based on events raised by signatures and policies. Bro is shipped with many active policies and signatures by default. There are policies for HTTP, HTTPS, Modbus, SNMP and many more~\cite{Bro}. 
 	
 	\begin{figure}[H]
 	 			\centering
 	 			\includegraphics[scale=0.5]{brodiag}
 	 			\caption[Bro and HosTaGe ICS environment setup]{\label{Bro and HosTaGe ICS environment setup}Bro and HosTaGe ICS environment setup}
 	\end{figure}
 	
 	
 	There are also policies that are classified as weird traffic based on some experiments by Bro. Bro has two traffic operation modes. Firstly, Bro can be setup to monitor live traffic, thereby applying desired policies and signatures. This option is dynamic and works like an active \ac{NIDS}. Malicious packets that are identified can also be dropped before entering the internal network.Secondly, Bro can be used to analyze offline traffic which can be fed as pcap files. The pcap files can be checked for policies and signatures. Hence Bro supports both online and offline traffic analysis. We leverage this to evaluate the signatures generated by HosTaGe ICS.
 	
 	\vspace{3mm}
  	Bro provides a flexible framework for evaluation and analysis of traffic. The signatures generated for Bro were also enabled for live traffic monitoring to check for accuracy by attacking the system in realtime. The signatures were accurate in detecting the attacks.
 	
 	\subsection{Detection Evaluation}\label{Detection Evaluation}
 	
 	The main aim of HosTaGe ICS is to detect \ac{ICS} specific attacks efficiently. HosTaGe ICS provides mechanisms to detect attacks with respect to individual protocols and also to detect popular attack strategies such as portscan, file injection and multistage attacks. It is very important to evaluate the results obtained using these mechanisms inorder to determine the efficiency and productivity of the application. We compare HosTaGe ICS with Conpot, an interactive \ac{ICS} honeypot, to evaluate the detection capability. Further, HosTaGe offers detecting malware that is propagated through protocols that are open in \ac{ICS} networks. We evaluate the detection of Stuxnet, a popular and devastating malware in \ac{ICS} SCADA systems. The evaluation of the featured capabilities discussed above is summarized in the following sub sections.
 	
 	\subsubsection{Analysis of Individual Protocol Attacks}\label{Analysis of Individual Protocol Attacks}
 	
 	\begin{table}[H]
 	\centering
 	\caption{Attack results per Protocol in HosTaGe ICS}
 	\label{tb:Attack results per Protocol in HosTaGe ICS}
 	\begin{tabular}{|l|l|l|l|l|l|}
 	\hline
 	\multirow{2}{*}{Week}                 & HTTP        & MODBUS      & TELNET      & S7 Comm     & SMTP        \\ \cline{2-6} 
 	                                      & HosTaGe ICS & HosTaGe ICS & HosTaGe ICS & HosTaGe ICS & HosTaGe ICS \\ \hline
 	Week 1 July 7th-12th                  & 83          & 9           & 177         & NA          & NA          \\ \hline
 	Week 2 July 13th-19th                 & 174         & 10          & 283         & NA          & NA          \\ \hline
 	Week 3 July 20th-26th                 & 127         & 11          & 514         & NA          & NA          \\ \hline
 	Week 4 July 27th-August 2nd           & 79          & 11          & 260         & NA          & NA          \\ \hline
 	Week 5 August 3rd-August 9th          & 83          & 9           & 225         & NA          & NA          \\ \hline
 	Week 6 August 10th-August 16th        & 94          & 11          & 154         & 2           & 12          \\ \hline
 	Week 7 August 17th-August 23rd        & 116         & 9           & 198         & 1           & 24          \\ \hline
 	Week 8 August 24th-August 30th        & 88          & 13          & 256         & 1           & 17          \\ \hline
 	Week 9 August 31st-September 6th      & 109         & 7           & 512         & 4           & 9           \\ \hline
 	Week 10 September 7th-September 13th  & 89          & 14          & 239         & 0           & 14          \\ \hline
 	Week 11 September 14th-September 20th & 104         & 10          & 212         & 2           & 18          \\ \hline
 	\end{tabular}
 	\end{table}
 	
 	
 	This section presents analysis of various protocols emulated by HosTaGe ICS. It consists of results for 11 weeks from 7th July to 20th September 2015. We need to note that HosTaGe is placed outside the firewalls, by making it face the Internet. The HosTaGe and Conpot had similar \ac{IP} address which means they were involved in the same/24 sub-network. The analysed result of protocols is shown in the Table~\ref{tb:Attack results per Protocol in HosTaGe ICS}.
 
 
 	\vspace{3mm} 	 	
 	The table~\ref{tb:Attack results per Protocol in HosTaGe ICS} consists of protocols emulated by HosTaGe ICS which includes HTTP, Modbus, Telnet, S7 and SMTP protocols. The Telnet protocol results have been gathered as it is considered to be an important attack factor for \ac{ICS} networks. Note that the S7 and SMTP protocol is considered for later weeks as its implementation was completed in a later stage. 
 	
 	
 	\vspace{3mm}
 	We also identified some genuine attack results for the Modbus protocol during the evaluation period. These were identified to be genuine as the attackers came back and tried to probe the target multiple times. The Table~\ref{tb:ICS Protocol Attacks Overview} shows a list of attacker \ac{IP}s with the location and the type of attacks carried out. This information forms a basis of detection evaluation of HosTaGe ICS. We provide only a overview of the attacks in the Table~\ref{tb:Attack results per Protocol in HosTaGe ICS} with respect \ac{ICS} profiles. There were a lot of interesting results that were observed during the evaluation period. One such interesting result was the detection of an attack pattern which led to the identification of multistage attacks which we discuss in further section. 
 	
 	\begin{table}[H]
 	\centering
	 \caption{ICS Protocol Attacks Overview}
 	\label{tb:ICS Protocol Attacks Overview}
	 \begin{tabular}{|l|l|l|l|}
	 \hline
 	\textbf{Attacker IP} & \textbf{Number of Attacks} & \textbf{Type of attack} & \textbf{Location} \\ \hline
	 5.200.120.251 & 3 & Modbus service check & Iran. \\ \hline
 	71.6.165.200 & 2 & Modbus port scan, with read coil & San Diego, California, United States. \\ \hline
	 198.20.69.98 & 6 & Modbus service check & Chicago, Illinois, United states. \\ \hline
	 188.138.1.218 & 2 & Modbus read coil & Germany. \\ \hline
	 101.69.178.234 & 5 & Modbus service check & Hangzhou, Zhejiang Sheng, China. \\ \hline
	 37.153.181.219 & 4 & Modbus port scan & Iran. \\ \hline
 	169.54.233.116 & 2 & Modbus service check & United States. \\ \hline
 52.10.40.42 & 4 & S7 Service detection & Boardman, Oregon, United States \\ \hline
 80.13.27.236 & 3 & Modbus service check & France \\ \hline
 \end{tabular}
 \end{table}
 	
 
 	\vspace{5mm} 
	\subsubsection{Conpot and HosTaGe ICS attack comparison}\label{Conpot and HosTaGe attack comparison}
	In the section~\ref{SCADA honeypots} Conpot  is described briefly. Conpot is an \ac{ICS} honeypot which simulates the behavior of a Siemens Simatic S7 200 system. It has the S7, Modbus , HTTP and \ac{SNMP} protocols implemented to support the simulation of the \ac{PLC}. 
	
	\vspace{3mm}
	The purpose of this experiment is to compare the two honeypots (HosTaGe and Conpot) and to detect the automated attacks that targets the \ac{ICS} networks and also the reason for not advertising the honeypots. However, it is shown that both honeypots are probed by well known search engine Shodan. c.f~\ref{Shodan Evasion}  
	
	\vspace{3mm}
	Conpot was deployed on one of our servers and was exposed to the Internet for attacks. All the connection packets received for HTTP, SNMP, S7 and Modbus are logged. However, the complete packet with the payload  information is not logged by Conpot. The logging mechanism is different for each protocol. 
	For evaluation, HosTaGe ICS was also deployed and exposed to the Internet for attacks~\cite{TUD-CS-2015183}.  Both honeypots were kept running for a specific evaluation period to analyze and compare the results at the end of the evaluation period.
	
	\vspace{3mm}
	The honeypots were deployed in controlled environments with no firewalls between them and the Internet in the same /24 subnet. The results of the analysis are shown in Figure~\ref{f:conpot and hostage comparison}. The results gathered from the two honeypots for the HTTP, Modbus and S7 protocols are compared. The Telnet protocol is a part of HosTaGe ICS simulation.  In addition to the mentioned protocols, HosTaGe ICS also received a lot of Telnet attacks. Telnet is considered important as it provides a shell access on the \ac{PLC} devices, through which command and control is possible directly. We also show the Telnet attacks in Figure~\ref{f:conpot and hostage comparison}. 
		
		
		\begin{figure}[ht]
				           \centering
				           \includegraphics[scale=1.4]{conpot-hostage.eps}
				           \caption[Conpot and Hostage comparison]{\label{f:conpot and hostage comparison}HosTaGe and Conpot Comparison}
				           \end{figure} 
		
		
		
		
		\vspace{5mm} 
		Inferring some results from the Figure~\ref{f:conpot and hostage comparison} we can note that HosTaGe exhibits good and on-par detection accuracy when compared to Conpot. In the case of HTTP, HosTaGe ICS is observed to have more attacks than Conpot and remains on par with Conpot on Modbus protocol. It is also observed that the results obtained through the Modbus protocol are a part of wide scans which are operated through research communities. For example~\cite{durumeric2013zmap}.
		It is important to mention that neither of the honeypots were advertised in any form on the Internet. However, Shodan the online vulnerable device database and search engine managed to find and probe both honeypots. This is further discussed in section~\ref{Shodan Evasion}.
		
		\vspace{3mm}
		The main goal of this evaluation was to compare the performance of HosTaGe ICS with Conpot in terms of attack detection and gathering. It is notable to observe that Conpot runs on a full PC environment while HosTaGe ICS performs equally better on a mobile environment.
		
		\vspace{3mm}
		This inference forms a basis for proving that HosTaGe ICS is a powerful honeypot which runs efficiently on limited resources and is capable of simulating the services of real devices on a mobile device platform. 	
	
		
	\subsubsection{Stuxnet Propagation}\label{Stuxnet Propagation}
	
	Stuxnet uses specific attack vectors to propagate to its target system. It remains dormant if it does not find the required attack vectors for distributing itself to the network. The most feasible attack vector that can be simulated in our evaluation environment is to simulate the network share service. 


	\vspace{3mm}
	Figure~\ref{f:Stuxnet Propagation Environment} shows the environment setup for the evaluation of Stuxnet propagation and detection.We create a suitable environment for the propagation of Stuxnet through HosTaGe ICS and a vulnerable Windows XP box. The experimental setup for the evaluation is performed as below:
	
	\begin{figure}
	\centering
	\includegraphics[scale=0.5]{stuxenv}
	\caption[Stuxnet Propagation Environment]{\label{f:Stuxnet Propagation Environment}Stuxnet Propagation Environment}
	\end{figure}
	
	
	\begin{enumerate}
	
	
	\item Stuxnet requires a vulnerable host with zero days to initially get active. A host system with vulnerable Windows XP operating system that is vulnerable to Zero Days is setup in the evaluation network.
	
	\item HosTaGe ICS is deployed in the subnet with Modbus Master profile active. This profile enables the \ac{SMB} protocol emulated that simulates a shared network drive in the network. This shared drive has a structure that resembles the SMB/CIFS file share of the earlier Windows systems. 
	
	\item We deploy a host system with a packet capture tool like Wireshark as a tap mechanism. Wireshark monitors all the information that is communicated between the vulnerable host and the target system.
	
	\item The shared network drive simulated by HosTaGe ICS is mapped in the Windows XP machine.
	
	\item The Windows XP host is injected with Stuxnet using a USB Flash drive. Stuxnet leverages the LNK\footnote{https://technet.microsoft.com/en-us/library/security/ms10-046.aspx} exploit of the Windows Zero Days to inject itself into the System files. 
	
	\item Once established, Stuxnet updates itself from a Command and Control server. Stuxnet checks if the current host communicates with a \ac{PLC}. If yes, it infects the \ac{PLC} through the Siemens Step7 software through a software attack vector. If not, it checks for other attack vectors to propagate itself to other hosts in the network.
	
	\item Stuxnet detects that the host is mapped to a network shared drive and uses this attack vector to propagate itself to the drive. It propagates using the dropper file, dropper.exe into the shared drive. This propagation is captured using the Wireshark tool. The Figure~\ref{f:Wireshark capture of Stuxnet Propagation} shows the propagation from the host system to the network shared drive which is simulated by HosTaGe ICS.
	
	\item The file propagated is detected by HosTaGe ICS. A hash of this file and its contents is computed and validated for malware from Virustotal malware database. Virustotal returns the name of the malware identified by various antivirus providers to HosTaGe ICS thereby validating the injected file as Stuxnet.
	
	\end{enumerate}
	\begin{figure}[H]
		\centering
		\includegraphics[scale=0.8]{stuxy}
		\caption[Wireshark capture of Stuxnet Propagation ]{\label{f:Wireshark capture of Stuxnet Propagation}Wireshark capture of Stuxnet Propagation}
		\end{figure}
	
	
	\subsubsection{Multistage attack detection and Inference}\label{Multistage attack detection and Inference}
	
	There were three significant multistage attacks. Among them the \ac{IP} '5.200.120.251' was interesting as the location of the \ac{IP} is approximated to be a Prison Facility in Tehran, Iran(which is also approx 10 kms away from the Tehran Nuclear Research Center). The attacker performed a portscan and sent a HTTP GET request to the \ac{IP}. The next step was an nmap script to check if a Modbus service is running on the host. Listing~\ref{lst:Nmap Modbus Discovery Script} shows the script used for this attack. The second interesting attack came from \ac{IP} '183.131.76.132' which is also approx 15kms away from Zhejiang Nuclear Industry in China. The attacker approached the HTTP protocol first and then targeted the Modbus protocol by sending a service query to the host. The third interesting attack was from \ac{IP} '181.143.236.179' which is 12 minutes away from EPM Power Plant in Colombia. The connection was first established to HTTP followed by a portscan and then by a Telnet session. 
	
	\begin{table}[H]
	\centering
	\caption{Multistage Attacks Overview}
	\label{Multistage Attacks Overview}
	\begin{tabular}{|l|l|l|}
	\hline
	\textbf{Attacker IP} & \textbf{Protocol/Attacks} & \textbf{Latitude, Longitude} \\ \hline
	183.131.76.132       & Portscan, HTTP, Modbus    & 29.1068, 119.6442            \\ \hline
	181.143.236.179      & HTTP, Portscan, TELNET    & 6.2518, -75.5636             \\ \hline
	5.200.120.251        & Portscan, HTTP, Modbus    & 35.6961, 51.4231             \\ \hline
	\end{tabular}
	\end{table}
	
	
	\subsection{Bro Signature Generation and Evaluation}\label{Bro Signature Generation and Evaluation}
	
		Generating multistage attack signatures is one of the mainstream features of HosTaGe ICS. We follow a series of steps to evaluate the signatures generated.
		
		
	 	
	 	\begin{enumerate}
	 	
	 	\item We pose as an adversary and attack HosTaGe ICS with multistage attacks as a goal. The attacks are captured on a packet capture tool like Wireshark and saved as a 'pcap' file for further reference.
	 	
	 	\item HosTaGe ICS detects the attacks from the previous step and generates signatures from both the protocol and payload-level interaction.
	 	
	 	
	 	\item To determine the applicability of the generated signatures, we perform two different tests. All tests utilize publicly available datasets of network traffic (in the form of pcap files). They consist of synthetic and real network captures\footnote{Small and Big flows datasets:http://tcpreplay.appneta.com/wiki/captures.html}  , malware focused traffic \footnote{CTU-13 Dataset: https://stratosphereips.org/category/dataset.html} , and honeypot captured traffic \footnote{HoneyBot Dataset: http://www.netresec.com/?page=PcapFiles}.
	 	
	 	\item We then determine whether the generated signatures detect false positives. We import the signatures into the Bro IDS and replay the network traffic of the test datasets. We utilize a time-window of tw = 15 (minutes) for our tests. No multi-stage attacks were detected by Bro, as expected.
	 	
	 	\item As a next step, we merge each dataset with the network traffic captured by Wireshark in the initial step (that includes our injected multi-stage attacks). Subsequently, we replay each modified network file while Bro is running. In all cases, Bro successfully
	 	detects all of the injected attacks without generating any false positives.
	\end{enumerate}
	
	
	
	\vspace{5mm} 
	\subsection{Shodan Evasion}\label{Shodan Evasion}
		One of the most important and essential features of a honeypot is to remain undetected as a decoy mechanism. This is a very hard feature to achieve and to be evaluated. During our experimental evaluation setup of Conpot and HosTaGe ICS, we received a lot of probes from Shodan.
		Shodan is one of the biggest search engines to find vulnerable devices on the Internet. It crawls the entire Internet to check for vulnerable devices. On the Shodan website, users can search for \ac{IP} addresses, specific devices, protocols and also vulnerabililtes like heartbleed~\cite{durumeric2014matter} or default password.
		Recently, Shodan started a new service called \textit{Honeypot Or Not?} which can identify honeypots. This service performs a series of probes or checks and subsequently creates a score, which is called \textit{Honeyscore} for each probed device. Shodan determines if the host is a honeypot based on this score.
		
		\vspace{3mm}
		During the evaluation period both honeypots (Conpot and HosTaGe ICS) received  probes from Shodan. The probes targetted HTTP, Modbus and S7 protocols. There were as many as 90 probes received over a period of 8 weeks. The probes were observed to be from 7 different subnets. 
		
		\vspace{5mm} 
		After 3 weeks, Shodan identified  Conpot to be a honeypot and listed this on its website at the honeypot or not site . Figure~\ref{f:Shodan Search result of Conpot Instance} shows the Shodan search result of our Conpot instance \ac{IP}. The open services and protocols were also listed. HosTage ICS was not detected as a honeypot by Shodan. The Shodan probes for HTTP and SSH were of low complexity, where the HTTP protocol involved a GET request and the SSH protocol was just the initial handshake message. The probes for Modbus and S7 protocols looked more sophisticated. An analysis made to the payload reveal that it could be a modified Nmap or Metasploit script to identify \ac{ICS}.
		
			
			\begin{figure}[H]
					\centering
					\includegraphics[scale=0.5]{shocon}
					\caption[Shodan Search result of Conpot Instance ]{\label{f:Shodan Search result of Conpot Instance}Shodan Search result of Conpot Instance \footnote{https://www.shodan.io/}}
					\end{figure}
		
		
		
		The S7 attack involved checks for the device type, location, serial number, plant identification and module name. The Modbus attack involved fetching details of certain units (i.e., unit number 0 and 255) and their slave data. Conpot could not respond as expected by Shodan in all the aforementioned requests (either due to static serial numbers or Modbus protocol simulation errors) and thus was classified as a honeypot. HosTaGe ICS managed to respond successfully and hence remain undetected. Figure~\ref{f:Conpot detected at Shodan Honeypot or Not page } shows the conpot instance detected as a honeypot in Shodan \textit{Honeypot or Not?} page.
		
			
			\begin{figure}[H]
					\centering
					\includegraphics[scale=0.5]{shohon}
					\caption[Conpot detected at Shodan Honeypot or Not page ]{\label{f:Conpot detected at Shodan Honeypot or Not page }Conpot detected at Shodan Honeypot or Not page \footnote{https://honeyscore.shodan.io/}}
					\end{figure}
		
		We tried to investigate Shodan's honeypot identification criteria, based on the probes received by Shodan. Finding the probes received from Shodan was a challenge. Shodan does not reveal it probes \ac{IP}. From a lot of study and mining on the Internet, we could identify some of the Shodan probes and tried to fetch data from our logs based on this information. Table~\ref{tb:Shodan Probes Overview} shows the various Shodan probes, hostname, location and the protocols they probe. Though we have the hostname, it is essential to know the packets that the probes use to get information to decide on whether the target system is a honeypot. In Listing~\ref{lst:Packets received from Shodan Probes} we show the packets received from the probes of Shodan. This information was logged by Conpot and Bro \ac{NIDS}. We aggregate the log information from both to get some interesting information. 
		
		
		
		\begin{table}[H]
		\centering
		\caption{Shodan Probes Overview}
		\label{tb:Shodan Probes Overview}
		\begin{tabular}{|l|l|l|l|l|}
		\hline
		\textbf{SlNo} & \textbf{Shodan Probe IP} & \textbf{Hostname} & \textbf{Protocol Targeted} & \textbf{Location} \\ \hline
		1 & 71.6.167.142 & census 9.shodan.io & S7 & \begin{tabular}[c]{@{}l@{}}San Diego.\\   California, United States\end{tabular} \\ \hline
		2 & 71.6.135.131 & census7.shodan.io & HTTP & \begin{tabular}[c]{@{}l@{}}San Diego.\\   California, United States\end{tabular} \\ \hline
		3 & 66.240.192.138 & census8.shodan.io & HTTP & \begin{tabular}[c]{@{}l@{}}San Diego.\\   California, United States\end{tabular} \\ \hline
		4 & 66.240.236.119 & census6.shodan.io & S7 & \begin{tabular}[c]{@{}l@{}}San Diego.\\   California, United States\end{tabular} \\ \hline
		5 & 93.120.27.62 & m247.ro.shodan.io & Modbus & Romania \\ \hline
		6 & 188.138.9.50 & atlantic.census.shodan.io & S7, Modbus & Germany \\ \hline
		7 & 85.25.103.50 & pacific.census.shodan.io & Modbus & Germany \\ \hline
		\end{tabular}
		\end{table}
	
			
		Further we analyse the impact of the above Shodan probes on our honeypot environments. To determine the impact of Shodan\cite{bodenheim2014impact} on our honeypots, we consider the attack data obtained before Shodan Probes and attacks after the Shodan probes. We observe a significant decline of attacks after Shodan declares Conpot instance as a honeypot. Shodan probed both the honeypots roughly around the same time period. Figure~\ref{f:Shodan Impact on Conpot Attacks} shows the impact of Shodan probes on Conpot instance before and after the probes attacked Conpot. A significant fall in the number of attacks is observed. On the contrary, Figure~\ref{f:Shodan Impact on HosTaGe ICS Attacks} shows the impact of Shodan probes on HosTaGe ICS. There seems to be negligible impact on HosTaGe ICS. However, to support our research that Shodan probes do make an impact on the honeypots, we need to prove that there is a realistic impact and that attackers look forward to Shodan for determining if the vulnerable systems on the Internet are actually honeypots. This research is important, considering honeypots to be stealthy detection systems. Such services pose a threat to the working aspect and productivity of a honeypot. 


	\begin{figure}[H]
	\centering
	\includegraphics[scale=0.9]{ShodanCon}
	\caption[Shodan Impact on Conpot Attacks ]{\label{f:Shodan Impact on Conpot Attacks}Shodan Impact on Conpot Attacks}
	\end{figure}
	\begin{figure}
	  \centering
	  \includegraphics[scale=0.9]{ShodanHost}
	  \caption[Shodan Impact on HosTaGe ICS Attacks ]{\label{f:Shodan Impact on HosTaGe ICS Attacks}Shodan Impact on Conpot Attacks}
	\end{figure}



	We can make use of Gaussian Distributions with the help of curve fitting algorithms to prove the impact is significant, provided we have more live data.  This is a continued research and will be focused on the future work of HosTaGe.  
	
	\vspace{5mm} 
	\subsection{Performance Evaluation}	\label{:Performance Evaluation}
	
		
	Honeypots are usually deployed on robust systems that can withstand strategized attacks. HosTaGe ICS is a mobile honeypot designed and implemented for the Android \ac{OS} platform. Android is highly flexible and open source offering a large number of developer \ac{API}s and libraries. 
	
		
	We intend to provide a robust environment for HosTaGe ICS through the Android platform leveraging its open source and developer friendly features. We try to achieve robustness through the protocol emulation mechanism. The mechanism has been carefully implemented for reply mechanisms of individual attacks. Further, by studying the behavior of target systems with respect to popular attacks towards individual protocols, a robust response mechanism has been implemented for handling the impact. HosTaGe ICS was deployed on a Samsung Galaxy S4 with a rooted configuration to allow emulation of protocols which have ports below 1024. The evaluation period was done for increasing interval periods of 5 minutes until 1 hour.


	HosTaGe ICS performance was evaluated using PowerTutor\cite{yang2012powertutor}, an android app to monitor the power consumed by major systems like the CPU, network interface, display, memory and group this power consumption by the appropriate applications. The primary goal of the application is to be able to track power consumption changes after modifying the application architecture and implementation details.



	HosTaGe ICS was monitored with various performance impact constraints such as  monitoring with the app minimized with running background, services stopped and with full functionality where the app runs in foreground and has all services enabled. To achieve the desired ICS specific functionality, we require the device to be rooted. This is necessary to achieve port binding to the protocols to be emulated. 
	
	
	\begin{figure}[H]
	\centering
	\includegraphics[scale=0.9]{output5}
	\caption[HosTaGe ICS Power Consumption Modes ]{\label{f:HosTaGe ICS Power Consumption Modes}HosTaGe ICS Power Consumption Modes}
	\end{figure}
	
	
	
	
	
	Figure~\ref{f:HosTaGe ICS Power Consumption Modes} shows the Power Consumption of HosTaGe ICS at three different modes.  We briefly discuss the energy efficiency of HosTaGe ICS during the three modes below.
	\begin{itemize}
	
	\item\textbf{HosTaGe ICS on screen with full functionality:} We evaluate the power consumed by HosTaGe ICS incrementally for a period of one hour. HosTaGe ICS was tested running foreground with multistage Detection service active. The Nuclear Power plant is set as active profile for monitoring. The default screen of the app consists of animation depicting the current state of network.
	
	\item\textbf{HosTaGe ICS running background:} The power consumption is evaluated with HosTaGe ICS running in background with multistage attack service active. The service is called every 15 minutes. This mode has fair utilization of power as there is no app GUI on foreground.
	
	\item\textbf{HosTaGe ICS running background with multistage service off: } In this mode, the power consumption evaluation with multistage attack service turned off. This mode provides best utilization of power in comparison with other modes.
	
	\end{itemize}
	
	
	\begin{figure}[H]
		\centering
		\includegraphics[scale=0.9]{output6}
		\caption[HosTaGe ICS Power Consumption in comparison with AVG]{\label{f:HosTaGe ICS Power Consumption in comparison with AVG}HosTaGe ICS Power Consumption in comparison with AVG}
		\end{figure}
	
	In addition to comparison with different modes, we compare HosTaGe ICS with AVG AV Free app to better evaluate the power consumption. While it was hard to find a suitable app to compare the power utilization of HosTaGe ICS, we chose to compare with AVG on the basis of functionality. As our app is deployed on a device, which would preferably not be used for usual applications, we restrict the comparison to an app which is similar to the resource consumption of HosTaGe ICS. Figure~\ref{f:HosTaGe ICS Power Consumption in comparison with AVG} shows the power consumption of HosTaGe ICS with comparison to AVG AV Free. HosTaGe ICS was running in background and minimized with the multistage attack detection service enabled. It is observed that HosTaGe ICS consumes fairly more power. It was also observed that the possible reason for increase in power utilization is due to the varied occurrence of attacks on HosTaGe ICS. The energy consumption depends on the protocol being targeted, the number of connections and packing the response to the request made. Therefore, the utilization changes based on the attacks received. 
	


	\subsection{Limitations}\label{Limitations}
	
	
	There were challenges faced during the implementation of some specific modules in HosTaGe ICS. Some design decisions such as rooting of the mobile device, had to be assumed to achieve the functionality of the app as a whole. The Siemens S7 protocol operates over the ISO TSAP transport layer. The regulation was imposed by DHS\footnote{http://www.dhs.gov/}claiming that TCP's lack of encryption capability. Due to some ambiguties in the design of ISO-TSAP on contrary to regular TCP, the S7 protocol was decided to be implemented as to operate on TCP in HosTaGe ICS. However, this does not deviate  the working mechanism of the S7 protocol. The protocol interaction works normally as on with TCP connections but sometimes failed if a connection is attempted from legacy software.
	
	
	\vspace{3mm}
	 The SNMP protocol is implemented as a basic service because of the absence of support on Android hardware and kernel. This is due to a restriction by the manufacturers enforced on the design. Emulation of the SNMP protocol is very limited and the functionality is not completely achieved. Siemens has its proprietary and customized FTP system. Emulation of FTP as per Siemens could not be achieved due to the lack of documentation. Multistage attacks consider portscans also as attack protocols for inferring the detection. However, signature policies for multistage attacks involving portscans could not be achieved due to development constraints on the Bro language. Nevertheless, a prototype is implemented as a part of an attempt. This implementation is not optimized and does not offer full functionality. We use the HTTP protocol for the Nuclear Power Plant profile. Once the user chooses this profile, the webserver is configured to host a HTML page automatically that looks like a Siemens device portal. The webpage designed is not according to the default portal. This is hard to implement due to limited support of Android towards creation of HTML content dynamically. The webpage generated as of now is indeed capable of attracting traffic, but a feel of the default site would make it even better.
	
	
	 \vspace{3mm}
	 
	  Signature generation for Bro IDS is an important feature of HosTaGe ICS. The signatures generated are stored in the device internal memory and have to be manually deployed to the Bro instance. An alternative would be to automate this process.
	
	
	
	
	\subsection{Summary}
	The evaluation section is briefly divided into experimental setup, detection evaluation, Bro signature generation evaluation, Shodan evasion, performance evaluation and limitations. The experimental setup has two parts and the first provides details about the setup made for the evaluation of HosTaGe ICS with Conpot. The second part provides information on the setup made to evaluate the signatures generated by HosTaGe ICS for Bro \ac{NIDS}.
	
	
	\vspace{3mm}
	The detection evaluation subsection is further divided into analysis of individual protocols, comparison of Conpot and HosTaGe ICS, Stuxnet propagation and lastly inferences and evaluation on multistage attacks.
	The analysis of individual protocols determine the detection capabilities of HostaGe ICS with respect to the Modbus, S7, HTTP and Telnet protocols implemented in our honeypot. Specific attacks on the Modbus protocol provide 	an insight into the attacker strategies for exploiting the system. The results observed are listed along with the attack data evaluation for Modbus protocol. HosTaGe ICS was evaluated for detection capabilities in comparison to Conpot. The two honeypots were deployed and made open to the Internet. The attacks received on the honeypots are compared and represented graphically. It is observed that HosTaGe ICS performed better on HTTP protocol and is nearly equal to Conpot on the detection of Modbus and S7 protocols. HosTaGe ICS also implemented the Telnet protocol and the attacks received on Telnet are also represented graphically. Detecting the Stuxnet malware on HosTaGe ICS is discussed further by evaluating it by setting up an environment feasible for it to propagate. The evaluation setup is discussed. The malware packets are captured during propagation from the infected host to the network share simulated by HosTaGe ICS. The file injected into the network share is further validated by HosTaGe by computing a hash and checking it with the VirusTotal database. We were successful in detecting the propagation of Stuxnet into our honeypot. During the evaluation period, we received attacks with a significant pattern. We implemented a module to specifically detect multistage attacks and received significant results. The results obtained are listed accordingly. 


	\vspace{3mm}
	HosTaGe ICS signature generation module generates signatures and policies for the Bro \ac{NIDS}. Evaluation of the signatures generated are discussed where the imposed attacks by us is merged along with large pcap file that are input to Bro for analysis. Bro detected the imposed attacks by us successfully in both live and offline detection modes. 
	
	\vspace{3mm}
	During the evaluation phase, we observed probes from Shodan search engine. We performed a deeper analysis into these probes and found about Shodans \textit{Honeypot or Not?} tool which detected if the end system is an honeypot. The Conpot instance was detected by this tool to be a honeypot but not HosTaGe ICS. HosTaGe ICS managed to send correct replies to these probes which made the tool infer that the host was a real system. Further, we investigated the impact of these probes on the attack detection capabilities of the honeypots. The results are represented through graphs which indicate the decline in the number of attacks for the Conpot instance but not for HosTaGe ICS. However, to prove the impact, we needed more data for representing through curve fitting techniques and distributions. This area will be focused in the future work.
	
	\vspace{3mm}
	Lastly, we discussed on performance evaluation which focused on power consumption of HostaGe ICS through three modes. The first mode involved running HosTaGe ICS on screen with the multistage attack detection service running in background. Second, we run HosTaGe ICS in the background along with multistage attack detection service and lastly we only run HosTaGe ICS in the background without multistage attack detection service.  We observed that HostaGe ICS in full on-screen consumes more power than the other two modes. We also compared the power consumption of HostaGe ICS with another running application like AVG antivirus and concluded that HostaGe ICS consumes quite more power when compared to the antivirus application and the reason would be because of varied occurence of attacks on HosTaGe ICS.
	
	The limitations sections describes the limitations faced during the implementation of HosTaGe ICS and the challenges which could be optimized in the future work. 

	\newpage
  	\section{Conclusion}\label{Conclusion}
	With almost every device being capable of accessible through the Internet, serious threats follow regarding the security of the devices and enterprise networks. \ac{ICS} form a backbone for basic services offered to humanity. These systems are also highly critical because of their functionality. Such devices have to be protected. The driving research questions behind this thesis were to efficiently detect \ac{ICS} specific attacks and detect complex malware like Stuxnet by extending HosTaGe capabilities. During the course of the thesis, some significant results observed led to resolving an emerging research question of identifying multistage attacks, a popular strategy followed by attackers to compromise the end systems. To address this question, related work on securing the \ac{ICS}
	has been discussed. \ac{ICS} environments today offer better managing capabilities to its users by implying newer hardware which are compatible with modern Internet infrastructure. This also poses as a serious threat as it opens up heaps of opportunities for attackers to gain illegal access to the infrastructure. 
	
	
	\vspace{3mm}
	Existing approaches include deploying \ac{NIDS} that rely on signatures to detect the attacks and honeypots that are capable of simulating the target environment. As \ac{IDS} rely on signatures, it is sometimes not possible to detect tailored attacks and related honeypots solutions do not provide enough flexibility to analyse these attacks. They provide lesser simulation capabilities. A flexible,user friendly, robust and interactive mechanism is necessary to tackle the discussed issues.


	\vspace{3mm}
	We introduced HosTaGe ICS a mobile based low interactive honeypot for ICS. HosTaGe ICS extends the functionality of HosTaGe, a low interaction honeypot capable of simulating various profiles and protocols. The core idea behind HosTaGe ICS is to be capable of detecting attacks with respect to \ac{ICS} environment. HosTaGe ICS successfully emulates \ac{ICS} specific protocols like Modbus and S7 thereby offering a complete simulation of \ac{ICS} specific devices.  The protocols bound together with other protocols like HTTP, Telnet, \ac{SMTP} and \ac{SMB} are successfully capable of simulating the master and slave profiles in an \ac{ICS} environment. The response mechanisms of individual protocols are designed considering previous exploits and malware in the area of \ac{ICS}. The services were replicated by studying the behavior of popularly deployed Siemens S7 \ac{PLC}s and studying their vulnerabilities.


	\vspace{3mm}
 	We formally modeled all the attack strategies to arrive at a generic attack model. This also led to the conclusion that formally modeling the attacks and the malware propagation techniques helps in reverse engineering and anatomy of a malware.  Inorder to detect complex malware like Stuxnet, we analyze the propagation of Stuxnet and formally modelling it for better understanding and implementation.


	\vspace{3mm}
	Further, HosTaGe ICS extends the detection capabilities of HosTaGe to detect file injection and multistage attacks. File injection detects any malware injected into the system  and verifies it with an online virus database. This feature enables HosTaGe ICS to detect complex malware like Stuxnet,  which are capable of devastating the \ac{ICS} infrastructure.  HosTaGe ICS also efficiently detects multistage attacks, a crucial strategy followed by adversaries for attacking a specific target. We extended HosTaGe ICS to be more productive by generating signatures for the attacks detected in specific protocols. These signatures can be directly deployed on Bro IDS to check incoming packets for malicious payload. Also, HostaGe ICS is capable of generating policies for multistage attacks detection. These policies help in reducing false positives and also provide front end protection for \ac{IPS} by dropping packets from blacklisted attackers.  The signatures and policies work effectively on both live and offline analysis using Bro.  Through these techniques, we were able to filter genuine attacks and reduce the number of false positives hereby improving the detection efficiency. 


	\vspace{3mm}
	The results obtained during the course of the thesis demonstrate that HosTaGe ICS is capable of effectively simulating \ac{ICS} specific profiles and protocols by attracting huge number of malicious traffic. Further, the implementation of an approach to detect Stuxnet propagation through File Injection proved that HosTaGe ICS is capable of detecting such complex malware. The results achieved also led to the identification of multistage attacks and devise a method to detect them in HosTaGe ICS. The attacks also led to the observation of Shodan probes that fetch information from target systems, to determine if an end system is real or a honeypot. Shodans \textit{Honeypot or Not?} service poses a huge threat for Enterprise that use honeypots as  a line of defence to monitor their networks. We describe Shodans strategy to determine honeypots by inspecting and analyzing their packets.
	
	
	\vspace{3mm}
	HosTaGe ICS like its predecessor is robust to attacks and offers good interaction capabilities. Ideal features like obfuscation, stealth, atmost simulation, low-complexity, reduced resource consumption, user-friendly, flexible make it a good player amongst other honeypots in the same area. We also argue that HosTaGe ICS being a mobile honeypot offers better detection capabilities than other related honeypot services available for \ac{ICS}. It also supports addition of more protocols and detection of attack strategies. HosTaGe ICS has been developed considering atmost productivity in live scenarios. Since it is deployed on a mobile device, the administration is fairly simple for all levels of users. Advanced features like service settings are also available for advanced users. \ac{ICS} are vital for human basic needs, said so there is a need to avoid inhibition and take measures to safeguard them. 

  	\vspace{3mm}
  	
  	\subsection{Future Work}\label{Future Work}
  	HosTaGe ICS opens up opportunities for future work and enhancing the system further. Some of the features were decided to be ignored based on some factors like time, development, library support and lack of information. What follows will be a brief overview of various enhancements for HosTaGe ICS.
  	There are a lot of developments in the field of \ac{ICS} and the technology is continuously evolving to provide better support and functionality. Newer protocols in the Profibus\cite{tovar1999real} module could be included to simulate \ac{ICS} environment better. Simulation of newer industrial devices such as \ac{PLC} with vulnerabilities could be added. HosTaGe ICS currently supports simulation of a single slave device. This could be enhanced to provide support for multiple slave devices connected together through the Modbus. This provides a more complex environment for the attackers hereby gaining more attention. 


  	\vspace{3mm}
  	Soon after the \ac{ICS} were first targeted by Stuxnet, newer malware were designed to create similar impact. Malware like Flame, Duqu , Duqu2 and Havex have had impacts similar to that of Stuxnet. Mechanisms to detect such malware could be added to HosTaGe ICS as an extended detection capability. 
  	The current signature generation module creates signatures for Bro IDS. This functionality can be extended to generate signatures for the Snort IDS\cite{roesch1999snort}. Snort is a widely deployed enterprise grade intrusion detection monitor. 
  	
  	
  	\vspace{3mm}
  	HostaGe ICS emulates different protocols and runs services in the background which utilize more power from the device. This could be optimized by providing different battery utilization modes for user to choose from. This could not only reduce the power consumption but also provide different specific working modes.
  
  
  	\vspace{3mm}
  	HosTaGe ICS could be made more flexible by adding a plugin where newer protocols developed could be integrated through the app itself. This adds a new advantage of quickly updating the existing protocols and also for adding newer protocols into the collection. 
  
  
  	\vspace{3mm}
  	We discussed about Shodan probes on our evaluation section. Shodans \textit{Honeypot or Not?} tool poses as huge threat for honeypot researchers and users as the main intention of deploying them is not achieved. The research could be extended further to assess the impact of Shodan probes and also handling them. Though HosTaGe ICS is currently not affected by this tool, ensuring that the probes do not impact its identity is important. 
  	
  	
  	 \vspace{3mm}
  	HosTaGe ICS currently supports detecting portscans, file injections and multistage attacks as a part of attack detection strategies. More strategies like Denial Of Service can be implemented to offer better detection capabilities.
  	
  	
  	\newpage
  	\begin{appendix}
  	  \listoffigures
  	  \newpage
  	   	\listoftables
  	    	  \newpage
  	  \lstlistoflistings
  	  
  	  \newpage
  	  
  	  	\section{List of Acronyms}
  	    \begin {acronym} [include-classes=abbrev,name=Abbreviations]
  	      \acro {ICS} {Industrial Control Systems}
  	      \acro {SCADA} {Supervisory Control and Data Acquisition}
  	      \acro {DCS} {Distributed Control Systems}
  	      \acro {PLC} {Programmable Logic Controllers}
  	      \acro {RTU} {Real Time Unit}
  	      \acro {WAN} {Wide Area Network}
  	      \acro {EFSM} {Extended Finite State Machine}
  	      \acro {PLD} {Payload Level Detection}
  	      \acro {SPLD} {Single Payload Level Detection}
  	      \acro {MSLD} {Multistage Level Detection}
  	      \acro {HMI} {Human Machine Interface}
  	      \acro {LAN} {Local Area Network}
  	      \acro {IETF} {Internet Engineering Task Force}
  	      \acro {RFC} {Request For Comments}
  	      \acro {TCP} {Transmission Control Protocol}
  	      \acro {ISN} {Initial Sequence Number}
  	      \acro {IP} {Internet Protocol}
  	      \acro {TLS} {Transport Layer Security}
  	      \acro {SSH} {Secure Shell}
  	      \acro {HTTPS} {Secure Hypertext Transfer Protocol}
  	      \acro {IPSec} {Internet Protocol layer security}
  	      \acro {IIS} {Internet Information Server}
  	      \acro {SMS} {Short Message Service}
  	      \acro {FTP} {File Transfer Protocol}
  	      \acro {DDOS} {Distributed Denial of Service}
  	      \acro {IDS} {Intrusion Detection System}
  	      \acro {IPS} {Intrusion Prevention System}
  	      \acro {API} {Application Programming Interface}
  	      \acro {CIFS} {Common Internet File System}
  	      \acro {WinCC} {Windows Control Center}
  	      \acro {DLL}	{Dynamic Link Library}
  	      \acro {SNMP} {Simple Network Management Protocol}
  	      \acro {SMTP} {Simple Mail Transfer Protocol}
  	      \acro {JSON} {Java Synchronous Object Notation}
  	      \acro {OS} {Operating System}
  	      \acro {FSM} {Finite State Machine}
  	      \acro {RPC} {Remote Procedure Call}
  	      \acro {CIDS} {Collaborative Intrusion Detection System}
  	      \acro {NIDS} {Network Intrusion Detection System}
  	      \acro {TCP} {Transmission Control Protocol}
  	      \acro {GUI} {Graphical User Interface}
  	      \acro {MTU} {Maximum Transmission Unit}
  	      \acro {COTP} {Connection Oriented Transport Protocol}
  	      \acro {PDU} {Protocol Data Unit}
  	      \acro {VoIP} {Voice over Internet Protocol} 
  	      \acro {IPv6} {Internet Protocol Version 6}
  	      \acro {SMB} {Server Message Block}
  	      \acro {SQL} {Structured Query Language}
  	      \acro {S7} {Siemens S7 protocol}
  	      \acro {HTTP} {Hyper Text Transfer Protocol}
  	      \acro {RDBMS} {Relational Database Management Systems}
  	      \end{acronym}
  	      
  	      \newpage
  	      \section{Listings}
  	      
  	      
  	      		\begin{lstlisting}[caption=Packets received from Shodan Probes,label=lst:Packets received from Shodan Probes]
  	      			//93.120.27.62 Modbus Packets
  	      			2015-07-27 15:55:40,505 New modbus session from 93.120.27.62 (1651567b-c1df-42da-aa4d-4596ff158256)
  	      			2015-07-27 15:55:40,510 New connection from 93.120.27.62:54601. (1651567b-c1df-42da-aa4d-4596ff158256)
  	      			2015-07-27 15:55:40,521 Modbus traffic from 93.120.27.62: {'function_code': None, 'slave_id': 0, 'request': '0000000000020011', 'response': '9101'} (1651567b-c1df-42da-aa4d-4596ff158256)
  	      			
  	      			
  	      			//66.240.236.119 S7 Packets
  	      			2015-08-27 22:08:27,989 New s7comm session from 66.240.236.119 (91a441e7-5059-4fb6-8bf4-7ebc5bae4086)
  	      			2015-08-27 22:08:27,994 New connection from 66.240.236.119:52994. (91a441e7-5059-4fb6-8bf4-7ebc5bae4086)
  	      			
  	      			
  	      			//71.6.135.131 HTTP Packets
  	      			2015-08-16 17:43:26,221 New http session from 71.6.135.131 (899d404f-c8c3-4a7f-abfa-b7b4c1bb1f4d)
  	      			2015-08-16 17:43:26,225 HTTP/1.1 GET request from ('71.6.135.131', 60616): ('/', ['Host: 130.83.208.166\r\n', 'Accept-Encoding: identity\r\n'], None). 899d404f-c8c3-4a7f-abfa-b7b4c1bb1f4d
  	      			2015-08-16 17:43:26,228 HTTP/1.1 response to ('71.6.135.131', 60616): 302. 899d404f-c8c3-4a7f-abfa-b7b4c1bb1f4d
  	      			2015-08-16 17:43:26,761 HTTP/1.1 GET request from ('71.6.135.131', 60892): ('/robots.txt', ['Host: 130.83.208.166\r\n', 'Accept-Encoding: identity\r\n'], None). 899d404f-c8c3-4a7f-abfa-b7b4c1bb1f4d
  	      			
  	      			
  	      			//66.240.192.138 HTTP Packets
  	      			2015-09-27 16:36:04,341 New http session from 66.240.192.138 (1651ea2c-fa69-4924-a44b-a4cb10d4e588)
  	      			2015-09-27 16:36:04,345 HTTP/1.1 GET request from ('66.240.192.138', 36902): ('/', ['Host: 130.83.208.166\r\n', 'Accept-Encoding: identity\r\n'], None). 1651ea2c-fa69-4924-a44b-a4cb10d4e588
  	      			2015-09-27 16:36:04,353 HTTP/1.1 response to ('66.240.192.138', 36902): 302. 1651ea2c-fa69-4924-a44b-a4cb10d4e588
  	      			2015-09-27 16:36:04,913 HTTP/1.1 GET request from ('66.240.192.138', 37419): ('/robots.txt', ['Host: 130.83.208.166\r\n', 'Accept-Encoding: identity\r\n'], None). 1651ea2c-fa69-4924-a44b-a4cb10d4e588
  	      			
  	      			//71.6.167.142 S7 Packets
  	      			2015-08-12 05:43:35,256 New s7comm session from 71.6.167.142 (3df36aea-2178-4f31-ae17-68d961bd99ff)
  	      			2015-08-12 05:43:35,261 New connection from 71.6.167.142:55412. (3df36aea-2178-4f31-ae17-68d961bd99ff)
  	      			2015-08-12 05:43:41,437 New connection from 71.6.167.142:56775. (3df36aea-2178-4f31-ae17-68d961bd99ff)
  	      			
  	      			
  	      			//188.138.9.50 S7 Packets
  	      			2015-07-19 03:18:56,463 New s7comm session from 188.138.9.50 (c2f351af-36a5-402b-93cd-28413f036836)
  	      			2015-07-19 03:18:56,467 New connection from 188.138.9.50:53854. (c2f351af-36a5-402b-93cd-28413f036836)
  	      			2015-07-19 03:18:56,666 New connection from 188.138.9.50:53868. (c2f351af-36a5-402b-93cd-28413f036836)
  	      			
  	      			 
  	      			//85.25.103.50 Modbus Packets
  	      			2015-08-24 07:02:56,907 New modbus session from 85.25.103.50 (499b645e-98ce-409c-ba4f-3f42d4240056)
  	      			2015-08-24 07:02:56,912 New connection from 85.25.103.50:51086. (499b645e-98ce-409c-ba4f-3f42d4240056)
  	      			2015-08-24 07:02:56,923 Modbus traffic from 85.25.103.50: {'function_code': None, 'slave_id': 0, 'request': '0000000000020011', 'response': '9101'} (499b645e-98ce-409c-ba4f-3f42d4240056)
  	      			\end{lstlisting}
  	      			
  	      			\newpage
  	      			
  	      			\begin{lstlisting}[caption=Bro Policy for Multistage Attack detected by HosTaGE ICS,label=lst:Bro Policy for Multistage Attack detected by HosTaGE ICS ]
  	      				
  	      				
  	      				@load base/frameworks/notice
  	      				
  	      				export{
  	      					redef enum Notice::Type += {
  	      						Multistage
  	      					};
  	      				}
  	      				
  	      				//Declaring and Initializing variables
  	      				global attack_ip = 130.83.208.167;
  	      				global attack_port : vector of port = vector(80/tcp,22/tcp);
  	      				global attack_count = 0;
  	      				global attack_subnet = 130.83.208.0/24;
  	      				
  	      				
  	      				//On a connection request	
  	      				event connection_established(c: connection)
  	      				{
  	      				
  	      				print fmt ("Initiating.............");
  	      				
  	      				//Check for attack on different protocols as inferred from HosTaGe ICS
  	      				for (i in attack_port){
  	      				
  	      				 	//Check if it is the first connection from the attacker
  	      					if(attack_count==0){
  	      				
  	      					if ((c$id$orig_h in attack_subnet) && (c$id$resp_p==attack_port[0]))
  	      				        {
  	      					local net_time1: time  = network_time();
  	      					print fmt("The first attack ip is ");
  	      					print c$id$orig_h;
  	      					print fmt("and the first port is");
  	      					print c$id$resp_p;
  	      				        ++attack_count;
  	      					print attack_count;
  	      					next;
  	      					
  	      				       		}
  	      				
  	      					}
  	      				//On the next connection
  	      				   else {
  	      					
  	      				        if ((c$id$orig_h in attack_subnet) && (c$id$resp_p == attack_port[1])){
  	      					local net_time2: time  = network_time();
  	      					print("The second ip is");
  	      					print c$id$orig_h;
  	      					print fmt("The second port is");
  	      					print c$id$resp_p;
  	      					
  	      					//Check for time interval
  	      					if(net_time2-net_time1 <= 15 min){
  	      					print fmt ("MULTISTAGE ATTACK!!!");
  	      					}        
  	      					NOTICE([$note = Multistage,
  	      				                $conn = c,
  	      				                $msg = fmt("Multistage Attack! from %s",c$id$orig_h)]);
  	      					attack_count = 0;
  	      					
  	      				        }
  	      				
  	      				 }
  	      				
  	      				 }
  	      				}
  	      				
  	      				
  	      				
  	      				\end{lstlisting}
  	      				
  	      				
  	      	\newpage
  	      				
  	      	\begin{lstlisting}[caption= Nmap Modbus Discovery Script, label=lst:Nmap Modbus Discovery Script]
  	      	ocal bin = require "bin"
  	      	local comm = require "comm"
  	      	local nmap = require "nmap"
  	      	local shortport = require "shortport"
  	      	local stdnse = require "stdnse"
  	      	local string = require "string"
  	      	local table = require "table"
  	      	
  	      	description = [[
  	      	Enumerates SCADA Modbus slave ids (sids) and collects their device information.
  	      	
  	      	Modbus is one of the popular SCADA protocols. This script does Modbus device
  	      	information disclosure. It tries to find legal sids (slave ids) of Modbus
  	      	devices and to get additional information about the vendor and firmware. This
  	      	script is improvement of modscan python utility written by Mark Bristow.
  	      	
  	      	Information about MODBUS protocol and security issues:
  	      	* MODBUS application protocol specification:  http://www.modbus.org/docs/Modbus_Application_Protocol_V1_1b.pdf
  	      	* Defcon 16 Modscan presentation: https://www.defcon.org/images/defcon-16/dc16-presentations/defcon-16-bristow.pdf
  	      	* Modscan utility is hosted at google code: http://code.google.com/p/modscan/
  	      	]]
  	      	
  	      	---
  	      	-- @usage
  	      	-- nmap --script modbus-discover.nse --script-args='modbus-discover.aggressive=true' -p 502 <host>
  	      	--
  	      	-- @args aggressive - boolean value defines find all or just first sid
  	      	--
  	      	-- @output
  	      	-- PORT    STATE SERVICE
  	      	-- 502/tcp open  modbus
  	      	-- | modbus-discover:
  	      	-- |   sid 0x64:
  	      	-- |     Slave ID data: \xFA\xFFPM710PowerMeter
  	      	-- |     Device identification: Schneider Electric PM710 v03.110
  	      	-- |   sid 0x96:
  	      	-- |_    error: GATEWAY TARGET DEVICE FAILED TO RESPONSE
  	      	--
  	      	-- @xmloutput
  	      	-- <table key="sid 0x64">
  	      	--   <elem key="Slave ID data">\xFA\xFFPM710PowerMeter</elem>
  	      	--   <elem key="Device identification">Schneider Electric PM710 v03.110</elem>
  	      	-- </table>
  	      	-- <table key="sid 0x96">
  	      	--   <elem key="error">GATEWAY TARGET DEVICE FAILED TO RESPONSE</elem>
  	      	-- </table>
  	      	
  	      	-- Version 0.2 - /12.12.10/ - script cleanup
  	      	-- Version 0.3 - /13.12.10/ - several bugfixes
  	      	
  	      	author = "Alexander Rudakov"
  	      	license = "Same as Nmap--See https://nmap.org/book/man-legal.html"
  	      	categories = {"discovery", "intrusive"}
  	      	
  	      	
  	      	portrule = shortport.port_or_service(502, "modbus")
  	      	
  	      	local form_rsid = function(sid, functionId, data)
  	      	  local payload_len = 2
  	      	  if  ( #data > 0 ) then
  	      	    payload_len = payload_len + #data
  	      	  end
  	      	  return "\0\0\0\0\0" .. bin.pack('CCC', payload_len, sid, functionId) .. data
  	      	end
  	      	
  	      	discover_device_id_recursive = function(host, port, sid, start_id, objects_table)
  	      	  local rsid = form_rsid(sid, 0x2B, "\x0E\x01" .. bin.pack('C', start_id))
  	      	  local status, result = comm.exchange(host, port, rsid)
  	      	  if ( status and (#result >= 8)) then
  	      	    local ret_code = string.byte(result, 8)
  	      	    if ( ret_code == 0x2B and #result >= 15 ) then
  	      	      local more_follows = string.byte(result, 12)
  	      	      local next_object_id = string.byte(result, 13)
  	      	      local number_of_objects = string.byte(result, 14)
  	      	      stdnse.debug1("more = 0x%x, next_id = 0x%x, obj_number = 0x%x", more_follows, next_object_id, number_of_objects)
  	      	      local offset = 15
  	      	      for i = start_id, (number_of_objects - 1) do
  	      	        local object_id = string.byte(result, offset)
  	      	        local object_len = string.byte(result, offset + 1)
  	      	        -- error data format --
  	      	        if object_len == nil then break end
  	      	        local object_value = string.sub(result, offset + 2, offset + 1 + object_len)
  	      	        stdnse.debug1("Object id = 0x%x, value = %s", object_id, object_value)
  	      	        table.insert(objects_table, object_id + 1, object_value)
  	      	        offset = offset + 2 + object_len
  	      	      end
  	      	      if ( more_follows == 0xFF and next_object_id ~= 0x00 ) then
  	      	        stdnse.debug1("Has more objects")
  	      	        return discover_device_id_recursive(host, port, sid, next_object_id, objects_table)
  	      	      end
  	      	    end
  	      	  end
  	      	  return objects_table
  	      	end
  	      	
  	      	local discover_device_id = function(host, port, sid)
  	      	  return discover_device_id_recursive(host, port, sid, 0x0, {})
  	      	end
  	      	
  	      	local extract_slave_id = function(response)
  	      	  local byte_count = string.byte(response, 9)
  	      	  if ( byte_count == nil or byte_count == 0) then return nil end
  	      	  local offset, slave_id = bin.unpack("A"..byte_count, response, 10)
  	      	  return slave_id
  	      	end
  	      	
  	      	modbus_exception_codes = {
  	      	  [1]  = "ILLEGAL FUNCTION",
  	      	  [2]  = "ILLEGAL DATA ADDRESS",
  	      	  [3]  = "ILLEGAL DATA VALUE",
  	      	  [4]  = "SLAVE DEVICE FAILURE",
  	      	  [5]  = "ACKNOWLEDGE",
  	      	  [6]  = "SLAVE DEVICE BUSY",
  	      	  [8]  = "MEMORY PARITY ERROR",
  	      	  [10] = "GATEWAY PATH UNAVAILABLE",
  	      	  [11] = "GATEWAY TARGET DEVICE FAILED TO RESPOND"
  	      	}
  	      	
  	      	action = function(host, port)
  	      	  -- If false, stop after first sid.
  	      	  local aggressive = stdnse.get_script_args('modbus-discover.aggressive')
  	      	
  	      	  local opts = {request_timeout=2000}
  	      	  local results = stdnse.output_table()
  	      	
  	      	  for sid = 1, 246 do
  	      	    stdnse.debug3("Sending command with sid = %d", sid)
  	      	    local rsid = form_rsid(sid, 0x11, "")
  	      	
  	      	    local status, result = comm.exchange(host, port, rsid, opts)
  	      	    if ( status and (#result >= 8) ) then
  	      	      local ret_code = string.byte(result, 8)
  	      	      if ( ret_code == (0x11) or ret_code == (0x11 + 128) ) then
  	      	        local sid_table = stdnse.output_table()
  	      	        if ret_code == (0x11) then
  	      	          local slave_id = extract_slave_id(result)
  	      	          sid_table["Slave ID data"] = slave_id or "<unknown>"
  	      	        elseif ret_code == (0x11 + 128) then
  	      	          local exception_code = string.byte(result, 9)
  	      	          local exception_string = modbus_exception_codes[exception_code]
  	      	          if ( exception_string == nil ) then
  	      	            exception_string = ("Unknown exception (0x%x)"):format(exception_code)
  	      	          end
  	      	          sid_table["error"] = exception_string
  	      	        end
  	      	
  	      	        local device_table = discover_device_id(host, port, sid)
  	      	        if ( #device_table > 0 ) then
  	      	          sid_table["Device identification"] = table.concat(device_table, " ")
  	      	        end
  	      	        if ( #sid_table > 0 ) then
  	      	          results[("sid 0x%x"):format(sid)] = sid_table
  	      	        end
  	      	        if ( not aggressive ) then break end
  	      	      end
  	      	    end
  	      	  end
  	      	
  	      	  if ( #results > 0 ) then
  	      	    port.state = "open"
  	      	    port.version.name = "modbus"
  	      	    nmap.set_port_version(host, port)
  	      	    return results
  	      	  end
  	      	end
  	      	
  	      	
  	      	
  	      	\end{lstlisting}
  	      	
  	      	\newpage
  	      	
  	      	\begin{lstlisting}[caption=Metasploit Modbus Detect Script,label=lst:Metasploit Modbus Detect Script]
  	      	
  	      	##
  	      	# This module requires Metasploit: http://metasploit.com/download
  	      	# Current source: https://github.com/rapid7/metasploit-framework
  	      	##
  	      	
  	      	require 'msf/core'
  	      	
  	      	class Metasploit3 < Msf::Auxiliary
  	      	
  	      	  include Msf::Exploit::Remote::Tcp
  	      	  include Msf::Auxiliary::Scanner
  	      	
  	      	  def initialize
  	      	    super(
  	      	      'Name'        => 'Modbus Version Scanner',
  	      	      'Description' => %q{
  	      	          This module detects the Modbus service, tested on a SAIA PCD1.M2 system.
  	      	        Modbus is a clear text protocol used in common SCADA systems, developed
  	      	        originally as a serial-line (RS232) async protocol, and later transformed to IP,
  	      	        which is called ModbusTCP.
  	      	      },
  	      	      'References'  =>
  	      	        [
  	      	          [ 'URL', 'http://www.saia-pcd.com/en/products/plc/pcd-overview/Pages/pcd1-m2.aspx' ],
  	      	          [ 'URL', 'http://en.wikipedia.org/wiki/Modbus:TCP' ]
  	      	        ],
  	      	      'Author'      => [ 'EsMnemon <esm[at]mnemonic.no>' ],
  	      	      'DisclosureDate' => 'Nov 1 2011',
  	      	      'License'     => MSF_LICENSE
  	      	      )
  	      	
  	      	    register_options(
  	      	      [
  	      	        Opt::RPORT(502),
  	      	        OptInt.new('UNIT_ID', [true, "ModBus Unit Identifier, 1..255, most often 1 ", 1]),
  	      	        OptInt.new('TIMEOUT', [true, 'Timeout for the network probe', 10])
  	      	      ], self.class)
  	      	  end
  	      	
  	      	  def run_host(ip)
  	      	    # read input register=func:04, register 1
  	      	    sploit="\x21\x00\x00\x00\x00\x06\x01\x04\x00\x01\x00\x00"
  	      	    sploit[6] = [datastore['UNIT_ID']].pack("C")
  	      	    connect()
  	      	    sock.put(sploit)
  	      	    data = sock.get_once
  	      	
  	      	    # Theory: When sending a modbus request of some sort, the endpoint will return
  	      	    # with at least the same transaction-id, and protocol-id
  	      	    if data
  	      	      if data[0,4] == "\x21\x00\x00\x00"
  	      	        print_good("#{ip}:#{rport} - MODBUS - received correct MODBUS/TCP header (unit-ID: #{datastore['UNIT_ID']})")
  	      	      else
  	      	        print_error("#{ip}:#{rport} - MODBUS - received incorrect data #{data[0,4].inspect} (not modbus/tcp?)")
  	      	      end
  	      	    else
  	      	      vprint_status("#{ip}:#{rport} - MODBUS - did not receive data.")
  	      	    end
  	      	
  	      	    disconnect()
  	      	  end
  	      	end
  	      	
  	      	\end{lstlisting}
  	      	
  	     \newpage
  	     
  	     \begin{lstlisting}[caption=Metasploit script of Writing into Modbus Registers, label=lst:Modbusclinet]
  	     
  	     ##
  	     # This module requires Metasploit: http://metasploit.com/download
  	     # Current source: https://github.com/rapid7/metasploit-framework
  	     ##
  	     
  	     require 'msf/core'
  	     
  	     class Metasploit3 < Msf::Auxiliary
  	     
  	       include Msf::Exploit::Remote::Tcp
  	     
  	       def initialize(info = {})
  	         super(update_info(info,
  	           'Name'          => 'Modbus Client Utility',
  	           'Description'   => %q{
  	             This module allows reading and writing data to a PLC using the Modbus protocol.
  	             This module is based on the 'modiconstop.rb' Basecamp module from DigitalBond,
  	             as well as the mbtget perl script.
  	           },
  	           'Author'         =>
  	             [
  	               'EsMnemon <esm[at]mnemonic.no>', # original write-only module
  	               'Arnaud SOULLIE  <arnaud.soullie[at]solucom.fr>' # new code that allows read/write
  	             ],
  	           'License'        => MSF_LICENSE,
  	           'Actions'        =>
  	             [
  	               ['READ_COIL', { 'Description' => 'Read one bit from a coil' } ],
  	               ['WRITE_COIL', { 'Description' => 'Write one bit to a coil' } ],
  	               ['READ_REGISTER', { 'Description' => 'Read one word from a register' } ],
  	               ['WRITE_REGISTER', { 'Description' => 'Write one word to a register' } ]
  	             ],
  	           'DefaultAction' => 'READ_REGISTER'
  	           ))
  	     
  	         register_options(
  	           [
  	             Opt::RPORT(502),
  	             OptInt.new('DATA', [false, "Data to write (WRITE_COIL and WRITE_REGISTER modes only)"]),
  	             OptInt.new('DATA_ADDRESS', [true, "Modbus data address"]),
  	             OptInt.new('UNIT_NUMBER', [false, "Modbus unit number", 1]),
  	           ], self.class)
  	     
  	       end
  	     
  	       # a wrapper just to be sure we increment the counter
  	       def send_frame(payload)
  	         sock.put(payload)
  	         @modbus_counter += 1
  	         sock.get_once(-1, sock.def_read_timeout)
  	       end
  	     
  	       def make_payload(payload)
  	         packet_data = [@modbus_counter].pack("n")
  	         packet_data += "\x00\x00\x00" #dunno what these are
  	         packet_data += [payload.size].pack("c") # size byte
  	         packet_data += payload
  	     
  	         packet_data
  	       end
  	     
  	       def make_read_payload
  	         payload = [datastore['UNIT_NUMBER']].pack("c")
  	         payload += [@function_code].pack("c")
  	         payload += [datastore['DATA_ADDRESS']].pack("n")
  	         payload += [1].pack("n")
  	         make_payload(payload)
  	       end
  	     
  	       def make_write_coil_payload(data)
  	         payload = [datastore['UNIT_NUMBER']].pack("c")
  	         payload += [@function_code].pack("c")
  	         payload += [datastore['DATA_ADDRESS']].pack("n")
  	         payload += [data].pack("c")
  	         payload += "\x00"
  	     
  	         packet_data = make_payload(payload)
  	     
  	         packet_data
  	       end
  	     
  	       def make_write_register_payload(data)
  	         payload = [datastore['UNIT_NUMBER']].pack("c")
  	         payload += [@function_code].pack("c")
  	         payload += [datastore['DATA_ADDRESS']].pack("n")
  	         payload += [data].pack("n")
  	     
  	         make_payload(payload)
  	       end
  	     
  	       def handle_error(response)
  	         case response.reverse.unpack("c")[0].to_i
  	         when 1
  	           print_error("Error : ILLEGAL FUNCTION")
  	         when 2
  	           print_error("Error : ILLEGAL DATA ADDRESS")
  	         when 3
  	           print_error("Error : ILLEGAL DATA VALUE")
  	         when 4
  	           print_error("Error : SLAVE DEVICE FAILURE")
  	         when 6
  	           print_error("Error : SLAVE DEVICE BUSY")
  	         else
  	           print_error("Unknown error")
  	         end
  	         return
  	       end
  	     
  	       def read_coil
  	         @function_code = 0x1
  	         print_status("Sending READ COIL...")
  	         response = send_frame(make_read_payload)
  	         if response.nil?
  	           print_error("No answer for the READ COIL")
  	           return
  	         elsif response.unpack("C*")[7] == (0x80 | @function_code)
  	           handle_error(response)
  	         elsif response.unpack("C*")[7] == @function_code
  	           value = response[9].unpack("c")[0]
  	           print_good("Coil value at address #{datastore['DATA_ADDRESS']} : #{value}")
  	         else
  	           print_error("Unknown answer")
  	         end
  	       end
  	     
  	       def read_register
  	         @function_code = 3
  	         print_status("Sending READ REGISTER...")
  	         response = send_frame(make_read_payload)
  	         if response.nil?
  	           print_error("No answer for the READ REGISTER")
  	         elsif response.unpack("C*")[7] == (0x80 | @function_code)
  	           handle_error(response)
  	         elsif response.unpack("C*")[7] == @function_code
  	           value = response[9..10].unpack("n")[0]
  	           print_good("Register value at address #{datastore['DATA_ADDRESS']} : #{value}")
  	         else
  	           print_error("Unknown answer")
  	         end
  	       end
  	     
  	       def write_coil
  	         @function_code = 5
  	         if datastore['DATA'] == 0
  	           data = 0
  	         elsif datastore['DATA'] == 1
  	           data = 255
  	         else
  	           print_error("Data value must be 0 or 1 in WRITE_COIL mode")
  	           return
  	         end
  	         print_status("Sending WRITE COIL...")
  	         response = send_frame(make_write_coil_payload(data))
  	         if response.nil?
  	           print_error("No answer for the WRITE COIL")
  	         elsif response.unpack("C*")[7] == (0x80 | @function_code)
  	           handle_error(response)
  	         elsif response.unpack("C*")[7] == @function_code
  	           print_good("Value #{datastore['DATA']} successfully written at coil address #{datastore['DATA_ADDRESS']}")
  	         else
  	           print_error("Unknown answer")
  	         end
  	       end
  	     
  	       def write_register
  	         @function_code = 6
  	         if datastore['DATA'] < 0 || datastore['DATA'] > 65535
  	           print_error("Data to write must be an integer between 0 and 65535 in WRITE_REGISTER mode")
  	           return
  	         end
  	         print_status("Sending WRITE REGISTER...")
  	         response = send_frame(make_write_register_payload(datastore['DATA']))
  	         if response.nil?
  	           print_error("No answer for the WRITE REGISTER")
  	         elsif response.unpack("C*")[7] == (0x80 | @function_code)
  	           handle_error(response)
  	         elsif response.unpack("C*")[7] == @function_code
  	           print_good("Value #{datastore['DATA']} successfully written at registry address #{datastore['DATA_ADDRESS']}")
  	         else
  	           print_error("Unknown answer")
  	         end
  	       end
  	     
  	       def run
  	         @modbus_counter = 0x0000 # used for modbus frames
  	         connect
  	         case action.name
  	         when "READ_COIL"
  	           read_coil
  	         when "READ_REGISTER"
  	           read_register
  	         when "WRITE_COIL"
  	           write_coil
  	         when "WRITE_REGISTER"
  	           write_register
  	         else
  	           print_error("Invalid ACTION")
  	         end
  	         disconnect
  	       end
  	     end
  	     
  	     \end{lstlisting}
  	      
  	  
  	\end{appendix}
 	
 	
 	
    \newpage      
    \bibliographystyle{plain}
      
      
      \bibliography{bibfile}
      
      
      
      \newpage
     
     
      
      \printglossaries     
             
     
      
\end{document}